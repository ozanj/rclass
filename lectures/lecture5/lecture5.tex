\documentclass[8pt,ignorenonframetext,dvipsnames]{beamer}
\setbeamertemplate{caption}[numbered]
\setbeamertemplate{caption label separator}{: }
\setbeamercolor{caption name}{fg=normal text.fg}
\beamertemplatenavigationsymbolsempty
\usepackage{lmodern}
\usepackage{amssymb,amsmath}
\usepackage{ifxetex,ifluatex}
\usepackage{fixltx2e} % provides \textsubscript
\ifnum 0\ifxetex 1\fi\ifluatex 1\fi=0 % if pdftex
  \usepackage[T1]{fontenc}
  \usepackage[utf8]{inputenc}
\else % if luatex or xelatex
  \ifxetex
    \usepackage{mathspec}
  \else
    \usepackage{fontspec}
  \fi
  \defaultfontfeatures{Ligatures=TeX,Scale=MatchLowercase}
\fi
% use upquote if available, for straight quotes in verbatim environments
\IfFileExists{upquote.sty}{\usepackage{upquote}}{}
% use microtype if available
\IfFileExists{microtype.sty}{%
\usepackage{microtype}
\UseMicrotypeSet[protrusion]{basicmath} % disable protrusion for tt fonts
}{}
\newif\ifbibliography
\hypersetup{
            pdftitle={Lecture 5: Augmented vectors and exploratory data analysis},
            pdfauthor={Ozan Jaquette},
            colorlinks=true,
            linkcolor=Maroon,
            citecolor=Blue,
            urlcolor=blue,
            breaklinks=true}
\urlstyle{same}  % don't use monospace font for urls
\usepackage{color}
\usepackage{fancyvrb}
\newcommand{\VerbBar}{|}
\newcommand{\VERB}{\Verb[commandchars=\\\{\}]}
\DefineVerbatimEnvironment{Highlighting}{Verbatim}{commandchars=\\\{\}}
% Add ',fontsize=\small' for more characters per line
\usepackage{framed}
\definecolor{shadecolor}{RGB}{248,248,248}
\newenvironment{Shaded}{\begin{snugshade}}{\end{snugshade}}
\newcommand{\KeywordTok}[1]{\textcolor[rgb]{0.13,0.29,0.53}{\textbf{#1}}}
\newcommand{\DataTypeTok}[1]{\textcolor[rgb]{0.13,0.29,0.53}{#1}}
\newcommand{\DecValTok}[1]{\textcolor[rgb]{0.00,0.00,0.81}{#1}}
\newcommand{\BaseNTok}[1]{\textcolor[rgb]{0.00,0.00,0.81}{#1}}
\newcommand{\FloatTok}[1]{\textcolor[rgb]{0.00,0.00,0.81}{#1}}
\newcommand{\ConstantTok}[1]{\textcolor[rgb]{0.00,0.00,0.00}{#1}}
\newcommand{\CharTok}[1]{\textcolor[rgb]{0.31,0.60,0.02}{#1}}
\newcommand{\SpecialCharTok}[1]{\textcolor[rgb]{0.00,0.00,0.00}{#1}}
\newcommand{\StringTok}[1]{\textcolor[rgb]{0.31,0.60,0.02}{#1}}
\newcommand{\VerbatimStringTok}[1]{\textcolor[rgb]{0.31,0.60,0.02}{#1}}
\newcommand{\SpecialStringTok}[1]{\textcolor[rgb]{0.31,0.60,0.02}{#1}}
\newcommand{\ImportTok}[1]{#1}
\newcommand{\CommentTok}[1]{\textcolor[rgb]{0.56,0.35,0.01}{\textit{#1}}}
\newcommand{\DocumentationTok}[1]{\textcolor[rgb]{0.56,0.35,0.01}{\textbf{\textit{#1}}}}
\newcommand{\AnnotationTok}[1]{\textcolor[rgb]{0.56,0.35,0.01}{\textbf{\textit{#1}}}}
\newcommand{\CommentVarTok}[1]{\textcolor[rgb]{0.56,0.35,0.01}{\textbf{\textit{#1}}}}
\newcommand{\OtherTok}[1]{\textcolor[rgb]{0.56,0.35,0.01}{#1}}
\newcommand{\FunctionTok}[1]{\textcolor[rgb]{0.00,0.00,0.00}{#1}}
\newcommand{\VariableTok}[1]{\textcolor[rgb]{0.00,0.00,0.00}{#1}}
\newcommand{\ControlFlowTok}[1]{\textcolor[rgb]{0.13,0.29,0.53}{\textbf{#1}}}
\newcommand{\OperatorTok}[1]{\textcolor[rgb]{0.81,0.36,0.00}{\textbf{#1}}}
\newcommand{\BuiltInTok}[1]{#1}
\newcommand{\ExtensionTok}[1]{#1}
\newcommand{\PreprocessorTok}[1]{\textcolor[rgb]{0.56,0.35,0.01}{\textit{#1}}}
\newcommand{\AttributeTok}[1]{\textcolor[rgb]{0.77,0.63,0.00}{#1}}
\newcommand{\RegionMarkerTok}[1]{#1}
\newcommand{\InformationTok}[1]{\textcolor[rgb]{0.56,0.35,0.01}{\textbf{\textit{#1}}}}
\newcommand{\WarningTok}[1]{\textcolor[rgb]{0.56,0.35,0.01}{\textbf{\textit{#1}}}}
\newcommand{\AlertTok}[1]{\textcolor[rgb]{0.94,0.16,0.16}{#1}}
\newcommand{\ErrorTok}[1]{\textcolor[rgb]{0.64,0.00,0.00}{\textbf{#1}}}
\newcommand{\NormalTok}[1]{#1}
\usepackage{longtable,booktabs}
\usepackage{caption}
% These lines are needed to make table captions work with longtable:
\makeatletter
\def\fnum@table{\tablename~\thetable}
\makeatother
\usepackage{graphicx,grffile}
\makeatletter
\def\maxwidth{\ifdim\Gin@nat@width>\linewidth\linewidth\else\Gin@nat@width\fi}
\def\maxheight{\ifdim\Gin@nat@height>\textheight0.8\textheight\else\Gin@nat@height\fi}
\makeatother
% Scale images if necessary, so that they will not overflow the page
% margins by default, and it is still possible to overwrite the defaults
% using explicit options in \includegraphics[width, height, ...]{}
\setkeys{Gin}{width=\maxwidth,height=\maxheight,keepaspectratio}

% Prevent slide breaks in the middle of a paragraph:
\widowpenalties 1 10000
\raggedbottom

\AtBeginPart{
  \let\insertpartnumber\relax
  \let\partname\relax
  \frame{\partpage}
}
\AtBeginSection{
  \ifbibliography
  \else
    \let\insertsectionnumber\relax
    \let\sectionname\relax
    \frame{\sectionpage}
  \fi
}
\AtBeginSubsection{
  \let\insertsubsectionnumber\relax
  \let\subsectionname\relax
  \frame{\subsectionpage}
}

\setlength{\parindent}{0pt}
\setlength{\parskip}{6pt plus 2pt minus 1pt}
\setlength{\emergencystretch}{3em}  % prevent overfull lines
\providecommand{\tightlist}{%
  \setlength{\itemsep}{0pt}\setlength{\parskip}{0pt}}
\setcounter{secnumdepth}{0}

%packages
\usepackage{graphicx}
\usepackage{rotating}
\usepackage{hyperref}

\usepackage{tikz} % used for text highlighting, amongst others
%title slide stuff
%\institute{Department of Education}
%\title{Managing and Manipulating Data Using R}

%
\setbeamertemplate{navigation symbols}{} % get rid of navigation icons:
\setbeamertemplate{footline}[page number]

%\setbeamertemplate{frametitle}{\thesection \hspace{0.2cm} \insertframetitle}
\setbeamertemplate{section in toc}[sections numbered]
%\setbeamertemplate{subsection in toc}[subsections numbered]
\setbeamertemplate{subsection in toc}{%
  \leavevmode\leftskip=3.2em\color{gray}\rlap{\hskip-2em\inserttocsectionnumber.\inserttocsubsectionnumber}\inserttocsubsection\par
}

%define colors
%\definecolor{uva_orange}{RGB}{216,141,42} % UVa orange (Rotunda orange)
\definecolor{mygray}{rgb}{0.95, 0.95, 0.95} % for highlighted text
	% grey is equal parts red, green, blue. higher values >> lighter grey
	%\definecolor{lightgraybo}{rgb}{0.83, 0.83, 0.83}

% new commands

%highlight text with very light grey
\newcommand*{\hlg}[1]{%
	\tikz[baseline=(X.base)] \node[rectangle, fill=mygray] (X) {#1};%
}
%, inner sep=0.3mm
%highlight text with very light grey and use font associated with code
\newcommand*{\hlgc}[1]{\texttt{\hlg{#1}}}

% Font
\usepackage[defaultfam,light,tabular,lining]{montserrat}
\usepackage[T1]{fontenc}
\renewcommand*\oldstylenums[1]{{\fontfamily{Montserrat-TOsF}\selectfont #1}}

% Change color of boldface text to darkgray
\renewcommand{\textbf}[1]{{\color{darkgray}\bfseries\fontfamily{Montserrat-TOsF}#1}}

% Bullet points
\setbeamertemplate{itemize item}{\color{BlueViolet}$\circ$}
\setbeamertemplate{itemize subitem}{\color{BrickRed}$\triangleright$}
\setbeamertemplate{itemize subsubitem}{$-$}

% Reduce space before lists
\addtobeamertemplate{itemize/enumerate body begin}{}{\vspace*{-8pt}}

\let\olditem\item
\renewcommand{\item}{%
  \olditem\vspace{4pt}
}

% decreasing space before and after level-2 bullet block
\addtobeamertemplate{itemize/enumerate subbody begin}{}{\vspace*{-3pt}}
\addtobeamertemplate{itemize/enumerate subbody end}{}{\vspace*{-3pt}}

% decreasing space before and after level-3 bullet block
\addtobeamertemplate{itemize/enumerate subsubbody begin}{}{\vspace*{-2pt}}
\addtobeamertemplate{itemize/enumerate subsubbody end}{}{\vspace*{-2pt}}

%Section numbering
\setbeamertemplate{section page}{%
    \begingroup
        \begin{beamercolorbox}[sep=10pt,center,rounded=true,shadow=true]{section title}
        \usebeamerfont{section title}\thesection~\insertsection\par
        \end{beamercolorbox}
    \endgroup
}

\setbeamertemplate{subsection page}{%
    \begingroup
        \begin{beamercolorbox}[sep=6pt,center,rounded=true,shadow=true]{subsection title}
        \usebeamerfont{subsection title}\thesection.\thesubsection~\insertsubsection\par
        \end{beamercolorbox}
    \endgroup
}

%modifying back ticks to add grey background
\let\OldTexttt\texttt
\renewcommand{\texttt}[1]{\OldTexttt{\hlg{#1}}}

\title{Lecture 5: Augmented vectors and exploratory data analysis}
\subtitle{EDUC 263: Managing and Manipulating Data Using R}
\author{Ozan Jaquette}
\date{}

\begin{document}
\frame{\titlepage}

\section{Introduction}\label{introduction}

\begin{frame}[fragile]{Logistics}

\textbf{Reading to do before next class:}

\begin{itemize}
\tightlist
\item
  Work through slides from lecture 5 that we don't get to in class
\item
  GW 15.1 - 15.2 (factors) {[}this is like 2-3 pages{]}
\item
  {[}OPTIONAL{]} GW 15.3 - 15.5 (remainder of ``factors'' chapter)
\item
  {[}OPTIONAL{]} GW 20.6 - 20.7 (attributes and augmented vectors)
\item
  {[}OPTIONAL{]} GW 10 (tibbles)
\end{itemize}

\textbf{Explanation about \texttt{beamer\_header.tex} in YAML header:}

\begin{itemize}
\tightlist
\item
  We are calling the beamer\_header.tex file in the background to
  customize our slides. Without this LaTeX file, our slides would
  compile according to the default beamer presentation (PDF).

  \begin{itemize}
  \tightlist
  \item
    Why would we want to do this?\\
  \item
    We can customize our slides with the beamer\_header.tex LaTeX file
    to include page numbers, change heading options, or change slide
    colors (in addition to other things).
  \end{itemize}
\item
  \texttt{includes} option in the YAML header customizes the beamer
  presentation slides

  \begin{itemize}
  \tightlist
  \item
    Here is a
    \href{https://bookdown.org/yihui/rmarkdown/pdf-document.html\#latex-options}{link}
    to a short description of the includes option in the YAML header.
  \end{itemize}
\end{itemize}

\end{frame}

\begin{frame}{What we will do today}

\tableofcontents

\end{frame}

\begin{frame}[fragile]{Libraries we will use today}

``Load'' the package we will use today (output omitted)

\begin{itemize}
\tightlist
\item
  \textbf{you must run this code chunk after installing these packages}
\end{itemize}

\begin{Shaded}
\begin{Highlighting}[]
\KeywordTok{library}\NormalTok{(tidyverse)}
\KeywordTok{library}\NormalTok{(haven)}
\KeywordTok{library}\NormalTok{(labelled)}
\end{Highlighting}
\end{Shaded}

\textbf{If package not yet installed}, then must install before you
load. Install in ``console'' rather than .Rmd file

\begin{itemize}
\tightlist
\item
  Generic syntax: \texttt{install.packages("package\_name")}
\item
  Install ``tidyverse'': \texttt{install.packages("tidyverse")}
\end{itemize}

Note: when we load package, name of package is not in quotes; but when
we install package, name of package is in quotes:

\begin{itemize}
\tightlist
\item
  \texttt{install.packages("tidyverse")}
\item
  \texttt{library(tidyverse)}
\end{itemize}

\end{frame}

\section{Augmented vectors}\label{augmented-vectors}

\begin{frame}[fragile]{Data we will use to introduce augmented vectors}

\begin{Shaded}
\begin{Highlighting}[]
\KeywordTok{rm}\NormalTok{(}\DataTypeTok{list =} \KeywordTok{ls}\NormalTok{()) }\CommentTok{# remove all objects}

\CommentTok{#load("../../data/prospect_list/western_washington_college_board_list.RData")}
\KeywordTok{load}\NormalTok{(}\KeywordTok{url}\NormalTok{(}\StringTok{"https://github.com/ozanj/rclass/raw/master/data/prospect_list/wwlist_merged.RData"}\NormalTok{))}
\end{Highlighting}
\end{Shaded}

\end{frame}

\subsection{Review data types and
structures}\label{review-data-types-and-structures}

\begin{frame}{\textbf{Vectors} are the primary data structures in R}

\medskip

Two types of vectors:

\begin{enumerate}
\def\labelenumi{\arabic{enumi}.}
\tightlist
\item
  \textbf{atomic vectors}
\item
  \textbf{lists}
\end{enumerate}

\medskip

\begin{figure}
\centering
\includegraphics[width=0.60000\textwidth]{data-structures-overview.png}
\caption{Overview of data structures (Grolemund and Wickham, 2018)}
\end{figure}

\end{frame}

\begin{frame}[fragile]{Review data structures: atomic vectors}

\medskip An \textbf{atomic vector} is a collection of values

\begin{itemize}
\tightlist
\item
  each value in an atomic vector is an \textbf{element}
\item
  all elements within vector must have same \textbf{data type}
\end{itemize}

\begin{Shaded}
\begin{Highlighting}[]
\NormalTok{(a <-}\StringTok{ }\KeywordTok{c}\NormalTok{(}\DecValTok{1}\NormalTok{,}\DecValTok{2}\NormalTok{,}\DecValTok{3}\NormalTok{)) }\CommentTok{# parentheses () assign and print object in one step}
\CommentTok{#> [1] 1 2 3}
\KeywordTok{length}\NormalTok{(a)}
\CommentTok{#> [1] 3}
\KeywordTok{typeof}\NormalTok{(a)}
\CommentTok{#> [1] "double"}
\KeywordTok{str}\NormalTok{(a)}
\CommentTok{#>  num [1:3] 1 2 3}
\end{Highlighting}
\end{Shaded}

Can assign \textbf{names} to vector elements, creating a \textbf{named
atomic vector}

\begin{Shaded}
\begin{Highlighting}[]
\NormalTok{(b <-}\StringTok{ }\KeywordTok{c}\NormalTok{(}\DataTypeTok{v1=}\DecValTok{1}\NormalTok{,}\DataTypeTok{v2=}\DecValTok{2}\NormalTok{,}\DataTypeTok{v3=}\DecValTok{3}\NormalTok{))}
\CommentTok{#> v1 v2 v3 }
\CommentTok{#>  1  2  3}
\KeywordTok{length}\NormalTok{(b)}
\CommentTok{#> [1] 3}
\KeywordTok{typeof}\NormalTok{(b)}
\CommentTok{#> [1] "double"}
\KeywordTok{str}\NormalTok{(b)}
\CommentTok{#>  Named num [1:3] 1 2 3}
\CommentTok{#>  - attr(*, "names")= chr [1:3] "v1" "v2" "v3"}
\end{Highlighting}
\end{Shaded}

\end{frame}

\begin{frame}[fragile]{Review data structures: lists}

\medskip

\begin{itemize}
\tightlist
\item
  Like atomic vectors, \textbf{lists} are objects that contain
  \textbf{elements}
\item
  However, \textbf{data type} can differ across elements within a list

  \begin{itemize}
  \tightlist
  \item
    an element of a list can be another list
  \end{itemize}
\end{itemize}

\begin{Shaded}
\begin{Highlighting}[]
\NormalTok{list_a <-}\StringTok{ }\KeywordTok{list}\NormalTok{(}\DecValTok{1}\NormalTok{,}\DecValTok{2}\NormalTok{,}\StringTok{"apple"}\NormalTok{)}
\KeywordTok{typeof}\NormalTok{(list_a)}
\CommentTok{#> [1] "list"}
\KeywordTok{length}\NormalTok{(list_a)}
\CommentTok{#> [1] 3}
\KeywordTok{str}\NormalTok{(list_a)}
\CommentTok{#> List of 3}
\CommentTok{#>  $ : num 1}
\CommentTok{#>  $ : num 2}
\CommentTok{#>  $ : chr "apple"}

\NormalTok{list_b <-}\StringTok{ }\KeywordTok{list}\NormalTok{(}\DecValTok{1}\NormalTok{, }\KeywordTok{c}\NormalTok{(}\StringTok{"apple"}\NormalTok{, }\StringTok{"orange"}\NormalTok{), }\KeywordTok{list}\NormalTok{(}\DecValTok{1}\NormalTok{, }\DecValTok{2}\NormalTok{))}
\KeywordTok{length}\NormalTok{(list_b)}
\CommentTok{#> [1] 3}
\KeywordTok{str}\NormalTok{(list_b)}
\CommentTok{#> List of 3}
\CommentTok{#>  $ : num 1}
\CommentTok{#>  $ : chr [1:2] "apple" "orange"}
\CommentTok{#>  $ :List of 2}
\CommentTok{#>   ..$ : num 1}
\CommentTok{#>   ..$ : num 2}
\end{Highlighting}
\end{Shaded}

\end{frame}

\begin{frame}[fragile]{Review data structures: lists}

Like atomic vectors, elements within a list can be named, thereby
creating a \textbf{named list}

\begin{Shaded}
\begin{Highlighting}[]
\CommentTok{# not named}
\KeywordTok{str}\NormalTok{(list_b) }
\CommentTok{#> List of 3}
\CommentTok{#>  $ : num 1}
\CommentTok{#>  $ : chr [1:2] "apple" "orange"}
\CommentTok{#>  $ :List of 2}
\CommentTok{#>   ..$ : num 1}
\CommentTok{#>   ..$ : num 2}

\CommentTok{# named}
\NormalTok{list_c <-}\StringTok{ }\KeywordTok{list}\NormalTok{(}\DataTypeTok{v1=}\DecValTok{1}\NormalTok{, }\DataTypeTok{v2=}\KeywordTok{c}\NormalTok{(}\StringTok{"apple"}\NormalTok{, }\StringTok{"orange"}\NormalTok{), }\DataTypeTok{v3=}\KeywordTok{list}\NormalTok{(}\DecValTok{1}\NormalTok{, }\DecValTok{2}\NormalTok{, }\DecValTok{3}\NormalTok{))}
\KeywordTok{str}\NormalTok{(list_c) }
\CommentTok{#> List of 3}
\CommentTok{#>  $ v1: num 1}
\CommentTok{#>  $ v2: chr [1:2] "apple" "orange"}
\CommentTok{#>  $ v3:List of 3}
\CommentTok{#>   ..$ : num 1}
\CommentTok{#>   ..$ : num 2}
\CommentTok{#>   ..$ : num 3}
\end{Highlighting}
\end{Shaded}

\end{frame}

\begin{frame}[fragile]{Review data structures: a data frame is a list}

A \textbf{data frame} is a list with the following characteristics:

\begin{itemize}
\tightlist
\item
  All the elements must be \textbf{vectors} with the same
  \textbf{length}
\item
  Data frames are \textbf{augmented lists} because they have additional
  \textbf{attributes} {[}described later{]}
\end{itemize}

\begin{Shaded}
\begin{Highlighting}[]
\CommentTok{#a regular list}
\NormalTok{list_d <-}\StringTok{ }\KeywordTok{list}\NormalTok{(}\DataTypeTok{col_a =} \KeywordTok{c}\NormalTok{(}\DecValTok{1}\NormalTok{,}\DecValTok{2}\NormalTok{,}\DecValTok{3}\NormalTok{), }\DataTypeTok{col_b =} \KeywordTok{c}\NormalTok{(}\DecValTok{4}\NormalTok{,}\DecValTok{5}\NormalTok{,}\DecValTok{6}\NormalTok{), }\DataTypeTok{col_c =} \KeywordTok{c}\NormalTok{(}\DecValTok{7}\NormalTok{,}\DecValTok{8}\NormalTok{,}\DecValTok{9}\NormalTok{))}
\KeywordTok{typeof}\NormalTok{(list_d)}
\CommentTok{#> [1] "list"}
\KeywordTok{str}\NormalTok{(list_d)}
\CommentTok{#> List of 3}
\CommentTok{#>  $ col_a: num [1:3] 1 2 3}
\CommentTok{#>  $ col_b: num [1:3] 4 5 6}
\CommentTok{#>  $ col_c: num [1:3] 7 8 9}

\CommentTok{#a data frame}
\NormalTok{df_a <-}\StringTok{ }\KeywordTok{data.frame}\NormalTok{(}\DataTypeTok{col_a =} \KeywordTok{c}\NormalTok{(}\DecValTok{1}\NormalTok{,}\DecValTok{2}\NormalTok{,}\DecValTok{3}\NormalTok{), }\DataTypeTok{col_b =} \KeywordTok{c}\NormalTok{(}\DecValTok{4}\NormalTok{,}\DecValTok{5}\NormalTok{,}\DecValTok{6}\NormalTok{), }\DataTypeTok{col_c =} \KeywordTok{c}\NormalTok{(}\DecValTok{7}\NormalTok{,}\DecValTok{8}\NormalTok{,}\DecValTok{9}\NormalTok{))}
\KeywordTok{typeof}\NormalTok{(df_a)}
\CommentTok{#> [1] "list"}
\KeywordTok{str}\NormalTok{(df_a)}
\CommentTok{#> 'data.frame':    3 obs. of  3 variables:}
\CommentTok{#>  $ col_a: num  1 2 3}
\CommentTok{#>  $ col_b: num  4 5 6}
\CommentTok{#>  $ col_c: num  7 8 9}
\end{Highlighting}
\end{Shaded}

\end{frame}

\subsection{Attributes and augmented
vectors}\label{attributes-and-augmented-vectors}

\begin{frame}{Atomic vectors versus augmented vectors}

\textbf{Atomic vectors} {[}our focus so far{]}

\begin{itemize}
\tightlist
\item
  I think of atomic vectors as ``just the data''
\item
  Atomic vectors are the building blocks for augmented vectors
\end{itemize}

\textbf{Augmented vectors}

\begin{itemize}
\tightlist
\item
  \textbf{Augmented vectors} are atomic vectors with additional
  \textbf{attributes} attached
\end{itemize}

\textbf{Attributes}

\begin{itemize}
\tightlist
\item
  \textbf{Attributes} are additional ``metadata'' that can be attached
  to any object (e.g., vector or list)
\item
  Examples of some important attributes in R:

  \begin{itemize}
  \tightlist
  \item
    \textbf{Names}: name the elements of a vector (e.g., variable names)
  \item
    \textbf{value labels}: character labels (e.g., ``Charter School'')
    attached to numeric values
  \item
    \textbf{Object class}: How object should be treated by object
    oriented programming language {[}discussed below{]}
  \end{itemize}
\end{itemize}

\textbf{Main takaway}:

\begin{itemize}
\tightlist
\item
  Augmented vectors are atomic vectors (just the data) with additional
  attributes attached
\end{itemize}

\end{frame}

\begin{frame}[fragile]{Attributes in vectors}

Identify attributes in any object using the \texttt{attributes()}
function

\begin{Shaded}
\begin{Highlighting}[]
\CommentTok{#vector with no attributes}
\NormalTok{vector1 <-}\StringTok{ }\KeywordTok{c}\NormalTok{(}\DecValTok{1}\NormalTok{,}\DecValTok{2}\NormalTok{,}\DecValTok{3}\NormalTok{,}\DecValTok{4}\NormalTok{)}
\NormalTok{vector1}
\CommentTok{#> [1] 1 2 3 4}
\KeywordTok{attributes}\NormalTok{(vector1)}
\CommentTok{#> NULL}

\CommentTok{#vector with attributes}
\NormalTok{vector2 <-}\StringTok{ }\KeywordTok{c}\NormalTok{(}\DataTypeTok{a =} \DecValTok{1}\NormalTok{, }\DataTypeTok{b=} \DecValTok{2}\NormalTok{, }\DataTypeTok{c=} \DecValTok{3}\NormalTok{, }\DataTypeTok{d =} \DecValTok{4}\NormalTok{)}
\NormalTok{vector2}
\CommentTok{#> a b c d }
\CommentTok{#> 1 2 3 4}
\KeywordTok{attributes}\NormalTok{(vector2)}
\CommentTok{#> $names}
\CommentTok{#> [1] "a" "b" "c" "d"}
\end{Highlighting}
\end{Shaded}

\end{frame}

\begin{frame}[fragile]{Attributes in lists}

\begin{Shaded}
\begin{Highlighting}[]
\CommentTok{#no attributes}
\NormalTok{list1 <-}\StringTok{ }\KeywordTok{list}\NormalTok{(}\KeywordTok{c}\NormalTok{(}\DecValTok{1}\NormalTok{,}\DecValTok{2}\NormalTok{,}\DecValTok{3}\NormalTok{), }\KeywordTok{c}\NormalTok{(}\DecValTok{4}\NormalTok{,}\DecValTok{5}\NormalTok{,}\DecValTok{6}\NormalTok{))}
\KeywordTok{attributes}\NormalTok{(list1)}
\CommentTok{#> NULL}

\CommentTok{#list with attributes}
\NormalTok{list2 <-}\StringTok{ }\KeywordTok{list}\NormalTok{(}\DataTypeTok{col_a =} \KeywordTok{c}\NormalTok{(}\DecValTok{1}\NormalTok{,}\DecValTok{2}\NormalTok{,}\DecValTok{3}\NormalTok{), }\DataTypeTok{col_b =} \KeywordTok{c}\NormalTok{(}\DecValTok{4}\NormalTok{,}\DecValTok{5}\NormalTok{,}\DecValTok{6}\NormalTok{))}
\KeywordTok{str}\NormalTok{(list2)}
\CommentTok{#> List of 2}
\CommentTok{#>  $ col_a: num [1:3] 1 2 3}
\CommentTok{#>  $ col_b: num [1:3] 4 5 6}
\KeywordTok{attributes}\NormalTok{(list2)}
\CommentTok{#> $names}
\CommentTok{#> [1] "col_a" "col_b"}

\CommentTok{#data frame with attributes}
\NormalTok{list3 <-}\StringTok{ }\KeywordTok{data.frame}\NormalTok{(}\DataTypeTok{col_a =} \KeywordTok{c}\NormalTok{(}\DecValTok{1}\NormalTok{,}\DecValTok{2}\NormalTok{,}\DecValTok{3}\NormalTok{), }\DataTypeTok{col_b =} \KeywordTok{c}\NormalTok{(}\DecValTok{4}\NormalTok{,}\DecValTok{5}\NormalTok{,}\DecValTok{6}\NormalTok{))}
\KeywordTok{str}\NormalTok{(list3)}
\CommentTok{#> 'data.frame':    3 obs. of  2 variables:}
\CommentTok{#>  $ col_a: num  1 2 3}
\CommentTok{#>  $ col_b: num  4 5 6}
\KeywordTok{attributes}\NormalTok{(list3)}
\CommentTok{#> $names}
\CommentTok{#> [1] "col_a" "col_b"}
\CommentTok{#> }
\CommentTok{#> $class}
\CommentTok{#> [1] "data.frame"}
\CommentTok{#> }
\CommentTok{#> $row.names}
\CommentTok{#> [1] 1 2 3}
\end{Highlighting}
\end{Shaded}

\end{frame}

\subsection{Object class}\label{object-class}

\begin{frame}[fragile]{Object class}

\medskip  Every object in R has a \textbf{class}

\begin{itemize}
\tightlist
\item
  Object class defines rules for how object can be treated by object
  oriented programming language (e.g., which functions you can apply to
  object)
\item
  class is an \textbf{attribute} of an object
\end{itemize}

Identify the class of an object using the \texttt{class()} function

\begin{Shaded}
\begin{Highlighting}[]
\NormalTok{(vector2 <-}\StringTok{ }\KeywordTok{c}\NormalTok{(}\DataTypeTok{a =} \DecValTok{1}\NormalTok{, }\DataTypeTok{b=} \DecValTok{2}\NormalTok{, }\DataTypeTok{c=} \DecValTok{3}\NormalTok{, }\DataTypeTok{d =} \DecValTok{4}\NormalTok{))}
\CommentTok{#> a b c d }
\CommentTok{#> 1 2 3 4}
\KeywordTok{class}\NormalTok{(vector2)}
\CommentTok{#> [1] "numeric"}
\end{Highlighting}
\end{Shaded}

When I encounter a new object I often investigate object by applying
\texttt{typeof()}, \texttt{class()}, and \texttt{attributes()} functions
to that object

\begin{Shaded}
\begin{Highlighting}[]
\NormalTok{vector2}
\CommentTok{#> a b c d }
\CommentTok{#> 1 2 3 4}
\KeywordTok{typeof}\NormalTok{(vector2)}
\CommentTok{#> [1] "double"}
\KeywordTok{class}\NormalTok{(vector2)}
\CommentTok{#> [1] "numeric"}
\KeywordTok{attributes}\NormalTok{(vector2)}
\CommentTok{#> $names}
\CommentTok{#> [1] "a" "b" "c" "d"}
\end{Highlighting}
\end{Shaded}

\end{frame}

\begin{frame}[fragile]{Object class}

Why is \textbf{class} important?

\begin{itemize}
\tightlist
\item
  Specific functions usually work with only particular \textbf{classes}
  of objects

  \begin{itemize}
  \tightlist
  \item
    e.g., ``date''" functions usually only work on objects with a date
    class
  \item
    ``string'' functions usually only work with on objects with a
    character class
  \item
    Functions that do mathematical computation usually work on objects
    with a numeric class
  \end{itemize}
\item
  Note: functions care about object \textbf{class}, not object
  \textbf{type}
\end{itemize}

object with \texttt{numeric} class (output omitted)

\begin{Shaded}
\begin{Highlighting}[]
\KeywordTok{str}\NormalTok{(wwlist)}

\KeywordTok{typeof}\NormalTok{(wwlist}\OperatorTok{$}\NormalTok{med_inc_zip)}
\KeywordTok{class}\NormalTok{(wwlist}\OperatorTok{$}\NormalTok{med_inc_zip)}
\KeywordTok{sum}\NormalTok{(wwlist}\OperatorTok{$}\NormalTok{med_inc_zip[}\DecValTok{1}\OperatorTok{:}\DecValTok{10}\NormalTok{], }\DataTypeTok{na.rm =} \OtherTok{TRUE}\NormalTok{) }\CommentTok{# numeric function}

\CommentTok{# load library with date functions}
\KeywordTok{library}\NormalTok{(lubridate) }
\CommentTok{#Sys.setenv(TZ="America/Los_Angeles") #setting time zone to Los Angeles time}
\KeywordTok{year}\NormalTok{(wwlist}\OperatorTok{$}\NormalTok{receive_date[}\DecValTok{1}\OperatorTok{:}\DecValTok{10}\NormalTok{]) }\CommentTok{# date function}
\end{Highlighting}
\end{Shaded}

\end{frame}

\begin{frame}[fragile]{Object class}

Why is \textbf{class} important?

\begin{itemize}
\tightlist
\item
  Specific functions usually work with only particular \textbf{classes}
  of objects
\item
  Note: functions care about object \textbf{class}, not object
  \textbf{type}
\end{itemize}

Object with \texttt{character} class

\begin{Shaded}
\begin{Highlighting}[]
\KeywordTok{str}\NormalTok{(wwlist}\OperatorTok{$}\NormalTok{hs_city)}
\KeywordTok{typeof}\NormalTok{(wwlist}\OperatorTok{$}\NormalTok{hs_city)}
\KeywordTok{class}\NormalTok{(wwlist}\OperatorTok{$}\NormalTok{hs_city)}

\KeywordTok{tolower}\NormalTok{(wwlist}\OperatorTok{$}\NormalTok{hs_city[}\DecValTok{1}\OperatorTok{:}\DecValTok{10}\NormalTok{]) }\CommentTok{# string function}
\KeywordTok{sum}\NormalTok{(wwlist}\OperatorTok{$}\NormalTok{hs_city, }\DataTypeTok{na.rm =} \OtherTok{TRUE}\NormalTok{) }\CommentTok{# numeric function}
\end{Highlighting}
\end{Shaded}

Object with a date class

\begin{Shaded}
\begin{Highlighting}[]
\KeywordTok{typeof}\NormalTok{(wwlist}\OperatorTok{$}\NormalTok{receive_date)}
\KeywordTok{class}\NormalTok{(wwlist}\OperatorTok{$}\NormalTok{receive_date)}

\KeywordTok{year}\NormalTok{(wwlist}\OperatorTok{$}\NormalTok{receive_date[}\DecValTok{1}\OperatorTok{:}\DecValTok{10}\NormalTok{]) }\CommentTok{# date function}
\KeywordTok{sum}\NormalTok{(wwlist}\OperatorTok{$}\NormalTok{receive_date) }\CommentTok{# numeric function}
\end{Highlighting}
\end{Shaded}

\end{frame}

\begin{frame}{Class and object oriented programming}

Definition of object oriented programming from this
\href{https://www.webopedia.com/TERM/O/object_oriented_programming_OOP.html}{LINK}

\begin{quote}
``Object-oriented programming (OOP) refers to a type of computer
programming in which programmers define not only the data type of a data
structure, but also the types of operations (functions) that can be
applied to the data structure.''
\end{quote}

Object \textbf{class} is fundamental to object oriented programming
because:

\begin{itemize}
\tightlist
\item
  object class determines which functions can be applied to the object
\item
  object class also determines what those functions do to the object
\end{itemize}

Many different object classes exist in R

\begin{itemize}
\tightlist
\item
  we can also create our own classes
\item
  but in this course we will work with classes that have been created by
  others
\end{itemize}

\end{frame}

\subsection{Class == factor}\label{class-factor}

\begin{frame}[fragile]{Factors}

\textbf{Factors} are an object \emph{class} used to display categorical
data (e.g., marital status)

\begin{itemize}
\tightlist
\item
  A factor is an \textbf{augmented vector} built by attaching a
  ``levels'' attribute to an (atomic) integer vectors
\end{itemize}

Usually, we would prefer a categorical variable (e.g., race, school
type) to be a factor variable rather than a character variable

\begin{itemize}
\tightlist
\item
  So far in the course I have made all categorical variables character
  variables because we had not introduced factors yet
\end{itemize}

Below, I'll create a factor version of the character variable
\texttt{ethn\_code}

\begin{itemize}
\tightlist
\item
  (don't worry about understanding this code; I'll explain it later)
\end{itemize}

\begin{Shaded}
\begin{Highlighting}[]
\KeywordTok{str}\NormalTok{(wwlist}\OperatorTok{$}\NormalTok{ethn_code)}
\CommentTok{#>  chr [1:268396] "other-2 or more" "white" "white" "other-2 or more" ...}
\CommentTok{# create factor var; tidyverse approach}
\NormalTok{wwlist <-}\StringTok{ }\NormalTok{wwlist }\OperatorTok\StringTok{ }\KeywordTok{mutate}\NormalTok{(}\DataTypeTok{ethn_code_fac =} \KeywordTok{factor}\NormalTok{(ethn_code)) }
\CommentTok{#wwlist$ethn_code_fac <- factor(wwlist$ethn_code) # base r approach}

\KeywordTok{str}\NormalTok{(wwlist}\OperatorTok{$}\NormalTok{ethn_code_fac)}
\CommentTok{#>  Factor w/ 10 levels "american indian or alaska native",..: 7 10 10 7 10 7 7 7 7 10 ...}
\end{Highlighting}
\end{Shaded}

\end{frame}

\begin{frame}[fragile]{Factors}

A factor is an \textbf{augmented vector} built by attaching a ``levels''
attribute to an (atomic) integer vector

Compare (character) \texttt{ethn\_code} to (factor)
\texttt{ethn\_code\_fac} (output omitted)

\begin{Shaded}
\begin{Highlighting}[]
\CommentTok{#character var}
\KeywordTok{typeof}\NormalTok{(wwlist}\OperatorTok{$}\NormalTok{ethn_code)}
\KeywordTok{class}\NormalTok{(wwlist}\OperatorTok{$}\NormalTok{ethn_code)}
\KeywordTok{str}\NormalTok{(wwlist}\OperatorTok{$}\NormalTok{ethn_code)}
\KeywordTok{attributes}\NormalTok{(wwlist}\OperatorTok{$}\NormalTok{ethn_code)}

\CommentTok{#factor var}
\KeywordTok{typeof}\NormalTok{(wwlist}\OperatorTok{$}\NormalTok{ethn_code_fac)}
\KeywordTok{class}\NormalTok{(wwlist}\OperatorTok{$}\NormalTok{ethn_code_fac)}
\KeywordTok{str}\NormalTok{(wwlist}\OperatorTok{$}\NormalTok{ethn_code_fac)}
\KeywordTok{attributes}\NormalTok{(wwlist}\OperatorTok{$}\NormalTok{ethn_code_fac)}
\end{Highlighting}
\end{Shaded}

\textbf{Main takeaway}

\begin{itemize}
\tightlist
\item
  \texttt{ethn\_code\_fac} has \texttt{type=integer} and
  \texttt{class=factor} because the variable has a ``levels'' attribute
\item
  Underlying data are integers but levels attribute is used to display
  the data.
\end{itemize}

\begin{Shaded}
\begin{Highlighting}[]
\NormalTok{wwlist}\OperatorTok{$}\NormalTok{ethn_code_fac[}\DecValTok{1}\OperatorTok{:}\DecValTok{4}\NormalTok{] }\CommentTok{# print first few obs of ethn_code_fac}
\CommentTok{#> [1] other-2 or more white           white           other-2 or more}
\CommentTok{#> 10 Levels: american indian or alaska native ...}
\end{Highlighting}
\end{Shaded}

\end{frame}

\begin{frame}[fragile]{Working with factor variables}

\begin{Shaded}
\begin{Highlighting}[]
\KeywordTok{attributes}\NormalTok{(wwlist}\OperatorTok{$}\NormalTok{ethn_code_fac)}
\end{Highlighting}
\end{Shaded}

Refer to categories of a factor by the values of the \textbf{level
attribute} rather than the underlying values of the variable

\textbf{Task}

\begin{itemize}
\tightlist
\item
  count the number of prospects in object \texttt{wwlist} who identify
  as ``white''
\end{itemize}

\begin{Shaded}
\begin{Highlighting}[]
\CommentTok{# referring to variable value; this doesn't work}
\NormalTok{wwlist }\OperatorTok\StringTok{ }\KeywordTok{filter}\NormalTok{(ethn_code_fac}\OperatorTok{==}\DecValTok{10}\NormalTok{) }\OperatorTok\StringTok{ }\NormalTok{count }
\CommentTok{#> # A tibble: 1 x 1}
\CommentTok{#>       n}
\CommentTok{#>   <int>}
\CommentTok{#> 1     0}

\CommentTok{#referring to value of level attribute; this works}
\NormalTok{wwlist }\OperatorTok\StringTok{ }\KeywordTok{filter}\NormalTok{(ethn_code_fac}\OperatorTok{==}\StringTok{"white"}\NormalTok{) }\OperatorTok\StringTok{ }\NormalTok{count }
\CommentTok{#> # A tibble: 1 x 1}
\CommentTok{#>        n}
\CommentTok{#>    <int>}
\CommentTok{#> 1 159680}
\end{Highlighting}
\end{Shaded}

\end{frame}

\begin{frame}[fragile]{Working with factor variables}

\textbf{Task}

\begin{itemize}
\tightlist
\item
  count the number of prospects in object \texttt{wwlist} who identify
  as ``white''
\end{itemize}

If you want to refer to underlying values, then apply
\texttt{as.integer()} function to the factor variable

\begin{Shaded}
\begin{Highlighting}[]
\KeywordTok{attributes}\NormalTok{(wwlist}\OperatorTok{$}\NormalTok{ethn_code_fac)}
\CommentTok{#> $levels}
\CommentTok{#>  [1] "american indian or alaska native"                  }
\CommentTok{#>  [2] "asian or native hawaiian or other pacific islander"}
\CommentTok{#>  [3] "black or african american"                         }
\CommentTok{#>  [4] "cuban"                                             }
\CommentTok{#>  [5] "mexican/mexican american"                          }
\CommentTok{#>  [6] "not reported"                                      }
\CommentTok{#>  [7] "other-2 or more"                                   }
\CommentTok{#>  [8] "other spanish/hispanic"                            }
\CommentTok{#>  [9] "puerto rican"                                      }
\CommentTok{#> [10] "white"                                             }
\CommentTok{#> }
\CommentTok{#> $class}
\CommentTok{#> [1] "factor"}
\NormalTok{wwlist }\OperatorTok\StringTok{ }\KeywordTok{filter}\NormalTok{(}\KeywordTok{as.integer}\NormalTok{(ethn_code_fac)}\OperatorTok{==}\DecValTok{10}\NormalTok{) }\OperatorTok\StringTok{ }\NormalTok{count }
\CommentTok{#> # A tibble: 1 x 1}
\CommentTok{#>        n}
\CommentTok{#>    <int>}
\CommentTok{#> 1 159680}
\end{Highlighting}
\end{Shaded}

\end{frame}

\begin{frame}[fragile]{How to identify the variable values associated
with factor levels}

Let's create a factor version of the character variable
\texttt{psat\_range}

\begin{Shaded}
\begin{Highlighting}[]
\NormalTok{wwlist <-}\StringTok{ }\NormalTok{wwlist }\OperatorTok\StringTok{ }\KeywordTok{mutate}\NormalTok{(}\DataTypeTok{psat_range_fac =} \KeywordTok{factor}\NormalTok{(psat_range)) }\CommentTok{# create factor var;}
\end{Highlighting}
\end{Shaded}

Run below code in console rather than code chunk to see values
associated with each factor

\begin{Shaded}
\begin{Highlighting}[]
\NormalTok{wwlist }\OperatorTok\StringTok{ }\KeywordTok{count}\NormalTok{(psat_range_fac)}
\end{Highlighting}
\end{Shaded}

Once you know values associated with factor, you can filter based on
values

\begin{Shaded}
\begin{Highlighting}[]
\NormalTok{wwlist }\OperatorTok\StringTok{ }\KeywordTok{filter}\NormalTok{(}\KeywordTok{as.integer}\NormalTok{(psat_range_fac)}\OperatorTok{==}\DecValTok{4}\NormalTok{) }\OperatorTok\StringTok{ }\KeywordTok{count}\NormalTok{()}
\CommentTok{#> # A tibble: 1 x 1}
\CommentTok{#>       n}
\CommentTok{#>   <int>}
\CommentTok{#> 1  8348}
\end{Highlighting}
\end{Shaded}

Or you can just filter based on value of \textbf{factor levels}

\begin{Shaded}
\begin{Highlighting}[]
\NormalTok{wwlist }\OperatorTok\StringTok{ }\KeywordTok{filter}\NormalTok{(psat_range}\OperatorTok{==}\StringTok{"1270-1520"}\NormalTok{) }\OperatorTok\StringTok{ }\KeywordTok{count}\NormalTok{()}
\CommentTok{#> # A tibble: 1 x 1}
\CommentTok{#>       n}
\CommentTok{#>   <int>}
\CommentTok{#> 1  8348}
\end{Highlighting}
\end{Shaded}

\end{frame}

\begin{frame}{Creating factor variables from character variables or from
integer variables}

See Appendix

\end{frame}

\begin{frame}[fragile]{Factor student exercise}

\begin{enumerate}
\def\labelenumi{\arabic{enumi}.}
\tightlist
\item
  After running the code below, use \texttt{typeof}, \texttt{class},
  \texttt{str}, and \texttt{attributes} functions to check the new
  variable \texttt{receive\_year}\\
\item
  Create a factor variable from the input variable
  \texttt{receive\_year} and name it \texttt{receive\_year\_fac}\\
\item
  Run the same functions (\texttt{typeof}, \texttt{class}, etc.) from
  the first question using the new variable you created\\
\item
  Get a count of \texttt{receive\_year\_fac}. \textbf{hint:} you could
  also run this in the console to see values associated with each factor
\end{enumerate}

Run this code to create a year variable from the input variable
``receive\_date''

\begin{Shaded}
\begin{Highlighting}[]
\CommentTok{#wwlist %>% glimpse()}

\KeywordTok{library}\NormalTok{(lubridate) }\CommentTok{#load library if you haven't already}
\NormalTok{wwlist <-}\StringTok{ }\NormalTok{wwlist }\OperatorTok
\StringTok{  }\KeywordTok{mutate}\NormalTok{(}\DataTypeTok{receive_year =} \KeywordTok{year}\NormalTok{(receive_date)) }\CommentTok{#creating year variable with the lubricate package}

\CommentTok{#Check variable}
\NormalTok{wwlist }\OperatorTok\StringTok{ }
\StringTok{  }\KeywordTok{count}\NormalTok{(receive_year)}

\NormalTok{wwlist }\OperatorTok
\StringTok{  }\KeywordTok{group_by}\NormalTok{(receive_year) }\OperatorTok\StringTok{ }
\StringTok{  }\KeywordTok{count}\NormalTok{(receive_date)}
\end{Highlighting}
\end{Shaded}

\end{frame}

\begin{frame}[fragile]{Factor student exercise solutions}

\begin{enumerate}
\def\labelenumi{\arabic{enumi}.}
\tightlist
\item
  Use \texttt{typeof}, \texttt{class}, \texttt{str}, and
  \texttt{attributes} functions to check the new variable
  \texttt{receive\_year}
\end{enumerate}

\begin{Shaded}
\begin{Highlighting}[]
\KeywordTok{typeof}\NormalTok{(wwlist}\OperatorTok{$}\NormalTok{receive_year)}
\CommentTok{#> [1] "double"}
\KeywordTok{class}\NormalTok{(wwlist}\OperatorTok{$}\NormalTok{receive_year)}
\CommentTok{#> [1] "numeric"}
\KeywordTok{str}\NormalTok{(wwlist}\OperatorTok{$}\NormalTok{receive_year)}
\CommentTok{#>  num [1:268396] 2016 2016 2016 2016 2016 ...}
\KeywordTok{attributes}\NormalTok{(wwlist}\OperatorTok{$}\NormalTok{receive_year) }
\CommentTok{#> NULL}
\end{Highlighting}
\end{Shaded}

\end{frame}

\begin{frame}[fragile]{Factor student exercise solutions}

\begin{enumerate}
\def\labelenumi{\arabic{enumi}.}
\setcounter{enumi}{1}
\tightlist
\item
  Now create a factor variable from the input variable
  \texttt{receive\_year} and name it \texttt{receive\_year\_fac}
\end{enumerate}

\begin{Shaded}
\begin{Highlighting}[]
\CommentTok{# create factor var; tidyverse approach}
\NormalTok{wwlist <-}\StringTok{ }\NormalTok{wwlist }\OperatorTok
\StringTok{  }\KeywordTok{mutate}\NormalTok{(}\DataTypeTok{receive_year_fac =} \KeywordTok{factor}\NormalTok{(receive_year))  }
\end{Highlighting}
\end{Shaded}

\end{frame}

\begin{frame}[fragile]{Factor student exercise solutions}

\begin{enumerate}
\def\labelenumi{\arabic{enumi}.}
\setcounter{enumi}{2}
\tightlist
\item
  Run the same functions (\texttt{typeof}, \texttt{class}, etc.) from
  the first question using the new variable you created
\end{enumerate}

\begin{Shaded}
\begin{Highlighting}[]
\KeywordTok{typeof}\NormalTok{(wwlist}\OperatorTok{$}\NormalTok{receive_year_fac)}
\CommentTok{#> [1] "integer"}
\KeywordTok{class}\NormalTok{(wwlist}\OperatorTok{$}\NormalTok{receive_year_fac)}
\CommentTok{#> [1] "factor"}
\KeywordTok{str}\NormalTok{(wwlist}\OperatorTok{$}\NormalTok{receive_year_fac)}
\CommentTok{#>  Factor w/ 3 levels "2016","2017",..: 1 1 1 1 1 1 1 1 1 1 ...}
\KeywordTok{attributes}\NormalTok{(wwlist}\OperatorTok{$}\NormalTok{receive_year_fac)   }
\CommentTok{#> $levels}
\CommentTok{#> [1] "2016" "2017" "2018"}
\CommentTok{#> }
\CommentTok{#> $class}
\CommentTok{#> [1] "factor"}
\end{Highlighting}
\end{Shaded}

\end{frame}

\begin{frame}[fragile]{Factor student exercise solutions}

\begin{enumerate}
\def\labelenumi{\arabic{enumi}.}
\setcounter{enumi}{3}
\tightlist
\item
  Get a count of \texttt{receive\_year\_fac}. \textbf{hint:} you could
  also run this in the console to see values associated with each factor
\end{enumerate}

\begin{Shaded}
\begin{Highlighting}[]
\NormalTok{wwlist }\OperatorTok
\StringTok{  }\KeywordTok{count}\NormalTok{(receive_year_fac)}
\CommentTok{#> # A tibble: 3 x 2}
\CommentTok{#>   receive_year_fac     n}
\CommentTok{#>   <fct>            <int>}
\CommentTok{#> 1 2016             89637}
\CommentTok{#> 2 2017             89816}
\CommentTok{#> 3 2018             88943}
\end{Highlighting}
\end{Shaded}

\end{frame}

\subsection{Class == labelled}\label{class-labelled}

\begin{frame}{Data we will use to introduce \texttt{labelled} class}

High school longitudinal surveys from National Center for Education
Statistics (NCES)

\begin{itemize}
\tightlist
\item
  Follow U.S. students from high school through college, labor market
\end{itemize}

We will be working with
\href{https://nces.ed.gov/surveys/hsls09/index.asp}{High School
Longitudinal Study of 2009 (HSLS:09)}

\begin{itemize}
\tightlist
\item
  Follows 9th graders from 2009
\item
  Data collection waves

  \begin{itemize}
  \tightlist
  \item
    Base Year (2009)
  \item
    First Follow-up (2012)
  \item
    2013 Update (2013)
  \item
    High School Transcripts (2013-2014)
  \item
    Second Follow-up (2016)
  \end{itemize}
\end{itemize}

\end{frame}

\begin{frame}[fragile]{\texttt{haven} package}

\href{https://haven.tidyverse.org/}{\texttt{haven}}, which is part of
\textbf{tidyverse}, ``enables R to read and write various data formats''
from the following statistical packages:

\begin{itemize}
\tightlist
\item
  SAS
\item
  SPSS
\item
  Stata
\end{itemize}

When using \texttt{haven} to read data, resulting R objects have these
characteristics:

\begin{itemize}
\tightlist
\item
  Are \textbf{tibbles}, a particular type of data frame we discuss
  future weeks
\item
  Transform variables with ``value labels'' into the \texttt{labelled()}
  class {[}our focus today{]}

  \begin{itemize}
  \tightlist
  \item
    \texttt{labelled} is an object \textbf{class} created by folks who
    created \texttt{haven} package
  \item
    \texttt{labelled} is an object class, just like \texttt{factor} is
    an object class
  \item
    \texttt{labelled} and \texttt{factor} classes are both viable
    alternatives for categorical variables
  \item
    Helpful description of \texttt{labelled} class
    \href{https://haven.tidyverse.org/articles/semantics.html}{HERE}
  \end{itemize}
\item
  Dates and times converted to R date/time classes
\item
  Character vectors not converted to factors
\end{itemize}

\end{frame}

\begin{frame}[fragile]{\texttt{haven} package}

Use \texttt{read\_dta()} function from \texttt{haven} to import Stata
dataset into R

\begin{Shaded}
\begin{Highlighting}[]
\NormalTok{hsls <-}\StringTok{ }\KeywordTok{read_dta}\NormalTok{(}\DataTypeTok{file=}\StringTok{"https://github.com/ozanj/rclass/raw/master/data/hsls/hsls_stu_small.dta"}\NormalTok{)}
\end{Highlighting}
\end{Shaded}

Let's examine the data {[}you \textbf{must} run this code chunk{]}

\begin{Shaded}
\begin{Highlighting}[]
\KeywordTok{names}\NormalTok{(hsls)}
\KeywordTok{names}\NormalTok{(hsls) <-}\StringTok{ }\KeywordTok{tolower}\NormalTok{(}\KeywordTok{names}\NormalTok{(hsls)) }\CommentTok{# convert names to lowercase}
\KeywordTok{names}\NormalTok{(hsls)}

\KeywordTok{str}\NormalTok{(hsls) }\CommentTok{# ugh}

\KeywordTok{str}\NormalTok{(hsls}\OperatorTok{$}\NormalTok{s3classes)}
\KeywordTok{attributes}\NormalTok{(hsls}\OperatorTok{$}\NormalTok{s3classes)}
\KeywordTok{typeof}\NormalTok{(hsls}\OperatorTok{$}\NormalTok{s3classes)}
\KeywordTok{class}\NormalTok{(hsls}\OperatorTok{$}\NormalTok{s3classes)}
\end{Highlighting}
\end{Shaded}

\end{frame}

\begin{frame}[fragile]{\texttt{labelled} package}

Purpose of the \texttt{labelled} package is to work with data imported
from SPSS/Stata/SAS using the \texttt{haven} package.

\begin{itemize}
\tightlist
\item
  In particular, \texttt{labelled} package creates functions to work
  with objects that have \texttt{labelled} class
\item
  From package documentation: ``purpose of the \texttt{labelled} package
  is to provide functions to manipulate \emph{metadata} as variable
  labels, value labels and defined missing values using the
  \texttt{labelled} class and the \texttt{label} attribute introduced in
  \texttt{haven} package.
\item
  More info on the \texttt{labelled} package:
  \href{https://cran.r-project.org/web/packages/labelled/vignettes/intro_labelled.html}{LINK}
\end{itemize}

Functions in \texttt{labelled} package

\begin{itemize}
\tightlist
\item
  \href{https://www.rdocumentation.org/packages/labelled/versions/1.1.0}{Full
  list}
\item
  A couple relevant functions

  \begin{itemize}
  \tightlist
  \item
    \texttt{val\_labels}: get or set variable \emph{value labels}
  \item
    \texttt{var\_label}: get or set a \emph{variable label}
  \end{itemize}
\end{itemize}

\begin{Shaded}
\begin{Highlighting}[]
\KeywordTok{attributes}\NormalTok{(hsls}\OperatorTok{$}\NormalTok{s3classes)}

\NormalTok{hsls }\OperatorTok\StringTok{ }\KeywordTok{select}\NormalTok{(s3classes) }\OperatorTok\StringTok{ }\NormalTok{var_label}
\NormalTok{hsls }\OperatorTok\StringTok{ }\KeywordTok{select}\NormalTok{(s3classes) }\OperatorTok\StringTok{ }\NormalTok{val_labels}
\end{Highlighting}
\end{Shaded}

\end{frame}

\begin{frame}[fragile]{Core concepts for understanding \texttt{labelled}
class {[}SKIP{]}}

\textbf{atomic vectors (and lists)} the underlying data

\begin{itemize}
\tightlist
\item
  data structures: vector or list
\item
  data type: numeric (integer or double); character; logical
\end{itemize}

\begin{Shaded}
\begin{Highlighting}[]
\KeywordTok{typeof}\NormalTok{(hsls}\OperatorTok{$}\NormalTok{s3classes)}
\CommentTok{#> [1] "double"}
\end{Highlighting}
\end{Shaded}

\textbf{augmented vectors} are atomic vectors with \textbf{attributes}
attached

\textbf{attributes} are ``metadata'' attached to an object. Examples

\begin{itemize}
\tightlist
\item
  \textbf{names}: names of elements of a vector or list (e.g., variable
  names)
\item
  \textbf{levels}: display output associated with values of a factor
  variable
\item
  \textbf{class}: e.g., factor, labelled
\end{itemize}

\begin{Shaded}
\begin{Highlighting}[]
\KeywordTok{attributes}\NormalTok{(hsls}\OperatorTok{$}\NormalTok{s3classes)}
\end{Highlighting}
\end{Shaded}

\textbf{class} is an object oriented programming concept. The
\texttt{class} of an object determines which functions can be applied to
the object and what those functions do

\begin{itemize}
\tightlist
\item
  e.g., can't apply \texttt{sum()} to an object where
  \texttt{class=character}
\end{itemize}

\end{frame}

\begin{frame}[fragile]{What is \texttt{labelled} class?}

\medskip

\begin{itemize}
\tightlist
\item
  \texttt{labelled} is an object class created by the \texttt{haven}
  package for importing variables from SAS/SPSS/Stata that have
  \textbf{value labels}
\item
  \textbf{value labels} {[}in Stata{]} are labels attached to specific
  values of a variable:

  \begin{itemize}
  \tightlist
  \item
    e.g., variable value \texttt{1} attached to value label ``married'',
    \texttt{2}=``single'', \texttt{3}=``divorced''
  \end{itemize}
\item
  Variables in an R data frame with \texttt{class==labelled}:

  \begin{itemize}
  \tightlist
  \item
    data \texttt{type} can be numeric(double) or character
  \item
    To see \texttt{value\ labels} associated with each value:

    \begin{itemize}
    \tightlist
    \item
      \texttt{attr(data\_frame\_name\$variable\_name,"labels")}
    \item
      e.g., \texttt{attr(hsls\$s3classes,"labels")}
    \end{itemize}
  \end{itemize}
\end{itemize}

Let's investigate the attributes of \texttt{hsls\$s3classes}

\begin{Shaded}
\begin{Highlighting}[]
\KeywordTok{typeof}\NormalTok{(hsls}\OperatorTok{$}\NormalTok{s3classes)}
\KeywordTok{class}\NormalTok{(hsls}\OperatorTok{$}\NormalTok{s3classes)}
\KeywordTok{str}\NormalTok{(hsls}\OperatorTok{$}\NormalTok{s3classes)}
\KeywordTok{attributes}\NormalTok{(hsls}\OperatorTok{$}\NormalTok{s3classes)}
\end{Highlighting}
\end{Shaded}

use \texttt{attr(object\_name,"attribute\_name")} to refer to each
attribute

\begin{Shaded}
\begin{Highlighting}[]
\KeywordTok{attr}\NormalTok{(hsls}\OperatorTok{$}\NormalTok{s3classes,}\StringTok{"label"}\NormalTok{)}
\KeywordTok{attr}\NormalTok{(hsls}\OperatorTok{$}\NormalTok{s3classes,}\StringTok{"labels"}\NormalTok{)}
\KeywordTok{attr}\NormalTok{(hsls}\OperatorTok{$}\NormalTok{s3classes,}\StringTok{"class"}\NormalTok{)}
\KeywordTok{attr}\NormalTok{(hsls}\OperatorTok{$}\NormalTok{s3classes,}\StringTok{"format.stata"}\NormalTok{)}
\end{Highlighting}
\end{Shaded}

\end{frame}

\begin{frame}[fragile]{Working with \texttt{labelled} class data}

\medskip

Show variable labels (\texttt{var\_label}); and show value labels
(\texttt{val\_labels})

\begin{Shaded}
\begin{Highlighting}[]
\NormalTok{hsls }\OperatorTok\StringTok{ }\KeywordTok{select}\NormalTok{(s3classes,s3clglvl) }\OperatorTok\StringTok{ }\NormalTok{var_label }\CommentTok{#show variable label}
\NormalTok{hsls }\OperatorTok\StringTok{ }\KeywordTok{select}\NormalTok{(s3classes,s3clglvl) }\OperatorTok\StringTok{ }\NormalTok{val_labels }\CommentTok{#show value labels}
\end{Highlighting}
\end{Shaded}

Create frequency tables with \texttt{labelled} class variables using
\texttt{count()}

\begin{itemize}
\tightlist
\item
  Default setting is to show variable \textbf{values} not \textbf{value
  labels}
\end{itemize}

\begin{Shaded}
\begin{Highlighting}[]
\NormalTok{hsls }\OperatorTok\StringTok{ }\KeywordTok{count}\NormalTok{(s3classes)}
\CommentTok{#investigate the object created}
\NormalTok{hsls_freq_temp <-}\StringTok{ }\NormalTok{hsls }\OperatorTok\StringTok{ }\KeywordTok{count}\NormalTok{(s3classes)}
\NormalTok{hsls_freq_temp}
\KeywordTok{rm}\NormalTok{(hsls_freq_temp)}
\end{Highlighting}
\end{Shaded}

To make frequency table show \textbf{value labels} add
\texttt{\%\textgreater{}\%\ as\_factor()} to pipe

\begin{itemize}
\tightlist
\item
  \texttt{as\_factor()} is function from \texttt{haven} that converts an
  object to a factor
\end{itemize}

\begin{Shaded}
\begin{Highlighting}[]
\NormalTok{hsls }\OperatorTok\StringTok{ }\KeywordTok{count}\NormalTok{(s3classes) }\OperatorTok\StringTok{ }\KeywordTok{as_factor}\NormalTok{()}
\CommentTok{#investigate the object created}
\NormalTok{hsls_freq_temp <-}\StringTok{ }\NormalTok{hsls }\OperatorTok\StringTok{ }\KeywordTok{count}\NormalTok{(s3classes)  }\OperatorTok\StringTok{ }\KeywordTok{as_factor}\NormalTok{()}
\NormalTok{hsls_freq_temp}
\KeywordTok{rm}\NormalTok{(hsls_freq_temp)}
\end{Highlighting}
\end{Shaded}

\end{frame}

\begin{frame}[fragile]{Working with \texttt{labelled} class data}

To isolate values of \texttt{labelled} class variables in
\texttt{filter()} function:

\begin{itemize}
\tightlist
\item
  refer to variable \textbf{value}, not the \textbf{value label}
\end{itemize}

\textbf{Task}

\begin{itemize}
\tightlist
\item
  how many observations in var \texttt{s3classes} associated with ``Unit
  non-response''
\item
  how many observations in var \texttt{s3classes} associated with
  ``Yes''
\end{itemize}

General steps to follow:

\begin{enumerate}
\def\labelenumi{\arabic{enumi}.}
\tightlist
\item
  investigate object
\item
  use filter to isolate desired observations
\end{enumerate}

Investigate object

\begin{Shaded}
\begin{Highlighting}[]
\KeywordTok{class}\NormalTok{(hsls}\OperatorTok{$}\NormalTok{s3classes)}
\NormalTok{hsls }\OperatorTok\StringTok{ }\KeywordTok{select}\NormalTok{(s3classes,s3clglvl) }\OperatorTok\StringTok{ }\NormalTok{var_label }\CommentTok{#show variable label}
\NormalTok{hsls }\OperatorTok\StringTok{ }\KeywordTok{count}\NormalTok{(s3classes) }\CommentTok{# freq table, values}
\NormalTok{hsls }\OperatorTok\StringTok{ }\KeywordTok{count}\NormalTok{(s3classes) }\OperatorTok\StringTok{ }\KeywordTok{as_factor}\NormalTok{() }\CommentTok{# freq table, value labels}
\end{Highlighting}
\end{Shaded}

filter specific values

\begin{Shaded}
\begin{Highlighting}[]
\NormalTok{hsls }\OperatorTok\StringTok{ }\KeywordTok{filter}\NormalTok{(s3classes}\OperatorTok{==-}\DecValTok{8}\NormalTok{) }\OperatorTok\StringTok{ }\KeywordTok{count}\NormalTok{() }\CommentTok{# -8 = unit non-response}
\NormalTok{hsls }\OperatorTok\StringTok{ }\KeywordTok{filter}\NormalTok{(s3classes}\OperatorTok{==}\DecValTok{1}\NormalTok{) }\OperatorTok\StringTok{ }\KeywordTok{count}\NormalTok{() }\CommentTok{# 1 = yes}
\end{Highlighting}
\end{Shaded}

\end{frame}

\begin{frame}[fragile]{Labelled student exercise}

\begin{enumerate}
\def\labelenumi{\arabic{enumi}.}
\tightlist
\item
  Get variable and value labels of \texttt{s3hs}\\
\item
  Get a count of the variable showing the values and the value labels.
  \textbf{hint} use factor()\\
\item
  Filter if value is associated with ``Missing''\\
\item
  Filter if value is associated with ``Missing'' or ``Unit
  non-response''
\end{enumerate}

\end{frame}

\begin{frame}[fragile]{Labelled student exercise solutions}

\begin{enumerate}
\def\labelenumi{\arabic{enumi}.}
\tightlist
\item
  Get variable and value labels of \texttt{s3hs}
\end{enumerate}

\begin{Shaded}
\begin{Highlighting}[]
\NormalTok{hsls }\OperatorTok\StringTok{ }
\StringTok{  }\KeywordTok{select}\NormalTok{(s3hs) }\OperatorTok\StringTok{ }
\StringTok{  }\KeywordTok{var_label}\NormalTok{() }
\CommentTok{#> $s3hs}
\CommentTok{#> [1] "S3 B01F Attending high school or homeschool as of Nov 1 2013"}

\NormalTok{hsls }\OperatorTok\StringTok{ }
\StringTok{  }\KeywordTok{select}\NormalTok{(s3hs) }\OperatorTok\StringTok{ }
\StringTok{  }\KeywordTok{val_labels}\NormalTok{()}
\CommentTok{#> $s3hs}
\CommentTok{#>                                      Missing }
\CommentTok{#>                                           -9 }
\CommentTok{#>                            Unit non-response }
\CommentTok{#>                                           -8 }
\CommentTok{#>                      Item legitimate skip/NA }
\CommentTok{#>                                           -7 }
\CommentTok{#>                     Component not applicable }
\CommentTok{#>                                           -6 }
\CommentTok{#> Item not administered: abbreviated interview }
\CommentTok{#>                                           -4 }
\CommentTok{#>                                          Yes }
\CommentTok{#>                                            1 }
\CommentTok{#>                                           No }
\CommentTok{#>                                            2 }
\CommentTok{#>                                   Don't know }
\CommentTok{#>                                            3}
\end{Highlighting}
\end{Shaded}

\end{frame}

\begin{frame}[fragile]{Labelled student exercise solutions}

\begin{enumerate}
\def\labelenumi{\arabic{enumi}.}
\setcounter{enumi}{1}
\tightlist
\item
  Get a count of the variable \texttt{s3hs} showing the value labels.
  \textbf{hint} use factor()
\end{enumerate}

\begin{Shaded}
\begin{Highlighting}[]
\NormalTok{hsls }\OperatorTok\StringTok{ }
\StringTok{  }\KeywordTok{count}\NormalTok{(s3hs) }
\CommentTok{#> # A tibble: 6 x 2}
\CommentTok{#>   s3hs          n}
\CommentTok{#>   <dbl+lbl> <int>}
\CommentTok{#> 1 -9           22}
\CommentTok{#> 2 -8         4945}
\CommentTok{#> 3 -7        16770}
\CommentTok{#> 4 " 1"        624}
\CommentTok{#> 5 " 2"        985}
\CommentTok{#> 6 " 3"        157}

\NormalTok{hsls }\OperatorTok\StringTok{ }
\StringTok{  }\KeywordTok{count}\NormalTok{(s3hs) }\OperatorTok\StringTok{ }
\StringTok{  }\KeywordTok{as_factor}\NormalTok{() }
\CommentTok{#> # A tibble: 6 x 2}
\CommentTok{#>   s3hs                        n}
\CommentTok{#>   <fct>                   <int>}
\CommentTok{#> 1 Missing                    22}
\CommentTok{#> 2 Unit non-response        4945}
\CommentTok{#> 3 Item legitimate skip/NA 16770}
\CommentTok{#> 4 Yes                       624}
\CommentTok{#> 5 No                        985}
\CommentTok{#> 6 Don't know                157}
\end{Highlighting}
\end{Shaded}

\end{frame}

\begin{frame}[fragile]{Labelled student exercise solutions}

\begin{enumerate}
\def\labelenumi{\arabic{enumi}.}
\setcounter{enumi}{2}
\tightlist
\item
  Filter if value is associated with ``Missing''
\end{enumerate}

\begin{Shaded}
\begin{Highlighting}[]
\NormalTok{hsls }\OperatorTok
\StringTok{  }\KeywordTok{filter}\NormalTok{(s3hs}\OperatorTok{==}\StringTok{ }\OperatorTok{-}\DecValTok{9}\NormalTok{) }\OperatorTok\StringTok{ }
\StringTok{  }\KeywordTok{count}\NormalTok{()}
\CommentTok{#> # A tibble: 1 x 1}
\CommentTok{#>       n}
\CommentTok{#>   <int>}
\CommentTok{#> 1    22}
\end{Highlighting}
\end{Shaded}

\end{frame}

\begin{frame}[fragile]{Labelled student exercise solutions}

\begin{enumerate}
\def\labelenumi{\arabic{enumi}.}
\setcounter{enumi}{3}
\tightlist
\item
  Filter if value is associated with ``Missing'' or ``Unit
  non-response''
\end{enumerate}

\begin{Shaded}
\begin{Highlighting}[]
\NormalTok{hsls }\OperatorTok
\StringTok{  }\KeywordTok{filter}\NormalTok{(s3hs}\OperatorTok{==}\StringTok{ }\OperatorTok{-}\DecValTok{9} \OperatorTok{|}\StringTok{ }\NormalTok{s3hs}\OperatorTok{==}\StringTok{ }\OperatorTok{-}\DecValTok{8}\NormalTok{) }\OperatorTok\StringTok{ }
\StringTok{  }\KeywordTok{count}\NormalTok{()}
\CommentTok{#> # A tibble: 1 x 1}
\CommentTok{#>       n}
\CommentTok{#>   <int>}
\CommentTok{#> 1  4967}
\end{Highlighting}
\end{Shaded}

\end{frame}

\subsection{Comparing labelled class to factor
class}\label{comparing-labelled-class-to-factor-class}

\begin{frame}[fragile]{Comparing \texttt{class==labelled} to
\texttt{class==factor}}

\begin{longtable}[]{@{}lll@{}}
\toprule
& \texttt{class==labelled} & \texttt{class==factor}\tabularnewline
\midrule
\endhead
data type & numeric or character & integer\tabularnewline
name of value label attribute & labels & levels\tabularnewline
refer to data using & variable values & levels attribute\tabularnewline
\bottomrule
\end{longtable}

\end{frame}

\begin{frame}[fragile]{Converting \texttt{class==labelled} to
\texttt{class==factor}}

The \texttt{as\_factor()} function from \texttt{haven} package converts
variables with \texttt{class==labelled} to \texttt{class==factor}

\begin{itemize}
\tightlist
\item
  Can be used for descriptive statistics
\end{itemize}

\begin{Shaded}
\begin{Highlighting}[]
\NormalTok{hsls }\OperatorTok\StringTok{ }\KeywordTok{select}\NormalTok{(s3classes) }\OperatorTok\StringTok{ }\KeywordTok{count}\NormalTok{(s3classes) }\OperatorTok\StringTok{ }\KeywordTok{as_factor}\NormalTok{()}
\end{Highlighting}
\end{Shaded}

\begin{itemize}
\tightlist
\item
  Can create object with some or all \texttt{labelled} vars converted to
  \texttt{factor}
\end{itemize}

\begin{Shaded}
\begin{Highlighting}[]
\NormalTok{hsls_f <-}\StringTok{ }\KeywordTok{as_factor}\NormalTok{(hsls,}\DataTypeTok{only_labelled =} \OtherTok{TRUE}\NormalTok{)}
\end{Highlighting}
\end{Shaded}

Let's examine this object

\begin{Shaded}
\begin{Highlighting}[]
\KeywordTok{glimpse}\NormalTok{(hsls_f)}
\NormalTok{hsls_f }\OperatorTok\StringTok{ }\KeywordTok{select}\NormalTok{(s3classes,s3clglvl) }\OperatorTok\StringTok{ }\KeywordTok{str}\NormalTok{()}
\KeywordTok{typeof}\NormalTok{(hsls_f}\OperatorTok{$}\NormalTok{s3classes)}
\KeywordTok{class}\NormalTok{(hsls_f}\OperatorTok{$}\NormalTok{s3classes)}
\KeywordTok{attributes}\NormalTok{(hsls_f}\OperatorTok{$}\NormalTok{s3classes)}

\NormalTok{hsls_f }\OperatorTok\StringTok{ }\KeywordTok{select}\NormalTok{(s3classes) }\OperatorTok\StringTok{ }\KeywordTok{var_label}\NormalTok{()}
\NormalTok{hsls_f }\OperatorTok\StringTok{ }\KeywordTok{select}\NormalTok{(s3classes) }\OperatorTok\StringTok{ }\KeywordTok{val_labels}\NormalTok{()}
\end{Highlighting}
\end{Shaded}

\end{frame}

\begin{frame}[fragile]{Working with \texttt{class==factor} data}

Showing values associated with factor levels

\begin{Shaded}
\begin{Highlighting}[]
\NormalTok{hsls_f }\OperatorTok\StringTok{ }\KeywordTok{count}\NormalTok{(s3classes)}
\CommentTok{#> # A tibble: 5 x 2}
\CommentTok{#>   s3classes             n}
\CommentTok{#>   <fct>             <int>}
\CommentTok{#> 1 Missing              59}
\CommentTok{#> 2 Unit non-response  4945}
\CommentTok{#> 3 Yes               13477}
\CommentTok{#> 4 No                 3401}
\CommentTok{#> 5 Don't know         1621}
\end{Highlighting}
\end{Shaded}

In code, refer \texttt{level} attribute not variable value

\begin{Shaded}
\begin{Highlighting}[]
\NormalTok{hsls_f }\OperatorTok\StringTok{ }\KeywordTok{filter}\NormalTok{(s3classes}\OperatorTok{==}\StringTok{"Yes"}\NormalTok{) }\OperatorTok\StringTok{ }\KeywordTok{count}\NormalTok{(s3classes)}
\CommentTok{#> # A tibble: 1 x 2}
\CommentTok{#>   s3classes     n}
\CommentTok{#>   <fct>     <int>}
\CommentTok{#> 1 Yes       13477}
\end{Highlighting}
\end{Shaded}

\end{frame}

\section{Exploratory data analysis
(EDA)}\label{exploratory-data-analysis-eda}

\begin{frame}{What is exploratory data analysis (EDA)?}

The
\href{https://towardsdatascience.com/exploratory-data-analysis-8fc1cb20fd15}{Towards
Data Science} website has a nice definition of EDA:

\begin{quote}
``Exploratory Data Analysis refers to the critical process of performing
initial investigations on data so as to discover patterns,to spot
anomalies,to test hypothesis and to check assumptions with the help of
summary statistics''
\end{quote}

\textbf{This course focuses on ``data management'':}

\begin{itemize}
\tightlist
\item
  investigating and cleaning data for the purpose of creating analysis
  variables
\item
  Basically, everything that happens \textbf{before} you conduct
  analyses
\end{itemize}

\textbf{I think about ``exploratory data analysis for data quality''}

\begin{itemize}
\tightlist
\item
  Investigating values and patterns of variables from ``input data''
\item
  Identifying and cleaning errors or values that need to be changed
\item
  Creating analysis variables
\item
  Checking values of analysis variables agains values of input variables
\end{itemize}

\end{frame}

\begin{frame}{How we will teach exploratory data analysis}

Will teach exploratory data analysis (EDA) in two sub-sections:

\begin{enumerate}
\def\labelenumi{\arabic{enumi}.}
\tightlist
\item
  Introduce ``Tools of EDA'':

  \begin{itemize}
  \tightlist
  \item
    Demonstrate code to investigate variables and relatioship between
    variables
  \item
    I'll focus on the \textbf{tidyverse} approach rather than
    \textbf{base R}\\
  \item
    Most of these tools are just the application of programming skills
    you have already learned
  \end{itemize}
\item
  Provide ``Guidelines for EDA''

  \begin{itemize}
  \tightlist
  \item
    Less about coding, more about practices you should follow and
    mentality necessary to ensure high data quality
  \end{itemize}
\end{enumerate}

\end{frame}

\subsection{Tools for EDA}\label{tools-for-eda}

\begin{frame}[fragile]{Tools of EDA}

\textbf{To do EDA for data quality, must master the following tools:}

\begin{itemize}
\tightlist
\item
  \medskip \textbf{Select, sort, filter, and print} in order to see data
  patterns, anomolies

  \begin{itemize}
  \tightlist
  \item
    Select and sort particular values of particular variables
  \item
    Print particular values of particular variables
  \end{itemize}
\item
  \textbf{One-way descriptive analyses} (i.e,. focus on one variable)

  \begin{itemize}
  \tightlist
  \item
    Descriptive analyses for continuous variables
  \item
    Descriptive analyses for discreet/categorical variables
  \end{itemize}
\item
  \textbf{Two-way descriptive analyses} (relationship between two
  variables)

  \begin{itemize}
  \tightlist
  \item
    Categorical by categorical
  \item
    Categorical by continuous
  \item
    Continuous by continuous
  \end{itemize}
\end{itemize}

Whenever using any of these tools, \textbf{pay close attention to
missing values and how they are coded}

\begin{itemize}
\tightlist
\item
  Often, the ``input'' variables don't code missing values as
  \texttt{NA}
\item
  Especially when working with survey data, missing values coded as a
  negative number (e.g., \texttt{-9},\texttt{-8},\texttt{-4}) with
  different negative values representing different reasons for data
  being missing
\item
  sometimes missing values coded as very high positive numbers
\item
  Therefore, important to investigate input vars prior to creating
  analysis vars
\end{itemize}

\end{frame}

\begin{frame}[fragile]{Tools of EDA}

First, Let's create a smaller version of the HSLS:09 dataset

\begin{Shaded}
\begin{Highlighting}[]
\CommentTok{#hsls %>% var_label()}
\NormalTok{hsls_small <-}\StringTok{ }\NormalTok{hsls }\OperatorTok
\StringTok{  }\KeywordTok{select}\NormalTok{(stu_id,x3univ1,x3sqstat,x4univ1,x4sqstat,s3classes,}
\NormalTok{         s3work,s3focus,s3clgft,s3workft,s3clgid,s3clgcntrl,}
\NormalTok{         s3clglvl,s3clgsel,s3clgstate,s3proglevel,x4evrappclg,}
\NormalTok{         x4evratndclg,x4atndclg16fb,x4ps1sector,x4ps1level,}
\NormalTok{         x4ps1ctrl,x4ps1select,x4refsector,x4reflevel,x4refctrl,}
\NormalTok{         x4refselect, x2sex,x2race,x2paredu,x2txmtscor,x4x2ses,x4x2sesq5)}
\end{Highlighting}
\end{Shaded}

\begin{Shaded}
\begin{Highlighting}[]
\KeywordTok{names}\NormalTok{(hsls_small)}
\NormalTok{hsls_small }\OperatorTok\StringTok{ }\KeywordTok{var_label}\NormalTok{()}
\end{Highlighting}
\end{Shaded}

\end{frame}

\begin{frame}[fragile]{Tools of EDA: select, sort, filter, and print}

We've already know \texttt{select()}, \texttt{arrange()},
\texttt{filter()}

\medskip Select, sort, and print specific vars

\begin{Shaded}
\begin{Highlighting}[]
\CommentTok{#sort and print}
\NormalTok{hsls_small }\OperatorTok\StringTok{ }\KeywordTok{arrange}\NormalTok{(}\KeywordTok{desc}\NormalTok{(stu_id)) }\OperatorTok\StringTok{ }
\StringTok{  }\KeywordTok{select}\NormalTok{(stu_id,x3univ1,x3sqstat,s3classes,s3clglvl)}

\CommentTok{#investigate variable attributes}
\NormalTok{hsls_small }\OperatorTok\StringTok{ }\KeywordTok{arrange}\NormalTok{(}\KeywordTok{desc}\NormalTok{(stu_id)) }\OperatorTok\StringTok{ }
\StringTok{  }\KeywordTok{select}\NormalTok{(stu_id,x3univ1,x3sqstat,s3classes,s3clglvl) }\OperatorTok\StringTok{ }\KeywordTok{str}\NormalTok{()}

\CommentTok{#print observations with value labels rather than variable values}
\NormalTok{hsls_small }\OperatorTok\StringTok{ }\KeywordTok{arrange}\NormalTok{(}\KeywordTok{desc}\NormalTok{(stu_id)) }\OperatorTok\StringTok{ }
\StringTok{  }\KeywordTok{select}\NormalTok{(stu_id,x3univ1,x3sqstat,s3classes,s3clglvl) }\OperatorTok\StringTok{ }\KeywordTok{as_factor}\NormalTok{()}
\end{Highlighting}
\end{Shaded}

Sometimes helpful to increase the number of observations printed

\begin{Shaded}
\begin{Highlighting}[]
\KeywordTok{class}\NormalTok{(hsls_small) }\CommentTok{#it's a tibble, which is the "tidyverse" version of a data frame}
\KeywordTok{options}\NormalTok{(}\DataTypeTok{tibble.print_min=}\DecValTok{50}\NormalTok{) }
\CommentTok{# execute this in console}
\NormalTok{hsls_small }\OperatorTok\StringTok{ }\KeywordTok{arrange}\NormalTok{(}\KeywordTok{desc}\NormalTok{(stu_id)) }\OperatorTok
\StringTok{  }\KeywordTok{select}\NormalTok{(stu_id,x3univ1,x3sqstat,s3classes,s3clglvl)}
\KeywordTok{options}\NormalTok{(}\DataTypeTok{tibble.print_min=}\DecValTok{10}\NormalTok{) }\CommentTok{# set default printing back to 10 lines}
\end{Highlighting}
\end{Shaded}

\end{frame}

\begin{frame}[fragile]{One-way descriptive stats for continuous vars,
Base R approach {[}SKIP{]}}

\begin{Shaded}
\begin{Highlighting}[]
\KeywordTok{mean}\NormalTok{(hsls_small}\OperatorTok{$}\NormalTok{x2txmtscor)}
\KeywordTok{sd}\NormalTok{(hsls_small}\OperatorTok{$}\NormalTok{x2txmtscor)}

\CommentTok{#Careful: summary stats include value of -8!}
\KeywordTok{min}\NormalTok{(hsls_small}\OperatorTok{$}\NormalTok{x2txmtscor)}
\KeywordTok{max}\NormalTok{(hsls_small}\OperatorTok{$}\NormalTok{x2txmtscor)}
\end{Highlighting}
\end{Shaded}

Be careful with \texttt{NA} values

\begin{Shaded}
\begin{Highlighting}[]
\CommentTok{#Create variable replacing -8 with NA}
\NormalTok{hsls_small_temp <-}\StringTok{ }\NormalTok{hsls_small }\OperatorTok\StringTok{ }
\StringTok{  }\KeywordTok{mutate}\NormalTok{(}\DataTypeTok{x2txmtscorv2=}\KeywordTok{ifelse}\NormalTok{(x2txmtscor}\OperatorTok{==-}\DecValTok{8}\NormalTok{,}\OtherTok{NA}\NormalTok{,x2txmtscor))}
\NormalTok{hsls_small_temp }\OperatorTok\StringTok{ }\KeywordTok{filter}\NormalTok{(}\KeywordTok{is.na}\NormalTok{(x2txmtscorv2)) }\OperatorTok\StringTok{ }\KeywordTok{count}\NormalTok{(x2txmtscorv2)}

\KeywordTok{mean}\NormalTok{(hsls_small_temp}\OperatorTok{$}\NormalTok{x2txmtscorv2)}
\KeywordTok{mean}\NormalTok{(hsls_small_temp}\OperatorTok{$}\NormalTok{x2txmtscorv2, }\DataTypeTok{na.rm=}\OtherTok{TRUE}\NormalTok{)}
\KeywordTok{rm}\NormalTok{(hsls_small_temp)}
\end{Highlighting}
\end{Shaded}

\end{frame}

\begin{frame}[fragile]{One-way descriptive stats for continuous vars,
Tidyverse approach}

Use \texttt{summarise\_at()}, a variation of \texttt{summarise()}, to
make descriptive stats

\begin{itemize}
\tightlist
\item
  explain \texttt{.args=list(na.rm=TRUE)} on following slides
\end{itemize}

\textbf{Task}:

\begin{itemize}
\tightlist
\item
  calculate descriptive stats for \texttt{x2txmtscor}, math test score
\end{itemize}

\begin{Shaded}
\begin{Highlighting}[]
\CommentTok{#?summarise_at}
\NormalTok{hsls_small }\OperatorTok\StringTok{ }\KeywordTok{select}\NormalTok{(x2txmtscor) }\OperatorTok\StringTok{ }\KeywordTok{var_label}\NormalTok{()}
\CommentTok{#> $x2txmtscor}
\CommentTok{#> [1] "X2 Mathematics standardized theta score"}
\NormalTok{hsls_small }\OperatorTok\StringTok{ }
\StringTok{  }\KeywordTok{summarise_at}\NormalTok{(}
    \DataTypeTok{.vars =} \KeywordTok{vars}\NormalTok{(x2txmtscor),}
    \DataTypeTok{.funs =} \KeywordTok{funs}\NormalTok{(mean, sd, min, max, }\DataTypeTok{.args=}\KeywordTok{list}\NormalTok{(}\DataTypeTok{na.rm=}\OtherTok{TRUE}\NormalTok{))}
\NormalTok{  )}
\CommentTok{#> # A tibble: 1 x 4}
\CommentTok{#>    mean    sd   min   max}
\CommentTok{#>   <dbl> <dbl> <dbl> <dbl>}
\CommentTok{#> 1  44.1  21.8    -8  84.9}
\end{Highlighting}
\end{Shaded}

\end{frame}

\begin{frame}[fragile]{One-way descriptive stats for continuous vars,
Tidyverse approach}

Can calculate descriptive stats for more than one variable at a time

\textbf{Task}:

\begin{itemize}
\tightlist
\item
  calculate descriptive stats for \texttt{x2txmtscor}, math test score,
  and \texttt{x4x2ses}, socioeconomic index score
\end{itemize}

\begin{Shaded}
\begin{Highlighting}[]
\NormalTok{hsls_small }\OperatorTok\StringTok{ }\KeywordTok{select}\NormalTok{(x2txmtscor,x4x2ses) }\OperatorTok\StringTok{ }\KeywordTok{var_label}\NormalTok{()}
\CommentTok{#> $x2txmtscor}
\CommentTok{#> [1] "X2 Mathematics standardized theta score"}
\CommentTok{#> }
\CommentTok{#> $x4x2ses}
\CommentTok{#> [1] "X4 Revised X2 Socio-economic status composite"}

\NormalTok{hsls_small }\OperatorTok\StringTok{ }
\StringTok{  }\KeywordTok{summarise_at}\NormalTok{(}
    \DataTypeTok{.vars =} \KeywordTok{vars}\NormalTok{(x2txmtscor,x4x2ses),}
    \DataTypeTok{.funs =} \KeywordTok{funs}\NormalTok{(mean, sd, min, max, }\DataTypeTok{.args=}\KeywordTok{list}\NormalTok{(}\DataTypeTok{na.rm=}\OtherTok{TRUE}\NormalTok{))}
\NormalTok{  )}
\CommentTok{#> # A tibble: 1 x 8}
\CommentTok{#>   x2txmtscor_mean x4x2ses_mean x2txmtscor_sd x4x2ses_sd x2txmtscor_min}
\CommentTok{#>             <dbl>        <dbl>         <dbl>      <dbl>          <dbl>}
\CommentTok{#> 1            44.1       -0.802          21.8       2.63             -8}
\CommentTok{#> # ... with 3 more variables: x4x2ses_min <dbl>, x2txmtscor_max <dbl>,}
\CommentTok{#> #   x4x2ses_max <dbl>}
\end{Highlighting}
\end{Shaded}

\end{frame}

\begin{frame}[fragile]{One-way descriptive stats for continuous vars,
Tidyverse approach}

``Input vars'' in survey data often have negative values for
missing/skips

\begin{Shaded}
\begin{Highlighting}[]
\NormalTok{hsls_small }\OperatorTok\StringTok{ }\KeywordTok{filter}\NormalTok{(x2txmtscor}\OperatorTok{<}\DecValTok{0}\NormalTok{) }\OperatorTok\StringTok{ }\KeywordTok{count}\NormalTok{(x2txmtscor)}
\end{Highlighting}
\end{Shaded}

R includes those negative values when calculating stats; you don't want
this

\begin{itemize}
\tightlist
\item
  Solution: create version of variable that replaces negative values
  with \texttt{NA}
\end{itemize}

\begin{Shaded}
\begin{Highlighting}[]
\NormalTok{hsls_small }\OperatorTok\StringTok{ }\KeywordTok{mutate}\NormalTok{(}\DataTypeTok{x2txmtscor_na=}\KeywordTok{ifelse}\NormalTok{(x2txmtscor}\OperatorTok{<}\DecValTok{0}\NormalTok{,}\OtherTok{NA}\NormalTok{,x2txmtscor)) }\OperatorTok
\StringTok{  }\KeywordTok{summarise_at}\NormalTok{(}
    \DataTypeTok{.vars =} \KeywordTok{vars}\NormalTok{(x2txmtscor_na),}
    \DataTypeTok{.funs =} \KeywordTok{funs}\NormalTok{(mean, sd, min, max, }\DataTypeTok{.args=}\KeywordTok{list}\NormalTok{(}\DataTypeTok{na.rm=}\OtherTok{TRUE}\NormalTok{))}
\NormalTok{  )}
\CommentTok{#> # A tibble: 1 x 4}
\CommentTok{#>    mean    sd   min   max}
\CommentTok{#>   <dbl> <dbl> <dbl> <dbl>}
\CommentTok{#> 1  51.5  10.2  22.2  84.9}
\end{Highlighting}
\end{Shaded}

What if you didn't include \texttt{.args=list(na.rm=TRUE)}?

\begin{Shaded}
\begin{Highlighting}[]
\NormalTok{hsls_small }\OperatorTok\StringTok{ }\KeywordTok{mutate}\NormalTok{(}\DataTypeTok{x2txmtscor_na=}\KeywordTok{ifelse}\NormalTok{(x2txmtscor}\OperatorTok{<}\DecValTok{0}\NormalTok{,}\OtherTok{NA}\NormalTok{,x2txmtscor)) }\OperatorTok
\StringTok{  }\KeywordTok{summarise_at}\NormalTok{(}
    \DataTypeTok{.vars =} \KeywordTok{vars}\NormalTok{(x2txmtscor_na),}
    \DataTypeTok{.funs =} \KeywordTok{funs}\NormalTok{(mean, sd, min, max))}
\CommentTok{#> # A tibble: 1 x 4}
\CommentTok{#>    mean    sd   min   max}
\CommentTok{#>   <dbl> <dbl> <dbl> <dbl>}
\CommentTok{#> 1    NA   NaN    NA    NA}
\end{Highlighting}
\end{Shaded}

\end{frame}

\begin{frame}[fragile]{One-way descriptive stats for continuous vars,
Tidyverse approach}

How to identify these missing/skip values if you don't have a codebook?

\begin{itemize}
\tightlist
\item
  \texttt{count()} combined with \texttt{filter()} helpful for finding
  extreme values of continuous vars, which are often associated with
  missing or skip
\end{itemize}

\begin{Shaded}
\begin{Highlighting}[]
\CommentTok{#variable x2txmtscor}
\NormalTok{hsls_small }\OperatorTok\StringTok{ }\KeywordTok{filter}\NormalTok{(x2txmtscor}\OperatorTok{<}\DecValTok{0}\NormalTok{) }\OperatorTok\StringTok{ }
\StringTok{  }\KeywordTok{count}\NormalTok{(x2txmtscor)}
\CommentTok{#> # A tibble: 1 x 2}
\CommentTok{#>   x2txmtscor     n}
\CommentTok{#>        <dbl> <int>}
\CommentTok{#> 1         -8  2909}

\CommentTok{#variable s3clglvl}
\NormalTok{hsls_small }\OperatorTok\StringTok{ }\KeywordTok{select}\NormalTok{(s3clglvl) }\OperatorTok\StringTok{ }\KeywordTok{var_label}\NormalTok{()}
\CommentTok{#> $s3clglvl}
\CommentTok{#> [1] "S3 Enrolled college IPEDS level"}

\NormalTok{hsls_small }\OperatorTok\StringTok{ }\KeywordTok{filter}\NormalTok{(s3clglvl}\OperatorTok{<}\DecValTok{0}\NormalTok{) }\OperatorTok
\StringTok{  }\KeywordTok{count}\NormalTok{(s3clglvl)}
\CommentTok{#> # A tibble: 3 x 2}
\CommentTok{#>   s3clglvl      n}
\CommentTok{#>   <dbl+lbl> <int>}
\CommentTok{#> 1 -9          487}
\CommentTok{#> 2 -8         4945}
\CommentTok{#> 3 -7         5022}
\end{Highlighting}
\end{Shaded}

\end{frame}

\begin{frame}[fragile]{One-way descriptive stats student exercise}

\begin{enumerate}
\def\labelenumi{\arabic{enumi}.}
\tightlist
\item
  Using the object \texttt{hsls}, identify variable type, variable
  class, and check the variable vakyes and value labels of
  \texttt{x4ps1start}

  \begin{itemize}
  \tightlist
  \item
    variable \texttt{x4ps1start} identifies month and year student first
    started postsecondary education
  \item
    \textbf{Note}: This variable is a bit counterintuitive.

    \begin{itemize}
    \tightlist
    \item
      e.g., the value \texttt{201105} refers to May 2011
    \end{itemize}
  \end{itemize}
\item
  Get a frequency count of the variable \texttt{x4ps1start}\\
\item
  Get a frequency count of the variable, but this time only observations
  that have negative values \textbf{hint}: use filter()\\
\item
  Create a new version of the variable \texttt{x4ps1start\_na} that
  replaces negative values with NAs and use \texttt{summarise\_at()} to
  get the min and max value.
\end{enumerate}

\end{frame}

\begin{frame}[fragile]{One-way descriptive stats student exercise
solutions}

\medskip

\begin{enumerate}
\def\labelenumi{\arabic{enumi}.}
\tightlist
\item
  Using the object \texttt{hsls}, identify variable type, variable
  class, and check the variable vakyes and value labels of
  \texttt{x4ps1start}
\end{enumerate}

\begin{Shaded}
\begin{Highlighting}[]
\KeywordTok{typeof}\NormalTok{(hsls}\OperatorTok{$}\NormalTok{x4ps1start)}
\CommentTok{#> [1] "double"}
\KeywordTok{class}\NormalTok{(hsls}\OperatorTok{$}\NormalTok{x4ps1start)}
\CommentTok{#> [1] "labelled"}

\NormalTok{hsls }\OperatorTok\StringTok{ }\KeywordTok{select}\NormalTok{(x4ps1start) }\OperatorTok\StringTok{ }\KeywordTok{var_label}\NormalTok{()}
\CommentTok{#> $x4ps1start}
\CommentTok{#> [1] "X4 Month and year of enrollment at first postsecondary institution"}

\NormalTok{hsls }\OperatorTok\StringTok{ }\KeywordTok{select}\NormalTok{(x4ps1start) }\OperatorTok\StringTok{ }\KeywordTok{val_labels}\NormalTok{()}
\CommentTok{#> $x4ps1start}
\CommentTok{#>                                      Missing }
\CommentTok{#>                                           -9 }
\CommentTok{#>                            Unit non-response }
\CommentTok{#>                                           -8 }
\CommentTok{#>                      Item legitimate skip/NA }
\CommentTok{#>                                           -7 }
\CommentTok{#>                     Component not applicable }
\CommentTok{#>                                           -6 }
\CommentTok{#> Item not administered: abbreviated interview }
\CommentTok{#>                                           -4 }
\CommentTok{#>                        Carry through missing }
\CommentTok{#>                                           -3 }
\CommentTok{#>                                   Don't know }
\CommentTok{#>                                           -1}
\end{Highlighting}
\end{Shaded}

\end{frame}

\begin{frame}[fragile]{One-way descriptive stats student exercise
solutions}

\begin{enumerate}
\def\labelenumi{\arabic{enumi}.}
\setcounter{enumi}{1}
\tightlist
\item
  Get a frequency count of the variable \texttt{x4ps1start}
\end{enumerate}

\begin{Shaded}
\begin{Highlighting}[]
\NormalTok{hsls }\OperatorTok
\StringTok{  }\KeywordTok{count}\NormalTok{(x4ps1start)}
\CommentTok{#> # A tibble: 9 x 2}
\CommentTok{#>   x4ps1start     n}
\CommentTok{#>   <dbl+lbl>  <int>}
\CommentTok{#> 1 "    -9"     107}
\CommentTok{#> 2 "    -8"    6168}
\CommentTok{#> 3 "    -7"    4281}
\CommentTok{#> 4 201100        57}
\CommentTok{#> 5 201200       206}
\CommentTok{#> 6 201300     10800}
\CommentTok{#> 7 201400      1295}
\CommentTok{#> 8 201500       471}
\CommentTok{#> 9 201600       118}
\end{Highlighting}
\end{Shaded}

\end{frame}

\begin{frame}[fragile]{One-way descriptive stats student exercise
solutions}

\begin{enumerate}
\def\labelenumi{\arabic{enumi}.}
\setcounter{enumi}{2}
\tightlist
\item
  Get a frequency count of the variable, but this time only observations
  that have negative values \textbf{hint}: use filter()
\end{enumerate}

\begin{Shaded}
\begin{Highlighting}[]
\NormalTok{hsls }\OperatorTok\StringTok{ }
\StringTok{  }\KeywordTok{filter}\NormalTok{(x4ps1start}\OperatorTok{<}\DecValTok{0}\NormalTok{) }\OperatorTok\StringTok{ }
\StringTok{  }\KeywordTok{count}\NormalTok{(x4ps1start)}
\CommentTok{#> # A tibble: 3 x 2}
\CommentTok{#>   x4ps1start     n}
\CommentTok{#>   <dbl+lbl>  <int>}
\CommentTok{#> 1 -9           107}
\CommentTok{#> 2 -8          6168}
\CommentTok{#> 3 -7          4281}
\end{Highlighting}
\end{Shaded}

\end{frame}

\begin{frame}[fragile]{One-way descriptive stats student exercise
solutions}

\begin{enumerate}
\def\labelenumi{\arabic{enumi}.}
\setcounter{enumi}{3}
\tightlist
\item
  Create a new version \texttt{x4ps1start\_na} of the variable
  \texttt{x4ps1start} that replaces negative values with NAs and use
  \texttt{summarise\_at()} to get the min and max value.
\end{enumerate}

\begin{Shaded}
\begin{Highlighting}[]
\NormalTok{hsls }\OperatorTok\StringTok{ }\KeywordTok{mutate}\NormalTok{(}\DataTypeTok{x4ps1start_na=}\KeywordTok{ifelse}\NormalTok{(x4ps1start}\OperatorTok{<}\DecValTok{0}\NormalTok{,}\OtherTok{NA}\NormalTok{,x4ps1start)) }\OperatorTok
\StringTok{  }\KeywordTok{summarise_at}\NormalTok{(}
    \DataTypeTok{.vars =} \KeywordTok{vars}\NormalTok{(x4ps1start_na),}
    \DataTypeTok{.funs =} \KeywordTok{funs}\NormalTok{(min, max, }\DataTypeTok{.args=}\KeywordTok{list}\NormalTok{(}\DataTypeTok{na.rm=}\OtherTok{TRUE}\NormalTok{))}
\NormalTok{  )}
\CommentTok{#> # A tibble: 1 x 2}
\CommentTok{#>      min    max}
\CommentTok{#>    <dbl>  <dbl>}
\CommentTok{#> 1 201100 201600}
\end{Highlighting}
\end{Shaded}

\end{frame}

\begin{frame}[fragile]{One-way descriptive stats for
discrete/categorical vars, Tidyverse approach}

Use \texttt{count()} to investigate values of discreet or categorical
variables

For variables where \texttt{class==labelled}

\begin{Shaded}
\begin{Highlighting}[]
\KeywordTok{class}\NormalTok{(hsls_small}\OperatorTok{$}\NormalTok{s3classes)}
\CommentTok{#show counts of variable values}
\NormalTok{hsls_small }\OperatorTok\StringTok{ }\KeywordTok{count}\NormalTok{(s3classes)}
\CommentTok{#show counts of value labels}
\NormalTok{hsls_small }\OperatorTok\StringTok{ }\KeywordTok{count}\NormalTok{(s3classes) }\OperatorTok\StringTok{ }\KeywordTok{as_factor}\NormalTok{()}
\end{Highlighting}
\end{Shaded}

\begin{itemize}
\tightlist
\item
  I like \texttt{count()} because the default setting is to show
  \texttt{NA} values too!
\end{itemize}

\begin{Shaded}
\begin{Highlighting}[]
\NormalTok{hsls_small }\OperatorTok\StringTok{ }\KeywordTok{mutate}\NormalTok{(}\DataTypeTok{s3classes_na=}\KeywordTok{ifelse}\NormalTok{(s3classes}\OperatorTok{<}\DecValTok{0}\NormalTok{,}\OtherTok{NA}\NormalTok{,s3classes)) }\OperatorTok\StringTok{ }
\StringTok{  }\KeywordTok{count}\NormalTok{(s3classes_na)}
\end{Highlighting}
\end{Shaded}

Simultaneously show both values and value labels on count tables for
\texttt{class==labelled}

\begin{itemize}
\tightlist
\item
  requires some concepts/functions we haven't introduced
\end{itemize}

\begin{Shaded}
\begin{Highlighting}[]
\NormalTok{x <-}\StringTok{ }\NormalTok{hsls_small }\OperatorTok\StringTok{ }\KeywordTok{count}\NormalTok{(s3classes)}
\NormalTok{y <-}\StringTok{ }\NormalTok{hsls_small }\OperatorTok\StringTok{ }\KeywordTok{count}\NormalTok{(s3classes) }\OperatorTok\StringTok{ }\KeywordTok{as_factor}\NormalTok{()}
\KeywordTok{bind_cols}\NormalTok{(x[,}\DecValTok{1}\NormalTok{], y)}
\end{Highlighting}
\end{Shaded}

\end{frame}

\begin{frame}[fragile]{One-way descriptive stats for factor vars
{[}OPTIONAL/SKIP{]}}

For variables where \texttt{class==factor}

\begin{itemize}
\tightlist
\item
  Note: data frame object \texttt{hsls\_f} created in previous section
\end{itemize}

\begin{Shaded}
\begin{Highlighting}[]
\CommentTok{#use variable from the hsls data frame where vars are factors}
\KeywordTok{typeof}\NormalTok{(hsls_f}\OperatorTok{$}\NormalTok{s3classes)}
\KeywordTok{class}\NormalTok{(hsls_f}\OperatorTok{$}\NormalTok{s3classes)}
\KeywordTok{attributes}\NormalTok{(hsls_f}\OperatorTok{$}\NormalTok{s3classes)}

\CommentTok{#show frequency table}
\NormalTok{hsls_f }\OperatorTok\StringTok{ }\KeywordTok{count}\NormalTok{(s3classes)}

\CommentTok{#Create VAR that converts different types of missing to NA and then create frequency table}

\CommentTok{#note: within ifelse() used levels(s3classes)[s3classes]) rather than s3classes  to show attribute levels not values}
\NormalTok{hsls_f }\OperatorTok\StringTok{ }\KeywordTok{mutate}\NormalTok{(}\DataTypeTok{s3classes_f=}\KeywordTok{ifelse}\NormalTok{(s3classes }\OperatorTok\StringTok{ }\KeywordTok{c}\NormalTok{(}\StringTok{"Missing"}\NormalTok{,}\StringTok{"Unit non-response"}\NormalTok{),}\OtherTok{NA}\NormalTok{,}\KeywordTok{levels}\NormalTok{(s3classes)[s3classes])) }\OperatorTok\StringTok{ }
\StringTok{  }\KeywordTok{count}\NormalTok{(s3classes_f)}
\end{Highlighting}
\end{Shaded}

\end{frame}

\begin{frame}[fragile]{Relationship between variables, categorical by
categorical}

Two-way frequency table, called ``cross tabulation'', important for data
quality

\begin{itemize}
\tightlist
\item
  When you create categorical analysis var from single categorical
  ``input'' var

  \begin{itemize}
  \tightlist
  \item
    Two-way tables show us whether we did this correctly\\
  \end{itemize}
\item
  Two-way tables helpful for understanding skip patterns in surveys
\end{itemize}

\textbf{key to syntax}

\begin{itemize}
\tightlist
\item
  \texttt{group\_by(var1)\ \%\textgreater{}\%\ count(var2)}
\item
  play around with which variable is \texttt{var1} and which variable is
  \texttt{var2}
\end{itemize}

\textbf{Task}:

\begin{itemize}
\tightlist
\item
  Create a two-way table between \texttt{s3classes} and
  \texttt{s3clglvl}
\end{itemize}

\begin{Shaded}
\begin{Highlighting}[]
\NormalTok{hsls_small }\OperatorTok\StringTok{ }\KeywordTok{select}\NormalTok{(s3classes,s3clglvl) }\OperatorTok\StringTok{ }\KeywordTok{var_label}\NormalTok{()}

\NormalTok{hsls_small }\OperatorTok\StringTok{ }\KeywordTok{group_by}\NormalTok{(s3classes) }\OperatorTok\StringTok{ }\KeywordTok{count}\NormalTok{(s3clglvl) }\CommentTok{# show values}
\NormalTok{hsls_small }\OperatorTok\StringTok{ }\KeywordTok{group_by}\NormalTok{(s3classes) }\OperatorTok\StringTok{ }\KeywordTok{count}\NormalTok{(s3clglvl) }\OperatorTok\StringTok{ }\KeywordTok{as_factor}\NormalTok{() }\CommentTok{# show value labels}
\end{Highlighting}
\end{Shaded}

\end{frame}

\begin{frame}[fragile]{Relationship between variables, categorical by
categorical}

Two-way frequency table, also called ``cross tabulation''

What if one of the variables has \texttt{NAs}?

\begin{itemize}
\tightlist
\item
  Table created by \texttt{group\_by()} and \texttt{count()} shows
  \texttt{NAs}!
\end{itemize}

\textbf{Task}:

\begin{itemize}
\tightlist
\item
  Create a version of \texttt{s3classes} called \texttt{s3classes\_na}
  that changes negative values to \texttt{NA}
\item
  Create a two-way table between \texttt{s3classes\_na} and
  \texttt{s3clglvl}
\end{itemize}

\begin{Shaded}
\begin{Highlighting}[]
\NormalTok{hsls_small }\OperatorTok\StringTok{ }
\StringTok{  }\KeywordTok{mutate}\NormalTok{(}\DataTypeTok{s3classes_na=}\KeywordTok{ifelse}\NormalTok{(s3classes}\OperatorTok{<}\DecValTok{0}\NormalTok{,}\OtherTok{NA}\NormalTok{,s3classes)) }\OperatorTok
\StringTok{  }\KeywordTok{group_by}\NormalTok{(s3classes_na) }\OperatorTok\StringTok{ }\KeywordTok{count}\NormalTok{(s3clglvl)}

\CommentTok{#example where we create some NA obs in the second variable}
\NormalTok{hsls_small }\OperatorTok\StringTok{ }
\StringTok{  }\KeywordTok{mutate}\NormalTok{(}\DataTypeTok{s3classes_na=}\KeywordTok{ifelse}\NormalTok{(s3classes}\OperatorTok{<}\DecValTok{0}\NormalTok{,}\OtherTok{NA}\NormalTok{,s3classes),}
         \DataTypeTok{s3clglvl_na=}\KeywordTok{ifelse}\NormalTok{(s3clglvl}\OperatorTok{==-}\DecValTok{7}\NormalTok{,}\OtherTok{NA}\NormalTok{,s3clglvl)) }\OperatorTok
\StringTok{  }\KeywordTok{group_by}\NormalTok{(s3classes_na) }\OperatorTok\StringTok{ }\KeywordTok{count}\NormalTok{(s3clglvl_na)}
\end{Highlighting}
\end{Shaded}

\end{frame}

\begin{frame}[fragile]{Relationship between variables, categorical by
categorical {[}SKIP{]}}

Tables above are pretty ugly

Use the \texttt{spread()} function from \texttt{tidyr} package to create
table with one variable as columns and the other variable as rows

\begin{itemize}
\tightlist
\item
  The variable you place in \texttt{spread()} will be columns
\item
  We learn \texttt{spread()} function next week
\end{itemize}

\begin{Shaded}
\begin{Highlighting}[]
\NormalTok{hsls_small }\OperatorTok\StringTok{ }\KeywordTok{group_by}\NormalTok{(s3classes) }\OperatorTok\StringTok{ }\KeywordTok{count}\NormalTok{(s3clglvl) }\OperatorTok\StringTok{ }
\StringTok{  }\KeywordTok{spread}\NormalTok{(s3classes, n)}

\NormalTok{hsls_small }\OperatorTok\StringTok{ }\KeywordTok{group_by}\NormalTok{(s3classes) }\OperatorTok\StringTok{ }\KeywordTok{count}\NormalTok{(s3clglvl) }\OperatorTok\StringTok{ }
\StringTok{  }\KeywordTok{as_factor}\NormalTok{() }\OperatorTok\StringTok{  }\KeywordTok{spread}\NormalTok{(s3classes, n)}
\NormalTok{hsls_small }\OperatorTok\StringTok{ }\KeywordTok{group_by}\NormalTok{(s3classes) }\OperatorTok\StringTok{ }\KeywordTok{count}\NormalTok{(s3clglvl) }\OperatorTok\StringTok{ }
\StringTok{  }\KeywordTok{as_factor}\NormalTok{() }\OperatorTok\StringTok{  }\KeywordTok{spread}\NormalTok{(s3clglvl, n)}
\end{Highlighting}
\end{Shaded}

\end{frame}

\begin{frame}[fragile]{Relationship between variables, categorical by
continuous}

Investigating relationship between multiple variables is a little
tougher when at least one of the variables is continuous

\textbf{Conditional mean} (like regression with continuous Y and one
categorical X):

\begin{itemize}
\tightlist
\item
  Shows average values of continous variables within groups
\item
  Groups are defined by your categorical variable(s)
\end{itemize}

\textbf{key to syntax}

\begin{itemize}
\tightlist
\item
  \texttt{group\_by(categorical\_var)\ \%\textgreater{}\%\ summarise\_at(.vars\ =\ vars(continuous\_var)}
\end{itemize}

\textbf{Task}

\begin{itemize}
\tightlist
\item
  Calculate mean math score, \texttt{x2txmtscor}, for each value of
  parental education, \texttt{x2paredu}
\end{itemize}

\begin{Shaded}
\begin{Highlighting}[]
\CommentTok{#first, investigate parental education}
\NormalTok{hsls_small }\OperatorTok\StringTok{ }\KeywordTok{count}\NormalTok{(x2paredu)}
\NormalTok{hsls_small }\OperatorTok\StringTok{ }\KeywordTok{count}\NormalTok{(x2paredu) }\OperatorTok\StringTok{ }\NormalTok{as_factor}

\CommentTok{# using dplyr to get average math score by parental education level}
\NormalTok{hsls_small }\OperatorTok\StringTok{ }\KeywordTok{group_by}\NormalTok{(x2paredu) }\OperatorTok
\StringTok{    }\KeywordTok{summarise_at}\NormalTok{(}\DataTypeTok{.vars =} \KeywordTok{vars}\NormalTok{(x2txmtscor),}
                 \DataTypeTok{.funs =} \KeywordTok{funs}\NormalTok{(mean, }\DataTypeTok{.args =} \KeywordTok{list}\NormalTok{(}\DataTypeTok{na.rm =} \OtherTok{TRUE}\NormalTok{))) }\OperatorTok
\StringTok{    }\KeywordTok{as_factor}\NormalTok{()}
\end{Highlighting}
\end{Shaded}

\end{frame}

\begin{frame}[fragile]{Relationship between variables, categorical by
continuous}

\textbf{Task}

\begin{itemize}
\tightlist
\item
  Calculate mean math score, \texttt{x2txmtscor}, for each value of
  \texttt{x2paredu}
\end{itemize}

For checking data quality, helpful to calculate other stats besides mean

\begin{Shaded}
\begin{Highlighting}[]
\NormalTok{hsls_small }\OperatorTok\StringTok{ }\KeywordTok{group_by}\NormalTok{(x2paredu) }\OperatorTok
\StringTok{    }\KeywordTok{summarise_at}\NormalTok{(}\DataTypeTok{.vars =} \KeywordTok{vars}\NormalTok{(x2txmtscor),}
                 \DataTypeTok{.funs =} \KeywordTok{funs}\NormalTok{(mean, min, max, }\DataTypeTok{.args =} \KeywordTok{list}\NormalTok{(}\DataTypeTok{na.rm =} \OtherTok{TRUE}\NormalTok{))) }\OperatorTok
\StringTok{    }\KeywordTok{as_factor}\NormalTok{()}
\end{Highlighting}
\end{Shaded}

Always Investigate presence of missing/skip values

\begin{Shaded}
\begin{Highlighting}[]
\NormalTok{hsls_small }\OperatorTok\StringTok{ }\KeywordTok{filter}\NormalTok{(x2paredu}\OperatorTok{<}\DecValTok{0}\NormalTok{) }\OperatorTok\StringTok{ }\KeywordTok{count}\NormalTok{(x2paredu)}
\NormalTok{hsls_small }\OperatorTok\StringTok{ }\KeywordTok{filter}\NormalTok{(x2txmtscor}\OperatorTok{<}\DecValTok{0}\NormalTok{) }\OperatorTok\StringTok{ }\KeywordTok{count}\NormalTok{(x2txmtscor)}
\end{Highlighting}
\end{Shaded}

Replace \texttt{-8} with \texttt{NA} and re-calculate conditional stats

\begin{Shaded}
\begin{Highlighting}[]
\NormalTok{hsls_small }\OperatorTok\StringTok{ }
\StringTok{  }\KeywordTok{mutate}\NormalTok{(}\DataTypeTok{x2paredu_na=}\KeywordTok{ifelse}\NormalTok{(x2paredu}\OperatorTok{<}\DecValTok{0}\NormalTok{,}\OtherTok{NA}\NormalTok{,x2paredu),}
         \DataTypeTok{x2txmtscor_na=}\KeywordTok{ifelse}\NormalTok{(x2txmtscor}\OperatorTok{<}\DecValTok{0}\NormalTok{,}\OtherTok{NA}\NormalTok{,x2txmtscor)) }\OperatorTok\StringTok{ }
\StringTok{  }\KeywordTok{group_by}\NormalTok{(x2paredu_na) }\OperatorTok
\StringTok{  }\KeywordTok{summarise_at}\NormalTok{(}\DataTypeTok{.vars =} \KeywordTok{vars}\NormalTok{(x2txmtscor_na),}
               \DataTypeTok{.funs =} \KeywordTok{funs}\NormalTok{(mean, min, max, }\DataTypeTok{.args =} \KeywordTok{list}\NormalTok{(}\DataTypeTok{na.rm =} \OtherTok{TRUE}\NormalTok{))) }\OperatorTok
\StringTok{  }\KeywordTok{as_factor}\NormalTok{()}
\end{Highlighting}
\end{Shaded}

\end{frame}

\begin{frame}[fragile]{Student exercise}

Can use same approach to calculate conditional mean by multiple
\texttt{group\_by()} variables

\begin{itemize}
\tightlist
\item
  Just add additional variables within \texttt{group\_by()}
\end{itemize}

\begin{enumerate}
\def\labelenumi{\arabic{enumi}.}
\tightlist
\item
  Calculate mean math test score (\texttt{x2txmtscor}), for each
  combination of parental education (\texttt{x2paredu}) and sex
  (\texttt{x2sex}).
\end{enumerate}

\end{frame}

\begin{frame}[fragile]{Student exercise solution}

\begin{enumerate}
\def\labelenumi{\arabic{enumi}.}
\tightlist
\item
  Calculate mean math test score (\texttt{x2txmtscor}), for each
  combination of parental education (\texttt{x2paredu}) and sex
  (\texttt{x2sex})
\end{enumerate}

\begin{Shaded}
\begin{Highlighting}[]
\CommentTok{#hsls_small %>% count(x2sex)}

\NormalTok{hsls_small }\OperatorTok
\StringTok{  }\KeywordTok{group_by}\NormalTok{(x2paredu,x2sex) }\OperatorTok
\StringTok{  }\KeywordTok{summarise_at}\NormalTok{(}\DataTypeTok{.vars =} \KeywordTok{vars}\NormalTok{(x2txmtscor),}
               \DataTypeTok{.funs =} \KeywordTok{funs}\NormalTok{(mean, }\DataTypeTok{.args =} \KeywordTok{list}\NormalTok{(}\DataTypeTok{na.rm =} \OtherTok{TRUE}\NormalTok{))) }\OperatorTok
\StringTok{  }\KeywordTok{as_factor}\NormalTok{()}
\end{Highlighting}
\end{Shaded}

\end{frame}

\subsection{Guidelines for EDA}\label{guidelines-for-eda}

\begin{frame}[fragile]{Guidelines for ``EDA for data quality''}

Assme that your goal in ``EDA for data quality'' is to investigate
``input'' data sources and create ``analysis variables''

\begin{itemize}
\tightlist
\item
  Usually, your analysis dataset will incorporate multiple sources of
  input data, including data you collect (primary data) and/or data
  collected by others (secondary data)
\end{itemize}

While this is not a linear process, these are the broad steps I follow

\begin{enumerate}
\def\labelenumi{\arabic{enumi}.}
\tightlist
\item
  Understand how input data sources were created

  \begin{itemize}
  \tightlist
  \item
    e.g., when working with survey data, have survey questionnaire and
    codebooks on hand
  \end{itemize}
\item
  For each input data source, identify the ``unit of analysis'' and
  which combination of variables uniquely identify observations
\item
  Investigate patterns in input variables
\item
  Create analysis variable from input variable(s)
\item
  Verify that analysis variable is created correctly through descriptive
  statistics that compare values of input variable(s) against values of
  the analysis variable
\end{enumerate}

\textbf{Always be aware of missing values}

\begin{itemize}
\tightlist
\item
  They will not always be coded as \texttt{NA} in input variables
\end{itemize}

\end{frame}

\begin{frame}{``Unit of analysis'' and which variables uniquely identify
observations}

``Unit of analysis'' refers to ``what does each observation represent''
in an input data source

\begin{itemize}
\tightlist
\item
  If each obs represents a student, you have ``student level data''
\item
  If each obs represents a student-course, you have ``student-course
  level data''
\item
  If each obs represents a school, you have ``school-level data''
\item
  If each obs represents a school-year, you have ``school-year level
  data''
\end{itemize}

How to identify unit of analysis

\begin{itemize}
\tightlist
\item
  data documentation
\item
  investigating the data set
\end{itemize}

We will go over syntax for identifying unit of analysis in subsequent
weeks

\end{frame}

\begin{frame}{Rules for variable creation}

Rules I follow for variable creation

\begin{enumerate}
\def\labelenumi{\arabic{enumi}.}
\tightlist
\item
  \medskip Never modify ``input variable''; instead create new variable
  based on input variable(s)

  \begin{itemize}
  \tightlist
  \item
    Always keep input variables used to create new variables
  \end{itemize}
\item
  Investigate input variable(s) and relationship between input variables
\item
  Developing a plan for creation of analysis variable

  \begin{itemize}
  \tightlist
  \item
    e.g., for each possible value of input variables, what should value
    of analysis variable be?
  \end{itemize}
\item
  Write code to create analysis variable
\item
  Run descriptive checks to verify new variables are constructed
  correctly

  \begin{itemize}
  \tightlist
  \item
    Can ``comment out'' these checks, but don't delete them
  \end{itemize}
\item
  Document new variables with notes and labels
\end{enumerate}

\end{frame}

\begin{frame}[fragile]{Rules for variable creation}

\textbf{Task}:

\begin{itemize}
\tightlist
\item
  Create analysis for variable ses qunitile called \texttt{sesq5} based
  on \texttt{x4x2sesq5} that converts negative values to \texttt{NAs}
\end{itemize}

\begin{Shaded}
\begin{Highlighting}[]
\CommentTok{#investigate input variable}
\NormalTok{hsls_small }\OperatorTok\StringTok{ }\KeywordTok{select}\NormalTok{(x4x2sesq5) }\OperatorTok\StringTok{ }\KeywordTok{var_label}\NormalTok{()}
\NormalTok{hsls_small }\OperatorTok\StringTok{ }\KeywordTok{select}\NormalTok{(x4x2sesq5) }\OperatorTok\StringTok{ }\KeywordTok{val_labels}\NormalTok{()}
\NormalTok{hsls_small }\OperatorTok\StringTok{ }\KeywordTok{select}\NormalTok{(x4x2sesq5) }\OperatorTok\StringTok{ }\KeywordTok{count}\NormalTok{(x4x2sesq5)}
\NormalTok{hsls_small }\OperatorTok\StringTok{ }\KeywordTok{select}\NormalTok{(x4x2sesq5) }\OperatorTok\StringTok{ }\KeywordTok{count}\NormalTok{(x4x2sesq5) }\OperatorTok\StringTok{ }\KeywordTok{as_factor}\NormalTok{()}

\CommentTok{#create analysis variable}
\NormalTok{hsls_small_temp <-}\StringTok{ }\NormalTok{hsls_small }\OperatorTok\StringTok{ }
\StringTok{  }\KeywordTok{mutate}\NormalTok{(}\DataTypeTok{sesq5=}\KeywordTok{ifelse}\NormalTok{(x4x2sesq5}\OperatorTok{==-}\DecValTok{8}\NormalTok{,}\OtherTok{NA}\NormalTok{,x4x2sesq5)) }\CommentTok{# approach 1}
\NormalTok{hsls_small_temp <-}\StringTok{ }\NormalTok{hsls_small }\OperatorTok\StringTok{ }
\StringTok{  }\KeywordTok{mutate}\NormalTok{(}\DataTypeTok{sesq5=}\KeywordTok{ifelse}\NormalTok{(x4x2sesq5}\OperatorTok{<}\DecValTok{0}\NormalTok{,}\OtherTok{NA}\NormalTok{,x4x2sesq5)) }\CommentTok{# approach 2}

\CommentTok{#verify}
\NormalTok{hsls_small_temp }\OperatorTok\StringTok{ }\KeywordTok{group_by}\NormalTok{(x4x2sesq5) }\OperatorTok\StringTok{ }\KeywordTok{count}\NormalTok{(sesq5)}
\end{Highlighting}
\end{Shaded}

\end{frame}

\begin{frame}{Overview of problem set due next week}

\textbf{Assignment}:

\begin{itemize}
\tightlist
\item
  create GPA from postsecondary transcript student-course level data
\end{itemize}

\textbf{Data source}: \href{https://nces.ed.gov/surveys/nls72/}{National
Longitudinal Study of 1972 (NLS72)}

\begin{itemize}
\tightlist
\item
  Follows 12th graders from 1972

  \begin{itemize}
  \tightlist
  \item
    Base year: 1972
  \item
    Follow-up surveys in: 1973, 1974, 1976, 1979, 1986
  \item
    Postsecondary transcripts collected in 1984
  \end{itemize}
\end{itemize}

\textbf{Why use such an old survey for this assignment?}

\begin{itemize}
\tightlist
\item
  NLS72 predates data privacy agreements; transcript data publicly
  available
\end{itemize}

\textbf{What we do to make assignment more manageable}

\begin{itemize}
\tightlist
\item
  we give you code for investigation of variables
\item
  we give you some hints/guidelines
\item
  but you are responsible for developing plan to create GPA vars and for
  executing plan (rather than us giving you step-by-step quations)
\end{itemize}

\textbf{Why this assignment?}

\begin{enumerate}
\def\labelenumi{\arabic{enumi}.}
\tightlist
\item
  Give you more practice investigating data, cleaning data, creating
  variables that require processing across rows
\item
  Sometimes social justice is creating student GPA from course-level
  data
\end{enumerate}

\end{frame}

\subsection{Skip patterns in survey
data}\label{skip-patterns-in-survey-data}

\begin{frame}[fragile]{What are skip patterns}

Pretty easy to create an analysis variable based on a single input
variable

Harder to create analysis variables based on multiple input variables

\begin{itemize}
\tightlist
\item
  When working with survey data, even seemingly simple analysis
  variables require multiple input variables due to ``skip patterns''
\end{itemize}

What are ``skip patterns''?

\begin{itemize}
\tightlist
\item
  Response on a particular survey item determines whether respondent
  answers some set of subsequent questions
\item
  What are some examples of this?
\end{itemize}

Key to working with skip patterns

\begin{itemize}
\tightlist
\item
  Have the survey questionnaire on hand
\item
  Sometimes it appears that analysis variable requires only one input
  variable, but really depends on several input variables because of
  skip patterns

  \begin{itemize}
  \tightlist
  \item
    Don't just blindly turn ``missing'' and ``skips'' from survey data
    to \texttt{NAs} in your analysis variable
  \item
    Rather, trace why these ``missing'' and ``skips'' appear and decide
    how they should be coded in your analysis variable
  \end{itemize}
\end{itemize}

\end{frame}

\begin{frame}[fragile]{Creating analysis variables in the presence of
skip patterns}

Task: Create a measure of ``level'' of postsecondary institution
attended in 2013 from HSLS:09 survey data

\begin{itemize}
\tightlist
\item
  ``level'' is highest award-level of the postsecondary institution

  \begin{itemize}
  \tightlist
  \item
    e.g., if highest award is associate's degree (a two-year degree),
    then `level==2'
  \end{itemize}
\item
  The measure, \texttt{pselev2013}, should have following
  {[}non-missing{]} values:

  \begin{enumerate}
  \def\labelenumi{\arabic{enumi}.}
  \tightlist
  \item
    Not attending postsecondary education institution
  \item
    Attending a 2-year or less-than-2-year institution
  \item
    Attending 4-year or greater-than-4year institution
  \end{enumerate}
\end{itemize}

Background info:

\begin{itemize}
\tightlist
\item
  In ``2013 Update'' of HSLS:09, students asked about college attendance

  \begin{itemize}
  \tightlist
  \item
    Variables from student responses to ``2013 Update'' have prefix
    \texttt{s3}
  \end{itemize}
\item
  Survey questionnaire for 2013 update can be found
  \href{https://nces.ed.gov/surveys/hsls09/pdf/2013_Student_Parent_Questionnaire.pdf}{HERE}
\item
  The ``online codebook'' website
  \href{https://nces.ed.gov/onlinecodebook}{HERE} has info about
  specific variables
\item
  Measure has 3 input variables {[}usually must figure this out
  yourself{]}:

  \begin{enumerate}
  \def\labelenumi{\arabic{enumi}.}
  \tightlist
  \item
    \texttt{x3sqstat}: ``X3 Student questionnaire status''
  \item
    \texttt{s3classes}: ``S3 B01A Taking postsecondary classes as of Nov
    1 2013''
  \item
    \texttt{s3clglvl}: ``S3 Enrolled college IPEDS level''
  \end{enumerate}
\end{itemize}

\begin{Shaded}
\begin{Highlighting}[]
\NormalTok{hsls_small }\OperatorTok\StringTok{ }\KeywordTok{select}\NormalTok{(x3sqstat,s3classes,s3clglvl) }\OperatorTok\StringTok{ }\KeywordTok{var_label}\NormalTok{()}
\end{Highlighting}
\end{Shaded}

You won't have time to complete this task, but develop a plan for the
task and get as far as you can

\end{frame}

\begin{frame}[fragile]{Creating analysis variables in the presence of
skip patterns}

Step 1a: Investigate each input variable separately

\begin{Shaded}
\begin{Highlighting}[]
\CommentTok{#variable labels}
\NormalTok{hsls_small }\OperatorTok\StringTok{ }\KeywordTok{select}\NormalTok{(x3sqstat,s3classes,s3clglvl) }\OperatorTok\StringTok{ }\KeywordTok{var_label}\NormalTok{()}

\NormalTok{hsls_small }\OperatorTok\StringTok{ }\KeywordTok{count}\NormalTok{(x3sqstat)}
\NormalTok{hsls_small }\OperatorTok\StringTok{ }\KeywordTok{count}\NormalTok{(x3sqstat) }\OperatorTok\StringTok{ }\KeywordTok{as_factor}\NormalTok{()}

\NormalTok{hsls_small }\OperatorTok\StringTok{ }\KeywordTok{count}\NormalTok{(s3classes)}
\NormalTok{hsls_small }\OperatorTok\StringTok{ }\KeywordTok{count}\NormalTok{(s3classes) }\OperatorTok\StringTok{ }\KeywordTok{as_factor}\NormalTok{()}

\NormalTok{hsls_small }\OperatorTok\StringTok{ }\KeywordTok{count}\NormalTok{(s3clglvl)}
\NormalTok{hsls_small }\OperatorTok\StringTok{ }\KeywordTok{count}\NormalTok{(s3clglvl) }\OperatorTok\StringTok{ }\KeywordTok{as_factor}\NormalTok{()}
\end{Highlighting}
\end{Shaded}

\end{frame}

\begin{frame}[fragile]{Creating analysis variables in the presence of
skip patterns}

Step 1b: Investigate relationship between input variables

\begin{Shaded}
\begin{Highlighting}[]
\CommentTok{#x3sqstate and s3classes}
\NormalTok{hsls_small }\OperatorTok\StringTok{ }\KeywordTok{group_by}\NormalTok{(x3sqstat) }\OperatorTok\StringTok{ }\KeywordTok{count}\NormalTok{(s3classes) }
\NormalTok{hsls_small }\OperatorTok\StringTok{ }\KeywordTok{group_by}\NormalTok{(x3sqstat) }\OperatorTok\StringTok{ }\KeywordTok{count}\NormalTok{(s3classes) }\OperatorTok\StringTok{ }\KeywordTok{as_factor}\NormalTok{()}

\NormalTok{hsls_small }\OperatorTok\StringTok{ }\KeywordTok{filter}\NormalTok{(x3sqstat}\OperatorTok{==}\DecValTok{8}\NormalTok{) }\OperatorTok\StringTok{ }\KeywordTok{count}\NormalTok{(s3classes)}
\NormalTok{hsls_small }\OperatorTok\StringTok{ }\KeywordTok{filter}\NormalTok{(x3sqstat}\OperatorTok{==}\DecValTok{8}\NormalTok{) }\OperatorTok\StringTok{ }\KeywordTok{count}\NormalTok{(s3classes}\OperatorTok{==-}\DecValTok{8}\NormalTok{)}
\NormalTok{hsls_small }\OperatorTok\StringTok{ }\KeywordTok{filter}\NormalTok{(x3sqstat }\OperatorTok{!=}\DecValTok{8}\NormalTok{) }\OperatorTok\StringTok{ }\KeywordTok{count}\NormalTok{(s3classes)}

\CommentTok{#x3sqstate, s3classes and s3clglvl}
\NormalTok{hsls_small }\OperatorTok\StringTok{ }\KeywordTok{group_by}\NormalTok{(s3classes) }\OperatorTok\StringTok{ }\KeywordTok{count}\NormalTok{(s3clglvl) }
\NormalTok{hsls_small }\OperatorTok\StringTok{ }\KeywordTok{group_by}\NormalTok{(s3classes) }\OperatorTok\StringTok{ }\KeywordTok{count}\NormalTok{(s3clglvl) }\OperatorTok\StringTok{ }\KeywordTok{as_factor}\NormalTok{()}

\CommentTok{#add filter for whether student did not respond to X3 questionnaire}
\NormalTok{hsls_small }\OperatorTok\StringTok{ }\KeywordTok{filter}\NormalTok{(x3sqstat}\OperatorTok{==}\DecValTok{8}\NormalTok{) }\OperatorTok\StringTok{ }\KeywordTok{group_by}\NormalTok{(s3classes) }\OperatorTok\StringTok{ }\KeywordTok{count}\NormalTok{(s3clglvl) }
\NormalTok{hsls_small }\OperatorTok\StringTok{ }\KeywordTok{filter}\NormalTok{(x3sqstat }\OperatorTok{!=}\DecValTok{8}\NormalTok{) }\OperatorTok\StringTok{ }\KeywordTok{group_by}\NormalTok{(s3classes) }\OperatorTok\StringTok{ }\KeywordTok{count}\NormalTok{(s3clglvl)}

\CommentTok{#continued on the next page}
\end{Highlighting}
\end{Shaded}

\end{frame}

\begin{frame}[fragile]{Creating analysis variables in the presence of
skip patterns}

Step 1b: Investigate relationship between input variables
continued\ldots{}

\begin{Shaded}
\begin{Highlighting}[]
\CommentTok{#add filter for s3classes is "missing" [-9]}
\NormalTok{hsls_small }\OperatorTok\StringTok{ }\KeywordTok{filter}\NormalTok{(x3sqstat }\OperatorTok{!=}\DecValTok{8}\NormalTok{,s3classes}\OperatorTok{==-}\DecValTok{9}\NormalTok{) }\OperatorTok\StringTok{ }\KeywordTok{group_by}\NormalTok{(s3classes) }\OperatorTok
\StringTok{  }\KeywordTok{count}\NormalTok{(s3clglvl)}
\CommentTok{#> # A tibble: 1 x 3}
\CommentTok{#> # Groups:   s3classes [1]}
\CommentTok{#>   s3classes s3clglvl      n}
\CommentTok{#>   <dbl+lbl> <dbl+lbl> <int>}
\CommentTok{#> 1 -9        -9           59}
\NormalTok{hsls_small }\OperatorTok\StringTok{ }\KeywordTok{filter}\NormalTok{(x3sqstat }\OperatorTok{!=}\DecValTok{8}\NormalTok{,s3classes}\OperatorTok{!=-}\DecValTok{9}\NormalTok{) }\OperatorTok\StringTok{ }\KeywordTok{group_by}\NormalTok{(s3classes) }\OperatorTok
\StringTok{  }\KeywordTok{count}\NormalTok{(s3clglvl)}
\CommentTok{#> # A tibble: 6 x 3}
\CommentTok{#> # Groups:   s3classes [3]}
\CommentTok{#>   s3classes s3clglvl      n}
\CommentTok{#>   <dbl+lbl> <dbl+lbl> <int>}
\CommentTok{#> 1 1         -9          428}
\CommentTok{#> 2 1         " 1"       8894}
\CommentTok{#> 3 1         " 2"       3929}
\CommentTok{#> 4 1         " 3"        226}
\CommentTok{#> 5 2         -7         3401}
\CommentTok{#> 6 3         -7         1621}

\CommentTok{#add filter for s3classes equal to "no" or "don't know"}
\NormalTok{hsls_small }\OperatorTok\StringTok{ }\KeywordTok{filter}\NormalTok{(x3sqstat }\OperatorTok{!=}\DecValTok{8}\NormalTok{,s3classes}\OperatorTok{!=-}\DecValTok{9}\NormalTok{, s3classes }\OperatorTok\StringTok{ }\KeywordTok{c}\NormalTok{(}\DecValTok{2}\NormalTok{,}\DecValTok{3}\NormalTok{)) }\OperatorTok
\StringTok{  }\KeywordTok{group_by}\NormalTok{(s3classes) }\OperatorTok\StringTok{ }\KeywordTok{count}\NormalTok{(s3clglvl)}
\CommentTok{#> # A tibble: 2 x 3}
\CommentTok{#> # Groups:   s3classes [2]}
\CommentTok{#>   s3classes s3clglvl      n}
\CommentTok{#>   <dbl+lbl> <dbl+lbl> <int>}
\CommentTok{#> 1 2         -7         3401}
\CommentTok{#> 2 3         -7         1621}
\NormalTok{hsls_small }\OperatorTok\StringTok{ }\KeywordTok{filter}\NormalTok{(x3sqstat }\OperatorTok{!=}\DecValTok{8}\NormalTok{,s3classes}\OperatorTok{!=-}\DecValTok{9}\NormalTok{, s3classes }\OperatorTok\StringTok{ }\KeywordTok{c}\NormalTok{(}\DecValTok{2}\NormalTok{,}\DecValTok{3}\NormalTok{)) }\OperatorTok\StringTok{ }
\StringTok{  }\KeywordTok{group_by}\NormalTok{(s3classes) }\OperatorTok\StringTok{ }\KeywordTok{count}\NormalTok{(s3clglvl) }\OperatorTok\StringTok{ }\KeywordTok{as_factor}\NormalTok{()}
\CommentTok{#> # A tibble: 2 x 3}
\CommentTok{#> # Groups:   s3classes [2]}
\CommentTok{#>   s3classes  s3clglvl                    n}
\CommentTok{#>   <fct>      <fct>                   <int>}
\CommentTok{#> 1 No         Item legitimate skip/NA  3401}
\CommentTok{#> 2 Don't know Item legitimate skip/NA  1621}

\NormalTok{hsls_small }\OperatorTok\StringTok{ }\KeywordTok{filter}\NormalTok{(x3sqstat }\OperatorTok{!=}\DecValTok{8}\NormalTok{,s3classes}\OperatorTok{!=-}\DecValTok{9}\NormalTok{, s3classes}\OperatorTok{==}\DecValTok{1}\NormalTok{) }\OperatorTok\StringTok{ }
\StringTok{  }\KeywordTok{group_by}\NormalTok{(s3classes) }\OperatorTok\StringTok{ }\KeywordTok{count}\NormalTok{(s3clglvl)}
\CommentTok{#> # A tibble: 4 x 3}
\CommentTok{#> # Groups:   s3classes [1]}
\CommentTok{#>   s3classes s3clglvl      n}
\CommentTok{#>   <dbl+lbl> <dbl+lbl> <int>}
\CommentTok{#> 1 1         -9          428}
\CommentTok{#> 2 1         " 1"       8894}
\CommentTok{#> 3 1         " 2"       3929}
\CommentTok{#> 4 1         " 3"        226}
\NormalTok{hsls_small }\OperatorTok\StringTok{ }\KeywordTok{filter}\NormalTok{(x3sqstat }\OperatorTok{!=}\DecValTok{8}\NormalTok{,s3classes}\OperatorTok{!=-}\DecValTok{9}\NormalTok{, s3classes}\OperatorTok{==}\DecValTok{1}\NormalTok{) }\OperatorTok\StringTok{ }
\StringTok{  }\KeywordTok{group_by}\NormalTok{(s3classes) }\OperatorTok\StringTok{ }\KeywordTok{count}\NormalTok{(s3clglvl) }\OperatorTok\StringTok{ }\KeywordTok{as_factor}\NormalTok{()}
\CommentTok{#> # A tibble: 4 x 3}
\CommentTok{#> # Groups:   s3classes [1]}
\CommentTok{#>   s3classes s3clglvl                                n}
\CommentTok{#>   <fct>     <fct>                               <int>}
\CommentTok{#> 1 Yes       Missing                               428}
\CommentTok{#> 2 Yes       4 or more years                      8894}
\CommentTok{#> 3 Yes       At least 2 but less than 4 years     3929}
\CommentTok{#> 4 Yes       Less than 2 years (below associate)   226}
\end{Highlighting}
\end{Shaded}

\end{frame}

\section{Appendix. Creating factor
variables}\label{appendix.-creating-factor-variables}

\begin{frame}[fragile]{Create factors {[}from string variables{]}}

To create a factor variable from string variable

\begin{enumerate}
\def\labelenumi{\arabic{enumi}.}
\tightlist
\item
  create a character vector containing underlying data
\item
  create a vector containing valid levels
\item
  Attach levels to the data using the \texttt{factor()} function
\end{enumerate}

\begin{Shaded}
\begin{Highlighting}[]
\CommentTok{#underlying data: months my fam is born}
\NormalTok{x1 <-}\StringTok{ }\KeywordTok{c}\NormalTok{(}\StringTok{"Jan"}\NormalTok{, }\StringTok{"Aug"}\NormalTok{, }\StringTok{"Apr"}\NormalTok{, }\StringTok{"Mar"}\NormalTok{)}
\CommentTok{#create vector with valid levels}
\NormalTok{month_levels <-}\StringTok{ }\KeywordTok{c}\NormalTok{(}\StringTok{"Jan"}\NormalTok{, }\StringTok{"Feb"}\NormalTok{, }\StringTok{"Mar"}\NormalTok{, }\StringTok{"Apr"}\NormalTok{, }\StringTok{"May"}\NormalTok{, }\StringTok{"Jun"}\NormalTok{, }
  \StringTok{"Jul"}\NormalTok{, }\StringTok{"Aug"}\NormalTok{, }\StringTok{"Sep"}\NormalTok{, }\StringTok{"Oct"}\NormalTok{, }\StringTok{"Nov"}\NormalTok{, }\StringTok{"Dec"}\NormalTok{)}
\CommentTok{#attach levels to data}
\NormalTok{x2 <-}\StringTok{ }\KeywordTok{factor}\NormalTok{(x1, }\DataTypeTok{levels =}\NormalTok{ month_levels)}
\end{Highlighting}
\end{Shaded}

Note how attributes differ

\begin{Shaded}
\begin{Highlighting}[]
\KeywordTok{str}\NormalTok{(x1)}
\CommentTok{#>  chr [1:4] "Jan" "Aug" "Apr" "Mar"}
\KeywordTok{str}\NormalTok{(x2)}
\CommentTok{#>  Factor w/ 12 levels "Jan","Feb","Mar",..: 1 8 4 3}
\end{Highlighting}
\end{Shaded}

Sorting differs

\begin{Shaded}
\begin{Highlighting}[]
\KeywordTok{sort}\NormalTok{(x1)}
\CommentTok{#> [1] "Apr" "Aug" "Jan" "Mar"}
\KeywordTok{sort}\NormalTok{(x2)}
\CommentTok{#> [1] Jan Mar Apr Aug}
\CommentTok{#> Levels: Jan Feb Mar Apr May Jun Jul Aug Sep Oct Nov Dec}
\end{Highlighting}
\end{Shaded}

\end{frame}

\begin{frame}[fragile]{Create factors {[}from string variables{]}}

Let's create a character version of variable \texttt{hs\_state} and then
turn it into a factor

\begin{Shaded}
\begin{Highlighting}[]
\CommentTok{#wwlist %>%}
\CommentTok{#  count(hs_state)}
\CommentTok{#Subset obs to West Coast states }
\NormalTok{wwlist_temp <-}\StringTok{ }\NormalTok{wwlist }\OperatorTok
\StringTok{  }\KeywordTok{filter}\NormalTok{(hs_state }\OperatorTok\StringTok{ }\KeywordTok{c}\NormalTok{(}\StringTok{"CA"}\NormalTok{, }\StringTok{"OR"}\NormalTok{, }\StringTok{"WA"}\NormalTok{))}

\CommentTok{#Create character version of high school state for West Coast states only}
\NormalTok{wwlist_temp}\OperatorTok{$}\NormalTok{hs_state_char <-}\StringTok{ }\KeywordTok{as.character}\NormalTok{(wwlist_temp}\OperatorTok{$}\NormalTok{hs_state)}

\CommentTok{#investigate character variable}
\KeywordTok{str}\NormalTok{(wwlist_temp}\OperatorTok{$}\NormalTok{hs_state_char)}
\KeywordTok{table}\NormalTok{(wwlist_temp}\OperatorTok{$}\NormalTok{hs_state_char)}

\CommentTok{#create new variable that assigns levels}
\NormalTok{wwlist_temp}\OperatorTok{$}\NormalTok{hs_state_fac <-}\StringTok{ }\KeywordTok{factor}\NormalTok{(wwlist_temp}\OperatorTok{$}\NormalTok{hs_state_char, }\DataTypeTok{levels =} \KeywordTok{c}\NormalTok{(}\StringTok{"CA"}\NormalTok{,}\StringTok{"OR"}\NormalTok{,}\StringTok{"WA"}\NormalTok{))}
\KeywordTok{str}\NormalTok{(wwlist_temp}\OperatorTok{$}\NormalTok{hs_state_fac)}

\CommentTok{#wwlist_temp %>%}
\CommentTok{#  count(hs_state_fac)}
\KeywordTok{rm}\NormalTok{(wwlist_temp)}
\end{Highlighting}
\end{Shaded}

\end{frame}

\begin{frame}[fragile]{Create factors {[}from string variables{]}}

How the \texttt{levels} argument works when underlying data is character

\begin{itemize}
\tightlist
\item
  Matches value of underlying data to value of the level attribute
\item
  Converts underlying data to integer, with level attribute attached
\end{itemize}

\medskip See chapter 15 of Wickham for more on factors (e.g., modifying
factor order, modifying factor levels)

\end{frame}

\begin{frame}[fragile]{Creating factors {[}from integer vectors{]}}

Factors are just integer vectors with level attributes attached to them.
So, to create a factor:

\begin{enumerate}
\def\labelenumi{\arabic{enumi}.}
\tightlist
\item
  create a vector for the underlying data
\item
  create a vector that has level attributes
\item
  Attach levels to the data using the \texttt{factor()} function
\end{enumerate}

\begin{Shaded}
\begin{Highlighting}[]
\NormalTok{a1 <-}\StringTok{ }\KeywordTok{c}\NormalTok{(}\DecValTok{1}\NormalTok{,}\DecValTok{1}\NormalTok{,}\DecValTok{1}\NormalTok{,}\DecValTok{0}\NormalTok{,}\DecValTok{1}\NormalTok{,}\DecValTok{1}\NormalTok{,}\DecValTok{0}\NormalTok{) }\CommentTok{#a vector of data}
\NormalTok{a2 <-}\StringTok{ }\KeywordTok{c}\NormalTok{(}\StringTok{"zero"}\NormalTok{,}\StringTok{"one"}\NormalTok{) }\CommentTok{#a vector of labels}

\CommentTok{#attach labels to values}
\NormalTok{a3 <-}\StringTok{ }\KeywordTok{factor}\NormalTok{(a1, }\DataTypeTok{labels =}\NormalTok{ a2)}
\NormalTok{a3}
\CommentTok{#> [1] one  one  one  zero one  one  zero}
\CommentTok{#> Levels: zero one}
\KeywordTok{str}\NormalTok{(a3)}
\CommentTok{#>  Factor w/ 2 levels "zero","one": 2 2 2 1 2 2 1}
\end{Highlighting}
\end{Shaded}

Note: By default, \texttt{factor()} function attached ``zero'' to the
lowest value of vector \texttt{a1} because ``zero'' was the first
element of vector \texttt{a2}

\end{frame}

\begin{frame}[fragile]{Creating factors {[}from integer vectors{]}}

Let's turn an integer variable into a factor variable in the
\texttt{wwlist} data frame

Create integer version of \texttt{receive\_year}

\begin{Shaded}
\begin{Highlighting}[]
\CommentTok{#typeof(wwlist_temp$receive_year)}
\NormalTok{wwlist}\OperatorTok{$}\NormalTok{receive_year_int <-}\StringTok{ }\KeywordTok{as.integer}\NormalTok{(wwlist}\OperatorTok{$}\NormalTok{receive_year)}
\KeywordTok{str}\NormalTok{(wwlist}\OperatorTok{$}\NormalTok{receive_year_int)}
\CommentTok{#>  int [1:268396] 2016 2016 2016 2016 2016 2016 2016 2016 2016 2016 ...}
\KeywordTok{typeof}\NormalTok{(wwlist}\OperatorTok{$}\NormalTok{receive_year_int)}
\CommentTok{#> [1] "integer"}
\end{Highlighting}
\end{Shaded}

Assign levels to values of integer variable

\begin{Shaded}
\begin{Highlighting}[]
\NormalTok{wwlist}\OperatorTok{$}\NormalTok{receive_year_fac <-}\StringTok{ }\KeywordTok{factor}\NormalTok{(wwlist}\OperatorTok{$}\NormalTok{receive_year_int, }\DataTypeTok{labels=}\KeywordTok{c}\NormalTok{(}\StringTok{"Twenty-sixteen"}\NormalTok{,}\StringTok{"Twenty-seventeen"}\NormalTok{,}\StringTok{"Twenty-eighteen"}\NormalTok{))}
\KeywordTok{str}\NormalTok{(wwlist}\OperatorTok{$}\NormalTok{receive_year_fac)}
\KeywordTok{str}\NormalTok{(wwlist}\OperatorTok{$}\NormalTok{receive_year)}

\CommentTok{#Check variable}
\NormalTok{wwlist }\OperatorTok
\StringTok{  }\KeywordTok{count}\NormalTok{(receive_year_fac)}

\NormalTok{wwlist }\OperatorTok
\StringTok{  }\KeywordTok{count}\NormalTok{(receive_year)}
\end{Highlighting}
\end{Shaded}

\end{frame}

\end{document}
