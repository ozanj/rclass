\documentclass[8pt,ignorenonframetext,dvipsnames]{beamer}
\setbeamertemplate{caption}[numbered]
\setbeamertemplate{caption label separator}{: }
\setbeamercolor{caption name}{fg=normal text.fg}
\beamertemplatenavigationsymbolsempty
\usepackage{lmodern}
\usepackage{amssymb,amsmath}
\usepackage{ifxetex,ifluatex}
\usepackage{fixltx2e} % provides \textsubscript
\ifnum 0\ifxetex 1\fi\ifluatex 1\fi=0 % if pdftex
  \usepackage[T1]{fontenc}
  \usepackage[utf8]{inputenc}
\else % if luatex or xelatex
  \ifxetex
    \usepackage{mathspec}
  \else
    \usepackage{fontspec}
  \fi
  \defaultfontfeatures{Ligatures=TeX,Scale=MatchLowercase}
\fi
% use upquote if available, for straight quotes in verbatim environments
\IfFileExists{upquote.sty}{\usepackage{upquote}}{}
% use microtype if available
\IfFileExists{microtype.sty}{%
\usepackage{microtype}
\UseMicrotypeSet[protrusion]{basicmath} % disable protrusion for tt fonts
}{}
\newif\ifbibliography
\hypersetup{
            pdftitle={Managing and Manipulating Data Using R},
            pdfauthor={Ozan Jaquette},
            pdfborder={0 0 0},
            breaklinks=true}
\urlstyle{same}  % don't use monospace font for urls
\usepackage{color}
\usepackage{fancyvrb}
\newcommand{\VerbBar}{|}
\newcommand{\VERB}{\Verb[commandchars=\\\{\}]}
\DefineVerbatimEnvironment{Highlighting}{Verbatim}{commandchars=\\\{\}}
% Add ',fontsize=\small' for more characters per line
\usepackage{framed}
\definecolor{shadecolor}{RGB}{248,248,248}
\newenvironment{Shaded}{\begin{snugshade}}{\end{snugshade}}
\newcommand{\KeywordTok}[1]{\textcolor[rgb]{0.13,0.29,0.53}{\textbf{#1}}}
\newcommand{\DataTypeTok}[1]{\textcolor[rgb]{0.13,0.29,0.53}{#1}}
\newcommand{\DecValTok}[1]{\textcolor[rgb]{0.00,0.00,0.81}{#1}}
\newcommand{\BaseNTok}[1]{\textcolor[rgb]{0.00,0.00,0.81}{#1}}
\newcommand{\FloatTok}[1]{\textcolor[rgb]{0.00,0.00,0.81}{#1}}
\newcommand{\ConstantTok}[1]{\textcolor[rgb]{0.00,0.00,0.00}{#1}}
\newcommand{\CharTok}[1]{\textcolor[rgb]{0.31,0.60,0.02}{#1}}
\newcommand{\SpecialCharTok}[1]{\textcolor[rgb]{0.00,0.00,0.00}{#1}}
\newcommand{\StringTok}[1]{\textcolor[rgb]{0.31,0.60,0.02}{#1}}
\newcommand{\VerbatimStringTok}[1]{\textcolor[rgb]{0.31,0.60,0.02}{#1}}
\newcommand{\SpecialStringTok}[1]{\textcolor[rgb]{0.31,0.60,0.02}{#1}}
\newcommand{\ImportTok}[1]{#1}
\newcommand{\CommentTok}[1]{\textcolor[rgb]{0.56,0.35,0.01}{\textit{#1}}}
\newcommand{\DocumentationTok}[1]{\textcolor[rgb]{0.56,0.35,0.01}{\textbf{\textit{#1}}}}
\newcommand{\AnnotationTok}[1]{\textcolor[rgb]{0.56,0.35,0.01}{\textbf{\textit{#1}}}}
\newcommand{\CommentVarTok}[1]{\textcolor[rgb]{0.56,0.35,0.01}{\textbf{\textit{#1}}}}
\newcommand{\OtherTok}[1]{\textcolor[rgb]{0.56,0.35,0.01}{#1}}
\newcommand{\FunctionTok}[1]{\textcolor[rgb]{0.00,0.00,0.00}{#1}}
\newcommand{\VariableTok}[1]{\textcolor[rgb]{0.00,0.00,0.00}{#1}}
\newcommand{\ControlFlowTok}[1]{\textcolor[rgb]{0.13,0.29,0.53}{\textbf{#1}}}
\newcommand{\OperatorTok}[1]{\textcolor[rgb]{0.81,0.36,0.00}{\textbf{#1}}}
\newcommand{\BuiltInTok}[1]{#1}
\newcommand{\ExtensionTok}[1]{#1}
\newcommand{\PreprocessorTok}[1]{\textcolor[rgb]{0.56,0.35,0.01}{\textit{#1}}}
\newcommand{\AttributeTok}[1]{\textcolor[rgb]{0.77,0.63,0.00}{#1}}
\newcommand{\RegionMarkerTok}[1]{#1}
\newcommand{\InformationTok}[1]{\textcolor[rgb]{0.56,0.35,0.01}{\textbf{\textit{#1}}}}
\newcommand{\WarningTok}[1]{\textcolor[rgb]{0.56,0.35,0.01}{\textbf{\textit{#1}}}}
\newcommand{\AlertTok}[1]{\textcolor[rgb]{0.94,0.16,0.16}{#1}}
\newcommand{\ErrorTok}[1]{\textcolor[rgb]{0.64,0.00,0.00}{\textbf{#1}}}
\newcommand{\NormalTok}[1]{#1}
\usepackage{longtable,booktabs}
\usepackage{caption}
% These lines are needed to make table captions work with longtable:
\makeatletter
\def\fnum@table{\tablename~\thetable}
\makeatother

% Prevent slide breaks in the middle of a paragraph:
\widowpenalties 1 10000
\raggedbottom

\AtBeginPart{
  \let\insertpartnumber\relax
  \let\partname\relax
  \frame{\partpage}
}
\AtBeginSection{
  \ifbibliography
  \else
    \let\insertsectionnumber\relax
    \let\sectionname\relax
    \frame{\sectionpage}
  \fi
}
\AtBeginSubsection{
  \let\insertsubsectionnumber\relax
  \let\subsectionname\relax
  \frame{\subsectionpage}
}

\setlength{\parindent}{0pt}
\setlength{\parskip}{6pt plus 2pt minus 1pt}
\setlength{\emergencystretch}{3em}  % prevent overfull lines
\providecommand{\tightlist}{%
  \setlength{\itemsep}{0pt}\setlength{\parskip}{0pt}}
\setcounter{secnumdepth}{0}

%packages
\usepackage{graphicx}
\usepackage{rotating}
\usepackage{hyperref}

\usepackage{tikz} % used for text highlighting, amongst others
%title slide stuff
%\institute{Department of Education}
%\title{Managing and Manipulating Data Using R}

%
\setbeamertemplate{navigation symbols}{} % get rid of navigation icons:

%\setbeamertemplate{frametitle}{\thesection \hspace{0.2cm} \insertframetitle}
\setbeamertemplate{section in toc}[sections numbered]
\setbeamertemplate{subsection in toc}[subsections numbered]

%define colors
%\definecolor{uva_orange}{RGB}{216,141,42} % UVa orange (Rotunda orange)
\definecolor{mygray}{rgb}{0.95, 0.95, 0.95} % for highlighted text
	% grey is equal parts red, green, blue. higher values >> lighter grey
	%\definecolor{lightgraybo}{rgb}{0.83, 0.83, 0.83}

% new commands

%highlight text with very light grey
\newcommand*{\hlg}[1]{%
	\tikz[baseline=(X.base)] \node[rectangle, fill=mygray] (X) {#1};%
}
%, inner sep=0.3mm
%highlight text with very light grey and use font associated with code
\newcommand*{\hlgc}[1]{\texttt{\hlg{#1}}}

% Font
\usepackage[defaultfam,light,tabular,lining]{montserrat}
\usepackage[T1]{fontenc}
\renewcommand*\oldstylenums[1]{{\fontfamily{Montserrat-TOsF}\selectfont #1}}

% Change color of boldface text to darkgray
\renewcommand{\textbf}[1]{{\color{darkgray}\bfseries\fontfamily{Montserrat-TOsF}#1}}

% Bullet points
\setbeamertemplate{itemize item}{\color{BlueViolet}$\circ$}
\setbeamertemplate{itemize subitem}{\color{BrickRed}$\triangleright$}
\setbeamertemplate{itemize subsubitem}{$-$}

% Reduce space before lists
\addtobeamertemplate{itemize/enumerate body begin}{}{\vspace*{-8pt}}

\title{Managing and Manipulating Data Using R}
\subtitle{Lecture 3}
\author{Ozan Jaquette}
\date{}

\begin{document}
\frame{\titlepage}

\begin{frame}
\tableofcontents[hideallsubsections]
\end{frame}

\begin{frame}

\end{frame}

\section{Introduction/logistics}\label{introductionlogistics}

\begin{frame}[fragile]{Libraries we will use today}

\begin{Shaded}
\begin{Highlighting}[]
\KeywordTok{library}\NormalTok{(tidyverse)}
\CommentTok{#> -- Attaching packages ------------------------------------------------------------------ tidyverse 1.2.1 --}
\CommentTok{#> v ggplot2 3.0.0     v purrr   0.2.5}
\CommentTok{#> v tibble  1.4.2     v dplyr   0.7.6}
\CommentTok{#> v tidyr   0.8.1     v stringr 1.3.1}
\CommentTok{#> v readr   1.1.1     v forcats 0.3.0}
\CommentTok{#> -- Conflicts --------------------------------------------------------------------- tidyverse_conflicts() --}
\CommentTok{#> x dplyr::filter() masks stats::filter()}
\CommentTok{#> x dplyr::lag()    masks stats::lag()}
\end{Highlighting}
\end{Shaded}

\end{frame}

\begin{frame}[fragile]{Data we will use today}

Data on off-campus recruiting events by public universities

\begin{Shaded}
\begin{Highlighting}[]
\KeywordTok{rm}\NormalTok{(}\DataTypeTok{list =} \KeywordTok{ls}\NormalTok{()) }\CommentTok{# remove all objects}

\CommentTok{#load dataset with one obs per recruiting event}
\KeywordTok{load}\NormalTok{(}\StringTok{"../../data/recruiting/recruit_event_somevars.Rdata"}\NormalTok{)}

\CommentTok{#load dataset with one obs per high school}
\KeywordTok{load}\NormalTok{(}\StringTok{"../../data/recruiting/recruit_school_somevars.Rdata"}\NormalTok{)}

\KeywordTok{load}\NormalTok{(}\StringTok{"../../data/prospect_list/western_washington_college_board_list.RData"}\NormalTok{)}
\end{Highlighting}
\end{Shaded}

Object \textbackslash{}hlgc\{df\_event\}

\begin{itemize}
\tightlist
\item
  One observation per university, recruiting event
\end{itemize}

Object \textbackslash{}hlgc\{df\_event\}

\begin{itemize}
\tightlist
\item
  One observation per high school (visited and non-visited)
\end{itemize}

\end{frame}

\section{Factors}\label{factors}

\subsection{Review data types and
structures}\label{review-data-types-and-structures}

\begin{frame}{Review data types}

Primary \textbf{data types} in R:

\begin{itemize}
\tightlist
\item
  numeric (integer \& double)
\item
  character
\item
  logical
\end{itemize}

R CODE CHUNK WITH EXAMPLES

\end{frame}

\begin{frame}[fragile]{Review data structures: vectors}

Primary \textbf{data structures} in R are \textbf{vectors} and
\textbf{lists}

\medskip A \textbf{vector} is a collection of values

\begin{itemize}
\tightlist
\item
  each value in a vector is an \textbf{element}
\item
  all elements within vector must have same \textbf{data type}
\end{itemize}

\begin{Shaded}
\begin{Highlighting}[]
\NormalTok{a <-}\StringTok{ }\KeywordTok{c}\NormalTok{(}\DecValTok{1}\NormalTok{,}\DecValTok{2}\NormalTok{,}\DecValTok{3}\NormalTok{)}
\NormalTok{a}
\CommentTok{#> [1] 1 2 3}
\KeywordTok{str}\NormalTok{(a)}
\CommentTok{#>  num [1:3] 1 2 3}
\end{Highlighting}
\end{Shaded}

You can assign \textbf{names} to elements of a vector, thereby creating
a \textbf{named vector}

\begin{Shaded}
\begin{Highlighting}[]
\NormalTok{b <-}\StringTok{ }\KeywordTok{c}\NormalTok{(}\DataTypeTok{v1=}\DecValTok{1}\NormalTok{,}\DataTypeTok{v2=}\DecValTok{2}\NormalTok{,}\DataTypeTok{v3=}\DecValTok{3}\NormalTok{)}
\NormalTok{b}
\CommentTok{#> v1 v2 v3 }
\CommentTok{#>  1  2  3}
\KeywordTok{str}\NormalTok{(b)}
\CommentTok{#>  Named num [1:3] 1 2 3}
\CommentTok{#>  - attr(*, "names")= chr [1:3] "v1" "v2" "v3"}
\end{Highlighting}
\end{Shaded}

\end{frame}

\begin{frame}[fragile]{Review data structures: lists}

Like vectors, \textbf{lists} are objects that contain \textbf{elements};
However, \textbf{data type} can differ across elements within a list; an
element of a list can be another list

Examples of lists:

\begin{Shaded}
\begin{Highlighting}[]
\NormalTok{list_a <-}\StringTok{ }\KeywordTok{list}\NormalTok{(}\DecValTok{1}\NormalTok{,}\DecValTok{2}\NormalTok{,}\StringTok{"apple"}\NormalTok{)}
\KeywordTok{str}\NormalTok{(list_a)}
\CommentTok{#> List of 3}
\CommentTok{#>  $ : num 1}
\CommentTok{#>  $ : num 2}
\CommentTok{#>  $ : chr "apple"}
\NormalTok{list_b <-}\StringTok{ }\KeywordTok{list}\NormalTok{(}\DecValTok{1}\NormalTok{, }\KeywordTok{c}\NormalTok{(}\StringTok{"apple"}\NormalTok{, }\StringTok{"orange"}\NormalTok{), }\KeywordTok{list}\NormalTok{(}\DecValTok{1}\NormalTok{, }\DecValTok{2}\NormalTok{, }\DecValTok{3}\NormalTok{))}
\KeywordTok{str}\NormalTok{(list_b)}
\CommentTok{#> List of 3}
\CommentTok{#>  $ : num 1}
\CommentTok{#>  $ : chr [1:2] "apple" "orange"}
\CommentTok{#>  $ :List of 3}
\CommentTok{#>   ..$ : num 1}
\CommentTok{#>   ..$ : num 2}
\CommentTok{#>   ..$ : num 3}
\end{Highlighting}
\end{Shaded}

\end{frame}

\begin{frame}[fragile]{Review data structures: lists}

Like vectors, elements within a list can be named, thereby creating a
\textbf{named list}

\begin{Shaded}
\begin{Highlighting}[]
\KeywordTok{str}\NormalTok{(list_b) }\CommentTok{# not named}
\CommentTok{#> List of 3}
\CommentTok{#>  $ : num 1}
\CommentTok{#>  $ : chr [1:2] "apple" "orange"}
\CommentTok{#>  $ :List of 3}
\CommentTok{#>   ..$ : num 1}
\CommentTok{#>   ..$ : num 2}
\CommentTok{#>   ..$ : num 3}

\NormalTok{list_c <-}\StringTok{ }\KeywordTok{list}\NormalTok{(}\DataTypeTok{v1=}\DecValTok{1}\NormalTok{, }\DataTypeTok{v2=}\KeywordTok{c}\NormalTok{(}\StringTok{"apple"}\NormalTok{, }\StringTok{"orange"}\NormalTok{), }\DataTypeTok{v3=}\KeywordTok{list}\NormalTok{(}\DecValTok{1}\NormalTok{, }\DecValTok{2}\NormalTok{, }\DecValTok{3}\NormalTok{))}
\KeywordTok{str}\NormalTok{(list_c) }\CommentTok{# named}
\CommentTok{#> List of 3}
\CommentTok{#>  $ v1: num 1}
\CommentTok{#>  $ v2: chr [1:2] "apple" "orange"}
\CommentTok{#>  $ v3:List of 3}
\CommentTok{#>   ..$ : num 1}
\CommentTok{#>   ..$ : num 2}
\CommentTok{#>   ..$ : num 3}
\end{Highlighting}
\end{Shaded}

\end{frame}

\begin{frame}[fragile]{Review data structures: a data frame is a list}

A \textbf{data frame} is a list with the following characteristics:

\begin{itemize}
\tightlist
\item
  All the elements must be \textbf{vectors} with the same
  \textbf{length}
\item
  Data frames are \textbf{augmented lists} because they have additional
  \textbf{attributes} {[}described later{]}
\end{itemize}

\begin{Shaded}
\begin{Highlighting}[]
\NormalTok{list_d <-}\StringTok{ }\KeywordTok{list}\NormalTok{(}\DataTypeTok{col_a =} \KeywordTok{c}\NormalTok{(}\DecValTok{1}\NormalTok{,}\DecValTok{2}\NormalTok{,}\DecValTok{3}\NormalTok{), }\DataTypeTok{col_b =} \KeywordTok{c}\NormalTok{(}\DecValTok{4}\NormalTok{,}\DecValTok{5}\NormalTok{,}\DecValTok{6}\NormalTok{), }\DataTypeTok{col_c =} \KeywordTok{c}\NormalTok{(}\DecValTok{7}\NormalTok{,}\DecValTok{8}\NormalTok{,}\DecValTok{9}\NormalTok{))}
\KeywordTok{typeof}\NormalTok{(list_d)}
\CommentTok{#> [1] "list"}
\KeywordTok{str}\NormalTok{(list_d)}
\CommentTok{#> List of 3}
\CommentTok{#>  $ col_a: num [1:3] 1 2 3}
\CommentTok{#>  $ col_b: num [1:3] 4 5 6}
\CommentTok{#>  $ col_c: num [1:3] 7 8 9}

\NormalTok{df_a <-}\StringTok{ }\KeywordTok{data.frame}\NormalTok{(}\DataTypeTok{col_a =} \KeywordTok{c}\NormalTok{(}\DecValTok{1}\NormalTok{,}\DecValTok{2}\NormalTok{,}\DecValTok{3}\NormalTok{), }\DataTypeTok{col_b =} \KeywordTok{c}\NormalTok{(}\DecValTok{4}\NormalTok{,}\DecValTok{5}\NormalTok{,}\DecValTok{6}\NormalTok{), }\DataTypeTok{col_c =} \KeywordTok{c}\NormalTok{(}\DecValTok{7}\NormalTok{,}\DecValTok{8}\NormalTok{,}\DecValTok{9}\NormalTok{))}
\KeywordTok{typeof}\NormalTok{(df_a)}
\CommentTok{#> [1] "list"}
\KeywordTok{str}\NormalTok{(df_a)}
\CommentTok{#> 'data.frame':    3 obs. of  3 variables:}
\CommentTok{#>  $ col_a: num  1 2 3}
\CommentTok{#>  $ col_b: num  4 5 6}
\CommentTok{#>  $ col_c: num  7 8 9}
\end{Highlighting}
\end{Shaded}

\end{frame}

\subsection{Attributes and augmented
vectors}\label{attributes-and-augmented-vectors}

\begin{frame}{Atomic vectors versus augmented vectors}

\textbf{Atomic vectors} {[}our focus so far{]} -
\href{http://r4ds.had.co.nz/diagrams/data-structures-overview.png}{(See
figure)} - I think of atomic vectors as ``just the data'' - Atomic
vectors are the building blocks for augmented vectors

Augmented vectors

\begin{itemize}
\tightlist
\item
  \textbf{Augmented vectors} are atomic vectors with additional
  \textbf{atributes} attached
\end{itemize}

\textbf{Attributes}

\begin{itemize}
\tightlist
\item
  \textbf{Attributes} are additional ``metadata'' that can be attached
  to any object (e.g., vector or list)
\item
  Important attributes in R:

  \begin{itemize}
  \tightlist
  \item
    \textbf{Names}: name the elements of a vector (e.g., variable names)
  \item
    \textbf{Class}: How object should be treated by object oriented
    programming language {[}discussed below{]}
  \end{itemize}
\end{itemize}

Main takaway:

\begin{itemize}
\tightlist
\item
  Augmented vectors are atomic vectors (just the data) with additional
  attributes attached
\end{itemize}

\end{frame}

\begin{frame}[fragile]{Attributes in vectors}

\begin{Shaded}
\begin{Highlighting}[]
\NormalTok{vector1 <-}\StringTok{ }\KeywordTok{c}\NormalTok{(}\DecValTok{1}\NormalTok{,}\DecValTok{2}\NormalTok{,}\DecValTok{3}\NormalTok{,}\DecValTok{4}\NormalTok{)}
\NormalTok{vector1}
\CommentTok{#> [1] 1 2 3 4}
\KeywordTok{attributes}\NormalTok{(vector1)}
\CommentTok{#> NULL}

\NormalTok{vector2 <-}\StringTok{ }\KeywordTok{c}\NormalTok{(}\DataTypeTok{a =} \DecValTok{1}\NormalTok{, }\DataTypeTok{b=} \DecValTok{2}\NormalTok{, }\DataTypeTok{c=} \DecValTok{3}\NormalTok{, }\DataTypeTok{d =} \DecValTok{4}\NormalTok{)}
\NormalTok{vector2}
\CommentTok{#> a b c d }
\CommentTok{#> 1 2 3 4}
\KeywordTok{attributes}\NormalTok{(vector2)}
\CommentTok{#> $names}
\CommentTok{#> [1] "a" "b" "c" "d"}
\end{Highlighting}
\end{Shaded}

\end{frame}

\begin{frame}[fragile]{Attributes in lists}

\begin{Shaded}
\begin{Highlighting}[]
\NormalTok{list1 <-}\StringTok{ }\KeywordTok{list}\NormalTok{(}\KeywordTok{c}\NormalTok{(}\DecValTok{1}\NormalTok{,}\DecValTok{2}\NormalTok{,}\DecValTok{3}\NormalTok{), }\KeywordTok{c}\NormalTok{(}\DecValTok{4}\NormalTok{,}\DecValTok{5}\NormalTok{,}\DecValTok{6}\NormalTok{))}
\KeywordTok{str}\NormalTok{(list1)}
\CommentTok{#> List of 2}
\CommentTok{#>  $ : num [1:3] 1 2 3}
\CommentTok{#>  $ : num [1:3] 4 5 6}
\KeywordTok{attributes}\NormalTok{(list1)}
\CommentTok{#> NULL}

\NormalTok{list2 <-}\StringTok{ }\KeywordTok{list}\NormalTok{(}\DataTypeTok{col_a =} \KeywordTok{c}\NormalTok{(}\DecValTok{1}\NormalTok{,}\DecValTok{2}\NormalTok{,}\DecValTok{3}\NormalTok{), }\DataTypeTok{col_b =} \KeywordTok{c}\NormalTok{(}\DecValTok{4}\NormalTok{,}\DecValTok{5}\NormalTok{,}\DecValTok{6}\NormalTok{))}
\KeywordTok{str}\NormalTok{(list2)}
\CommentTok{#> List of 2}
\CommentTok{#>  $ col_a: num [1:3] 1 2 3}
\CommentTok{#>  $ col_b: num [1:3] 4 5 6}
\KeywordTok{attributes}\NormalTok{(list2)}
\CommentTok{#> $names}
\CommentTok{#> [1] "col_a" "col_b"}

\NormalTok{list3 <-}\StringTok{ }\KeywordTok{data.frame}\NormalTok{(}\DataTypeTok{col_a =} \KeywordTok{c}\NormalTok{(}\DecValTok{1}\NormalTok{,}\DecValTok{2}\NormalTok{,}\DecValTok{3}\NormalTok{), }\DataTypeTok{col_b =} \KeywordTok{c}\NormalTok{(}\DecValTok{4}\NormalTok{,}\DecValTok{5}\NormalTok{,}\DecValTok{6}\NormalTok{))}
\KeywordTok{str}\NormalTok{(list3)}
\CommentTok{#> 'data.frame':    3 obs. of  2 variables:}
\CommentTok{#>  $ col_a: num  1 2 3}
\CommentTok{#>  $ col_b: num  4 5 6}
\KeywordTok{attributes}\NormalTok{(list3)}
\CommentTok{#> $names}
\CommentTok{#> [1] "col_a" "col_b"}
\CommentTok{#> }
\CommentTok{#> $class}
\CommentTok{#> [1] "data.frame"}
\CommentTok{#> }
\CommentTok{#> $row.names}
\CommentTok{#> [1] 1 2 3}
\end{Highlighting}
\end{Shaded}

\end{frame}

\begin{frame}[fragile]{Object class}

\begin{Shaded}
\begin{Highlighting}[]
\NormalTok{vector1 <-}\StringTok{ }\KeywordTok{c}\NormalTok{(}\DecValTok{1}\NormalTok{,}\DecValTok{2}\NormalTok{,}\DecValTok{3}\NormalTok{,}\DecValTok{4}\NormalTok{)}
\NormalTok{vector1}
\CommentTok{#> [1] 1 2 3 4}
\KeywordTok{typeof}\NormalTok{(vector1)}
\CommentTok{#> [1] "double"}
\KeywordTok{class}\NormalTok{(vector1)}
\CommentTok{#> [1] "numeric"}
\KeywordTok{attributes}\NormalTok{(vector1)}
\CommentTok{#> NULL}

\NormalTok{vector2 <-}\StringTok{ }\KeywordTok{c}\NormalTok{(}\DataTypeTok{a =} \DecValTok{1}\NormalTok{, }\DataTypeTok{b=} \DecValTok{2}\NormalTok{, }\DataTypeTok{c=} \DecValTok{3}\NormalTok{, }\DataTypeTok{d =} \DecValTok{4}\NormalTok{)}
\NormalTok{vector2}
\CommentTok{#> a b c d }
\CommentTok{#> 1 2 3 4}
\KeywordTok{attributes}\NormalTok{(vector2)}
\CommentTok{#> $names}
\CommentTok{#> [1] "a" "b" "c" "d"}
\KeywordTok{typeof}\NormalTok{(vector2)}
\CommentTok{#> [1] "double"}
\KeywordTok{class}\NormalTok{(vector2)}
\CommentTok{#> [1] "numeric"}
\end{Highlighting}
\end{Shaded}

\end{frame}

\section{FACTORS}\label{factors-1}

\begin{frame}[fragile]{Factors}

\textbf{Factors} are used to display categorical data (e.g., marital
status)

\begin{itemize}
\tightlist
\item
  A factor is an \textbf{augmented vector} built by attaching a
  ``levels'' attribute to an (atomic) integer vectors
\end{itemize}

The \hlgc{str()} function is useful for identifying which variables are
factors. Let's examine the factor variable \hlgc{ethn\_code}

\begin{Shaded}
\begin{Highlighting}[]
\KeywordTok{typeof}\NormalTok{(wwlist}\OperatorTok{$}\NormalTok{ethn_code)}
\CommentTok{#> [1] "integer"}
\KeywordTok{class}\NormalTok{(wwlist}\OperatorTok{$}\NormalTok{ethn_code)}
\CommentTok{#> [1] "factor"}
\KeywordTok{str}\NormalTok{(wwlist}\OperatorTok{$}\NormalTok{ethn_code)}
\CommentTok{#>  Factor w/ 11 levels "American Indian or Alaska Native",..: 8 11 11 8 11 8 8 8 8 11 ...}
\end{Highlighting}
\end{Shaded}

Note that \hlgc{ethn\_code} has \hlgc{type=integer} and
\hlgc{class=factor} because the variable has a ``levels'' attribute

\begin{Shaded}
\begin{Highlighting}[]
\KeywordTok{attributes}\NormalTok{(wwlist}\OperatorTok{$}\NormalTok{ethn_code)}
\end{Highlighting}
\end{Shaded}

Main takeaway:

\begin{itemize}
\tightlist
\item
  The underlying data are integers but the levels attribute is used to
  display the data.
\end{itemize}

\end{frame}

\begin{frame}[fragile]{Working with factor variables}

\begin{Shaded}
\begin{Highlighting}[]
\KeywordTok{attributes}\NormalTok{(wwlist}\OperatorTok{$}\NormalTok{ethn_code)}
\end{Highlighting}
\end{Shaded}

Refer to categories of a factor by the values of the level attribute
rather than the underlying values of the variable

\begin{Shaded}
\begin{Highlighting}[]
\KeywordTok{count}\NormalTok{(}\KeywordTok{filter}\NormalTok{(wwlist,ethn_code}\OperatorTok{==}\DecValTok{11}\NormalTok{))}
\CommentTok{#> # A tibble: 1 x 1}
\CommentTok{#>       n}
\CommentTok{#>   <int>}
\CommentTok{#> 1     0}

\KeywordTok{count}\NormalTok{(}\KeywordTok{filter}\NormalTok{(wwlist,ethn_code}\OperatorTok{==}\StringTok{"White"}\NormalTok{))}
\CommentTok{#> # A tibble: 1 x 1}
\CommentTok{#>        n}
\CommentTok{#>    <int>}
\CommentTok{#> 1 159680}
\end{Highlighting}
\end{Shaded}

If you want to refer to underlying values, then apply
\hlgc{as.integer()} function to the factor variable

\begin{Shaded}
\begin{Highlighting}[]
\KeywordTok{count}\NormalTok{(}\KeywordTok{filter}\NormalTok{(wwlist,}\KeywordTok{as.integer}\NormalTok{(ethn_code)}\OperatorTok{==}\DecValTok{11}\NormalTok{))}
\CommentTok{#> # A tibble: 1 x 1}
\CommentTok{#>        n}
\CommentTok{#>    <int>}
\CommentTok{#> 1 159680}
\end{Highlighting}
\end{Shaded}

\end{frame}

\begin{frame}[fragile]{How to identify the variable values associated
with factor levels}

MAYBE CUT THIS SLIDE IF YOU CAN'T DO THIS WITHOUT PIPES

\begin{Shaded}
\begin{Highlighting}[]

\NormalTok{wwlist }\OperatorTok\StringTok{ }\KeywordTok{count}\NormalTok{(psat_range) }\OperatorTok\StringTok{ }\KeywordTok{as_factor}\NormalTok{()}
\CommentTok{#> # A tibble: 8 x 2}
\CommentTok{#>   psat_range     n}
\CommentTok{#>   <fct>      <int>}
\CommentTok{#> 1 1030-1160  45708}
\CommentTok{#> 2 1030-1520  67192}
\CommentTok{#> 3 1170-1520  48982}
\CommentTok{#> 4 1270-1520   8348}
\CommentTok{#> 5 930-1160   17387}
\CommentTok{#> 6 930-1260   15660}
\CommentTok{#> 7 990-1260   27628}
\CommentTok{#> 8 <NA>       37491}
\end{Highlighting}
\end{Shaded}

\begin{Shaded}
\begin{Highlighting}[]
\KeywordTok{count}\NormalTok{(}\KeywordTok{filter}\NormalTok{(wwlist,}\KeywordTok{as.integer}\NormalTok{(psat_range)}\OperatorTok{==}\DecValTok{4}\NormalTok{))}
\CommentTok{#> # A tibble: 1 x 1}
\CommentTok{#>       n}
\CommentTok{#>   <int>}
\CommentTok{#> 1  8348}
\KeywordTok{count}\NormalTok{(}\KeywordTok{filter}\NormalTok{(wwlist,psat_range}\OperatorTok{==}\StringTok{"1270-1520"}\NormalTok{))}
\CommentTok{#> # A tibble: 1 x 1}
\CommentTok{#>       n}
\CommentTok{#>   <int>}
\CommentTok{#> 1  8348}
\end{Highlighting}
\end{Shaded}

\end{frame}

\begin{frame}[fragile]{Some in-class exercise involving factors}

\begin{Shaded}
\begin{Highlighting}[]
\KeywordTok{str}\NormalTok{(wwlist)}
\CommentTok{#> Classes 'tbl_df', 'tbl' and 'data.frame':    268396 obs. of  19 variables:}
\CommentTok{#>  $ receive_date    : Date, format: "2016-05-31" "2016-05-31" ...}
\CommentTok{#>  $ psat_range      : Factor w/ 7 levels "1030-1160","1030-1520",..: 5 4 7 3 7 3 3 3 5 1 ...}
\CommentTok{#>  $ sat_range       : Factor w/ 3 levels "1030-1600","930-1600",..: NA NA NA NA NA NA NA NA NA NA ...}
\CommentTok{#>  $ ap_range        : Factor w/ 2 levels "1 or higher",..: NA NA NA NA NA NA NA NA NA NA ...}
\CommentTok{#>  $ gpa_b_aplus     : Factor w/ 1 level "x": 1 1 1 1 1 1 1 1 1 NA ...}
\CommentTok{#>  $ gpa_b_aplus_null: Factor w/ 1 level "x": NA NA NA NA NA NA NA NA NA NA ...}
\CommentTok{#>  $ gpa_bplus_aplus : Factor w/ 1 level "x": NA NA NA NA NA NA NA NA NA 1 ...}
\CommentTok{#>  $ state           : chr  "WA" "WA" "WA" "WA" ...}
\CommentTok{#>  $ zip             : chr  "98103-3528" "98030-7964" "98290-8659" "98105-0002" ...}
\CommentTok{#>  $ for_country     : chr  NA NA NA NA ...}
\CommentTok{#>  $ sex             : Factor w/ 3 levels "F","M","U": 2 1 2 1 1 2 2 1 2 2 ...}
\CommentTok{#>  $ hs_ceeb_code    : int  481112 480539 480391 481115 480585 481080 480118 481128 481335 130595 ...}
\CommentTok{#>  $ hs_name         : chr  "Ingraham High School" "Kentwood Senior High School" "Archbishop Thomas J Murphy HS" "Garfield High School" ...}
\CommentTok{#>  $ hs_city         : chr  "Seattle" "Covington" "Everett" "Seattle" ...}
\CommentTok{#>  $ hs_state        : chr  "WA" "WA" "WA" "WA" ...}
\CommentTok{#>  $ hs_grad_date    : Date, format: "2018-06-01" "2017-06-01" ...}
\CommentTok{#>  $ ethn_code       : Factor w/ 11 levels "American Indian or Alaska Native",..: 8 11 11 8 11 8 8 8 8 11 ...}
\CommentTok{#>  $ homeschool      : Factor w/ 2 levels "N","Y": 1 1 1 1 1 1 1 1 1 1 ...}
\CommentTok{#>  $ firstgen        : Factor w/ 2 levels "N","Y": NA 1 1 1 NA 1 1 2 2 1 ...}
\end{Highlighting}
\end{Shaded}

\end{frame}

\begin{frame}[fragile]{Creating factors {[}from integer vectors{]}}

Factors are just integer vectors with level attributes attached to them.
So, to create a factor:

\begin{enumerate}
\def\labelenumi{\arabic{enumi}.}
\tightlist
\item
  create a vector for the underlying data
\item
  create a vector that has level attributes
\item
  Attach levels to the data using the \hlgc{factor()} function
\end{enumerate}

\begin{Shaded}
\begin{Highlighting}[]
\NormalTok{a1 <-}\StringTok{ }\KeywordTok{c}\NormalTok{(}\DecValTok{1}\NormalTok{,}\DecValTok{1}\NormalTok{,}\DecValTok{1}\NormalTok{,}\DecValTok{0}\NormalTok{,}\DecValTok{1}\NormalTok{,}\DecValTok{1}\NormalTok{,}\DecValTok{0}\NormalTok{) }\CommentTok{#a vector of data}
\NormalTok{a2 <-}\StringTok{ }\KeywordTok{c}\NormalTok{(}\StringTok{"zero"}\NormalTok{,}\StringTok{"one"}\NormalTok{) }\CommentTok{#a vector of labels}

\CommentTok{#attach labels to values}
\NormalTok{a3 <-}\StringTok{ }\KeywordTok{factor}\NormalTok{(a1, }\DataTypeTok{labels =}\NormalTok{ a2)}
\NormalTok{a3}
\CommentTok{#> [1] one  one  one  zero one  one  zero}
\CommentTok{#> Levels: zero one}
\KeywordTok{str}\NormalTok{(a3)}
\CommentTok{#>  Factor w/ 2 levels "zero","one": 2 2 2 1 2 2 1}
\end{Highlighting}
\end{Shaded}

Note: By default, \hlgc{factor()} function attached ``zero'' to the
lowest value of vector \hlgc{a1} because ``zero'' was the first element
of vector \hlgc{a2}

\end{frame}

\begin{frame}[fragile]{Creating factors {[}from integer vectors{]}}

Let's turn an integer variable into a factor variable in the
\hlgc{wwlist} data frame

Create integer version of \hlgc{sex}

\begin{Shaded}
\begin{Highlighting}[]
\NormalTok{wwlist}\OperatorTok{$}\NormalTok{sex_int <-}\StringTok{ }\KeywordTok{as.integer}\NormalTok{(wwlist}\OperatorTok{$}\NormalTok{sex)}
\KeywordTok{str}\NormalTok{(wwlist}\OperatorTok{$}\NormalTok{sex_int)}
\CommentTok{#>  int [1:268396] 2 1 2 1 1 2 2 1 2 2 ...}
\CommentTok{#wwlist %>% count(sex) %>% as_factor()}
\end{Highlighting}
\end{Shaded}

Assume we know that 1=female, 2=male, 3=unknown

Assign levels to values of integer variable

\begin{Shaded}
\begin{Highlighting}[]
\NormalTok{wwlist}\OperatorTok{$}\NormalTok{sex_int <-}\StringTok{ }\KeywordTok{factor}\NormalTok{(wwlist}\OperatorTok{$}\NormalTok{sex_int, }\DataTypeTok{labels=}\KeywordTok{c}\NormalTok{(}\StringTok{"female"}\NormalTok{,}\StringTok{"male"}\NormalTok{,}\StringTok{"unknown"}\NormalTok{))}
\KeywordTok{str}\NormalTok{(wwlist}\OperatorTok{$}\NormalTok{sex_int)}
\CommentTok{#>  Factor w/ 3 levels "female","male",..: 2 1 2 1 1 2 2 1 2 2 ...}
\KeywordTok{str}\NormalTok{(wwlist}\OperatorTok{$}\NormalTok{sex)}
\CommentTok{#>  Factor w/ 3 levels "F","M","U": 2 1 2 1 1 2 2 1 2 2 ...}
\end{Highlighting}
\end{Shaded}

\end{frame}

\begin{frame}[fragile]{Create factors {[}from string variables{]}}

To create a factor variable from string variable

\begin{enumerate}
\def\labelenumi{\arabic{enumi}.}
\tightlist
\item
  create a character vector containing underlying data
\item
  create a vector containing valid levels
\item
  Attach levels to the data using the \hlgc{factor()} function
\end{enumerate}

\begin{Shaded}
\begin{Highlighting}[]
\CommentTok{#underlying data: months my fam is born}
\NormalTok{x1 <-}\StringTok{ }\KeywordTok{c}\NormalTok{(}\StringTok{"Jan"}\NormalTok{, }\StringTok{"Aug"}\NormalTok{, }\StringTok{"Apr"}\NormalTok{, }\StringTok{"Mar"}\NormalTok{)}
\CommentTok{#create vector with valid levels}
\NormalTok{month_levels <-}\StringTok{ }\KeywordTok{c}\NormalTok{(}\StringTok{"Jan"}\NormalTok{, }\StringTok{"Feb"}\NormalTok{, }\StringTok{"Mar"}\NormalTok{, }\StringTok{"Apr"}\NormalTok{, }\StringTok{"May"}\NormalTok{, }\StringTok{"Jun"}\NormalTok{, }
  \StringTok{"Jul"}\NormalTok{, }\StringTok{"Aug"}\NormalTok{, }\StringTok{"Sep"}\NormalTok{, }\StringTok{"Oct"}\NormalTok{, }\StringTok{"Nov"}\NormalTok{, }\StringTok{"Dec"}\NormalTok{)}
\CommentTok{#attach levels to data}
\NormalTok{x2 <-}\StringTok{ }\KeywordTok{factor}\NormalTok{(x1, }\DataTypeTok{levels =}\NormalTok{ month_levels)}
\end{Highlighting}
\end{Shaded}

Note how attributes differ

\begin{Shaded}
\begin{Highlighting}[]
\KeywordTok{str}\NormalTok{(x1)}
\CommentTok{#>  chr [1:4] "Jan" "Aug" "Apr" "Mar"}
\KeywordTok{str}\NormalTok{(x2)}
\CommentTok{#>  Factor w/ 12 levels "Jan","Feb","Mar",..: 1 8 4 3}
\end{Highlighting}
\end{Shaded}

Sorting differs

\begin{Shaded}
\begin{Highlighting}[]
\KeywordTok{sort}\NormalTok{(x1)}
\CommentTok{#> [1] "Apr" "Aug" "Jan" "Mar"}
\KeywordTok{sort}\NormalTok{(x2)}
\CommentTok{#> [1] Jan Mar Apr Aug}
\CommentTok{#> Levels: Jan Feb Mar Apr May Jun Jul Aug Sep Oct Nov Dec}
\end{Highlighting}
\end{Shaded}

\end{frame}

\begin{frame}[fragile]{Create factors {[}from string variables{]}}

Let's create a character version of variable \hlgc{sex} and then turn it
into a factor

\begin{Shaded}
\begin{Highlighting}[]
\CommentTok{#Create character version of sex}
\NormalTok{wwlist}\OperatorTok{$}\NormalTok{sex_char <-}\StringTok{ }\KeywordTok{as.character}\NormalTok{(wwlist}\OperatorTok{$}\NormalTok{sex)}

\CommentTok{#investigate character variable}
\KeywordTok{str}\NormalTok{(wwlist}\OperatorTok{$}\NormalTok{sex_char)}
\CommentTok{#>  chr [1:268396] "M" "F" "M" "F" "F" "M" "M" "F" "M" "M" "M" "F" "M" ...}
\KeywordTok{table}\NormalTok{(wwlist}\OperatorTok{$}\NormalTok{sex_char)}
\CommentTok{#> }
\CommentTok{#>      F      M      U }
\CommentTok{#> 147434 120470    492}

\CommentTok{#create new variable that assigns levels}
\NormalTok{sex_fac <-}\StringTok{ }\KeywordTok{factor}\NormalTok{(wwlist}\OperatorTok{$}\NormalTok{sex_char, }\DataTypeTok{levels =} \KeywordTok{c}\NormalTok{(}\StringTok{"F"}\NormalTok{,}\StringTok{"M"}\NormalTok{,}\StringTok{"U"}\NormalTok{))}
\KeywordTok{str}\NormalTok{(wwlist}\OperatorTok{$}\NormalTok{sex_char)}
\CommentTok{#>  chr [1:268396] "M" "F" "M" "F" "F" "M" "M" "F" "M" "M" "M" "F" "M" ...}
\end{Highlighting}
\end{Shaded}

How the \hlgc{levels} argument works when underlying data is character

\begin{itemize}
\tightlist
\item
  Matches value of underlying data to value of the level attribute
\item
  Converts underlying data to integer, with level attribute attached
\end{itemize}

\medskip See chapter 15 of Wickham for more on factors (e.g., modifying
factor order, modifying factor levels)

\end{frame}

\begin{frame}{Substantial exercise on using/creating factors, using
either df\_school or df\_event datasets}

\end{frame}

\section{Labeling variables}\label{labeling-variables}

\section{Pipes}\label{pipes}

\begin{frame}{What are ``pipes'', \%\textgreater{}\%}

\textbf{Pipes} are a means of perfoming multiple steps in a single line
of code

\begin{itemize}
\tightlist
\item
  Pipes are part of \textbf{tidyverse} suite of packages, not
  \textbf{base R}
\item
  When writing code, the pipe symbol is \hlgc{\%>\%}
\item
  Basic flow of using pipes in code:

  \begin{itemize}
  \tightlist
  \item
    \hlgc{object \%>\% some\_function \%>\% some\_function, \ldots}\\
  \end{itemize}
\item
  Pipes work from left to right:

  \begin{itemize}
  \tightlist
  \item
    The object/result from left of \hlgc{\%>\%} pipe symbol is the input
    of function to the right of the \hlgc{\%>\%} pipe symbol
  \item
    In turn, the resulting output becomes the input of the function to
    the right of the next \hlgc{\%>\%} pipe symbol
  \end{itemize}
\end{itemize}

\end{frame}

\begin{frame}[fragile]{Do some tasks with and without pipes}

Print data for ``first-generation'' prospects

\begin{Shaded}
\begin{Highlighting}[]
\KeywordTok{filter}\NormalTok{(wwlist, firstgen }\OperatorTok{==}\StringTok{ "Y"}\NormalTok{)}
\NormalTok{wwlist }\OperatorTok\StringTok{ }\KeywordTok{filter}\NormalTok{(firstgen }\OperatorTok{==}\StringTok{ "Y"}\NormalTok{)}
\end{Highlighting}
\end{Shaded}

Comparing the two approaches:

\begin{itemize}
\tightlist
\item
  In the ``without pipes'' approach, the object is the first argument
  \hlgc{filter()} function
\item
  In the ``pipes'' approach, you don't specify the object as the first
  argument of \hlgc{filter()}

  \begin{itemize}
  \tightlist
  \item
    Why? Because \hlgc{\%>\%} ``pipes'' the object to the left of the
    \hlgc{\%>\%} operator into the function to the right of the
    \hlgc{\%>\%} operator
  \end{itemize}
\end{itemize}

Main takeaway:

\begin{itemize}
\tightlist
\item
  Whenever you write code using pipes, functions to the right of a
  \hlgc{\%>\%} pipe operator should not explicitly name the object that
  is the input to that function. Rather, the object to the left of the
  \hlgc{\%>\%} pipe operator is automatically the input.
\end{itemize}

\end{frame}

\begin{frame}[fragile]{Do some tasks with and without pipes}

Print data for ``first-generation'' prospects for selected variables
{[}output omitted{]}

\begin{Shaded}
\begin{Highlighting}[]
\KeywordTok{select}\NormalTok{(}\KeywordTok{filter}\NormalTok{(wwlist, firstgen }\OperatorTok{==}\StringTok{ "Y"}\NormalTok{), state, hs_city, ethn_code)}

\NormalTok{wwlist }\OperatorTok\StringTok{ }\KeywordTok{filter}\NormalTok{(firstgen }\OperatorTok{==}\StringTok{ "Y"}\NormalTok{) }\OperatorTok\StringTok{ }\KeywordTok{select}\NormalTok{(state, hs_city, ethn_code)}
\end{Highlighting}
\end{Shaded}

Comparing the two approaches:

\begin{itemize}
\tightlist
\item
  In the ``without pipes'' approach, code is written ``inside out''

  \begin{itemize}
  \tightlist
  \item
    The first step in the task -- identifying the object -- is the
    innermost part of code
  \item
    The last step in task -- selecting variables to print -- is the
    outermost part of code
  \end{itemize}
\item
  In ``pipes'' approach the left-to-right order of code matches how we
  think about the task

  \begin{itemize}
  \tightlist
  \item
    First, we start with an object \textbf{\emph{and then}}
    (\hlgc{\%>\%}) we use \hlgc{filter()} to isolate first-gen students
    \textbf{\emph{and then}} (\hlgc{\%>\%}) we select which variables to
    print
  \end{itemize}
\end{itemize}

\end{frame}

\begin{frame}[fragile]{Do some tasks with and without pipes}

Count the number ``first-generation'' prospects from the state of
Washington

\begin{Shaded}
\begin{Highlighting}[]
\KeywordTok{count}\NormalTok{(}\KeywordTok{filter}\NormalTok{(wwlist, firstgen }\OperatorTok{==}\StringTok{ "Y"}\NormalTok{, state }\OperatorTok{==}\StringTok{ "WA"}\NormalTok{))}
\CommentTok{#> # A tibble: 1 x 1}
\CommentTok{#>       n}
\CommentTok{#>   <int>}
\CommentTok{#> 1 32428}

\NormalTok{wwlist }\OperatorTok\StringTok{ }\KeywordTok{filter}\NormalTok{(firstgen }\OperatorTok{==}\StringTok{ "Y"}\NormalTok{, state }\OperatorTok{==}\StringTok{ "WA"}\NormalTok{) }\OperatorTok\StringTok{ }\KeywordTok{count}\NormalTok{()}
\CommentTok{#> # A tibble: 1 x 1}
\CommentTok{#>       n}
\CommentTok{#>   <int>}
\CommentTok{#> 1 32428}
\end{Highlighting}
\end{Shaded}

\end{frame}

\begin{frame}[fragile]{Do some tasks with and without pipes {[}last
example{]}}

Create frequency table of \hlgc{sex} for ``first-generation'' prospects
from WA

\begin{Shaded}
\begin{Highlighting}[]
\NormalTok{wwlist_temp <-}\StringTok{ }\KeywordTok{filter}\NormalTok{(wwlist, firstgen }\OperatorTok{==}\StringTok{ "Y"}\NormalTok{, state }\OperatorTok{==}\StringTok{ "WA"}\NormalTok{)}
\KeywordTok{table}\NormalTok{(wwlist_temp}\OperatorTok{$}\NormalTok{sex, }\DataTypeTok{useNA =} \StringTok{"ifany"}\NormalTok{)}
\CommentTok{#> }
\CommentTok{#>     F     M     U }
\CommentTok{#> 19080 13282    66}

\NormalTok{wwlist }\OperatorTok\StringTok{ }\KeywordTok{filter}\NormalTok{(firstgen }\OperatorTok{==}\StringTok{ "Y"}\NormalTok{, state }\OperatorTok{==}\StringTok{ "WA"}\NormalTok{) }\OperatorTok\StringTok{ }\KeywordTok{count}\NormalTok{(sex)}
\CommentTok{#> # A tibble: 3 x 2}
\CommentTok{#>   sex       n}
\CommentTok{#>   <fct> <int>}
\CommentTok{#> 1 F     19080}
\CommentTok{#> 2 M     13282}
\CommentTok{#> 3 U        66}
\end{Highlighting}
\end{Shaded}

Comparison of two approaches

\begin{itemize}
\tightlist
\item
  without pipes, task requires multiple lines of code; this is quite
  common

  \begin{itemize}
  \tightlist
  \item
    first line creates object; second line analyzes object
  \end{itemize}
\item
  with pipes, task can be completed in one line of code and you aren't
  left with objects you don't care about
\end{itemize}

Note: the pipes approach above is a useful way to show the
\textbf{values} associated with each \textbf{factor level} for factor
variables

\end{frame}

\begin{frame}{Student exercises with pipes}

CREATE STUDENT EXERCISES

\end{frame}

\section{Creating variables using
mutate}\label{creating-variables-using-mutate}

\begin{frame}{Our plan for learning how to create new variables}

Recall that the \hlgc{dplyr} package within the \hlgc{tidyverse} provide
a set of functions that can be described as ``verbs'':

\begin{itemize}
\tightlist
\item
  \textbf{subsetting}, \textbf{sorting}, and \textbf{transforming}
\end{itemize}

\begin{longtable}[]{@{}ll@{}}
\toprule
What we've done & Where we're going\tabularnewline
\midrule
\endhead
\textbf{Subsetting data} & \textbf{Transforming data}\tabularnewline
- \hlgc{select()} variables & - \hlgc{mutate()} creates new
variables\tabularnewline
- \hlgc{filter()} observations & - \hlgc{summarize()} calculates across
rows\tabularnewline
\textbf{Sorting data} & - \hlgc{group\_by()} to calculate across rows
within groups\tabularnewline
- \hlgc{arrange()} &\tabularnewline
\bottomrule
\end{longtable}

\textbf{Today}

\begin{itemize}
\tightlist
\item
  we'll use \hlgc{mutate()} to create new variables based on
  calculations across columns within a row
\end{itemize}

\textbf{Next week}

\begin{itemize}
\tightlist
\item
  we'll combine \hlgc{mutate()} with \hlgc{summarize()} and
  \hlgc{group\_by()} to create variables based on calculations across
  rows
\end{itemize}

\end{frame}

\begin{frame}[fragile]{Introduce \hlgc{mutate()} function}

\hlgc{mutate()} creates new columns (variblaes) that are functions of
existing columns

\begin{itemize}
\tightlist
\item
  \textbackslash{}hlgc\{mutate() is the \textbf{tidyverse} approach to
  creating variables, not the \textbf{Base R} approach
\item
  \textbackslash{}hlgc\{mutate() works best with pipes \hlgc{\%>\%}
\end{itemize}

We'll create variables from data frame \hlgc{df\_school} which has one
observation for each high school

\begin{itemize}
\tightlist
\item
  Task: create pct of students on free/reduced lunch (output omitted)
\end{itemize}

\begin{Shaded}
\begin{Highlighting}[]
\NormalTok{school_sml <-}\StringTok{ }\NormalTok{df_school }\OperatorTok\StringTok{ }\KeywordTok{filter}\NormalTok{(school_type }\OperatorTok{==}\StringTok{ "public"}\NormalTok{) }\OperatorTok\StringTok{ }
\StringTok{    }\KeywordTok{select}\NormalTok{(ncessch, num_fr_lunch, total_students)}

\NormalTok{school_sml }\OperatorTok\StringTok{ }\KeywordTok{mutate}\NormalTok{(}\DataTypeTok{pct_fr_lunch =}\NormalTok{ num_fr_lunch}\OperatorTok{/}\NormalTok{total_students)}
\end{Highlighting}
\end{Shaded}

Can combine \hlgc{(select())} with pipes \hlgc{\%>\%} to control which
columns printed, so no need to create dataset with fewer columns. But
let's create data frame with public high schools only

\begin{Shaded}
\begin{Highlighting}[]
\NormalTok{school_pub <-}\StringTok{ }\NormalTok{df_school }\OperatorTok\StringTok{ }\KeywordTok{filter}\NormalTok{(school_type }\OperatorTok{==}\StringTok{ "public"}\NormalTok{)}
\end{Highlighting}
\end{Shaded}

\end{frame}

\begin{frame}[fragile]{Introduce \hlgc{mutate()} function}

New variable not retained unless we \textbf{assign} \hlgc{<-} it to an
object (existing or new)

\begin{Shaded}
\begin{Highlighting}[]
\NormalTok{school_pub }\OperatorTok\StringTok{ }\KeywordTok{mutate}\NormalTok{(}\DataTypeTok{pct_fr_lunch =}\NormalTok{ num_fr_lunch}\OperatorTok{/}\NormalTok{total_students)}
\KeywordTok{names}\NormalTok{(school_pub)}

\NormalTok{school_pub_temp <-}\StringTok{ }\NormalTok{school_pub }\OperatorTok\StringTok{ }
\StringTok{  }\KeywordTok{mutate}\NormalTok{(}\DataTypeTok{pct_fr_lunch =}\NormalTok{ num_fr_lunch}\OperatorTok{/}\NormalTok{total_students) }
\KeywordTok{names}\NormalTok{(school_pub_temp)}
\end{Highlighting}
\end{Shaded}

How to create percent free/reduced lunch in \textbf{Base R}

\begin{Shaded}
\begin{Highlighting}[]
\NormalTok{school_pub}\OperatorTok{$}\NormalTok{pct_fr_lunch <-}\StringTok{ }\NormalTok{school_pub}\OperatorTok{$}\NormalTok{num_fr_lunch}\OperatorTok{/}\NormalTok{school_pub}\OperatorTok{$}\NormalTok{total_students}
\end{Highlighting}
\end{Shaded}

\hlgc{mutate()} can create multiple variables at once

\begin{Shaded}
\begin{Highlighting}[]
\NormalTok{school_pub }\OperatorTok\StringTok{ }
\StringTok{  }\KeywordTok{mutate}\NormalTok{(}\DataTypeTok{pct_fr_lunch =}\NormalTok{ num_fr_lunch}\OperatorTok{/}\NormalTok{total_students,}
         \DataTypeTok{pct_prof_math=}\NormalTok{ num_prof_math}\OperatorTok{/}\NormalTok{num_took_math) }\OperatorTok
\StringTok{  }\KeywordTok{select}\NormalTok{(num_fr_lunch, total_students, pct_fr_lunch, }
\NormalTok{         num_prof_math, num_took_math, pct_prof_math)}
\end{Highlighting}
\end{Shaded}

\end{frame}

\begin{frame}{Student exercise using mutate()}

\end{frame}

\begin{frame}[fragile]{Mutate to create indicator variables}

We often create dichotomous (0/1) indicator variables of whether
something happened (or whether something is TRUE)

\begin{itemize}
\tightlist
\item
  Variables that are of substantive interest to project

  \begin{itemize}
  \tightlist
  \item
    e.g., did student graduate from college
  \end{itemize}
\item
  Variables that help you investigate data, check quality

  \begin{itemize}
  \tightlist
  \item
    e.g., indicator of whether an observation is missing/non-missing for
    a particular variable
  \end{itemize}
\end{itemize}

Let's conduct some investigations of \hlgc{df\_school}, which has one
observation for each high school

Rename some variables (output omitted)

\begin{Shaded}
\begin{Highlighting}[]
\KeywordTok{str}\NormalTok{(df_school)}
\NormalTok{df_schoolv2 <-}\StringTok{ }\NormalTok{df_school }\OperatorTok\StringTok{ }
\StringTok{  }\KeywordTok{rename}\NormalTok{(}
    \DataTypeTok{visits_berkeley =}\NormalTok{ visits_by_}\DecValTok{110635}\NormalTok{,}
    \DataTypeTok{visits_boulder =}\NormalTok{ visits_by_}\DecValTok{126614}\NormalTok{,}
    \DataTypeTok{visits_bama =}\NormalTok{ visits_by_}\DecValTok{100751}\NormalTok{,}
    \DataTypeTok{state_berkeley =}\NormalTok{ inst_}\DecValTok{110635}\NormalTok{,}
    \DataTypeTok{state_boulder =}\NormalTok{ inst_}\DecValTok{126614}\NormalTok{,}
    \DataTypeTok{state_bama =}\NormalTok{ inst_}\DecValTok{100751}\NormalTok{)}

\KeywordTok{names}\NormalTok{(df_schoolv2)}
\end{Highlighting}
\end{Shaded}

\end{frame}

\begin{frame}[fragile]{Creating indicators for \hlgc{df\_schoolv2} data
frame}

Create TRUE/FALSE indicator that median household income greater than
\$50,000

\begin{Shaded}
\begin{Highlighting}[]
\NormalTok{df_schoolv2_temp <-}\StringTok{ }\NormalTok{df_schoolv2 }\OperatorTok\StringTok{ }\KeywordTok{mutate}\NormalTok{(}\DataTypeTok{incgt50k =}\NormalTok{ avgmedian_inc_}\DecValTok{2564}\OperatorTok{>}\DecValTok{50000}\NormalTok{)}

\NormalTok{df_schoolv2_temp }\OperatorTok\StringTok{ }\KeywordTok{select}\NormalTok{(avgmedian_inc_}\DecValTok{2564}\NormalTok{, incgt50k) }\OperatorTok\StringTok{ }\KeywordTok{head}\NormalTok{(}\DataTypeTok{n=}\DecValTok{3}\NormalTok{)}
\CommentTok{#> # A tibble: 3 x 2}
\CommentTok{#>   avgmedian_inc_2564 incgt50k}
\CommentTok{#>                <dbl> <lgl>   }
\CommentTok{#> 1              76160 TRUE    }
\CommentTok{#> 2              76160 TRUE    }
\CommentTok{#> 3                 NA NA}

\NormalTok{df_schoolv2_temp }\OperatorTok\StringTok{ }\KeywordTok{filter}\NormalTok{(}\KeywordTok{is.na}\NormalTok{(avgmedian_inc_}\DecValTok{2564}\NormalTok{)) }\OperatorTok\StringTok{ }\KeywordTok{count}\NormalTok{(incgt50k)}
\CommentTok{#> # A tibble: 1 x 2}
\CommentTok{#>   incgt50k     n}
\CommentTok{#>   <lgl>    <int>}
\CommentTok{#> 1 NA         624}
\end{Highlighting}
\end{Shaded}

Important takeaway:

\begin{itemize}
\tightlist
\item
  Variable created by \hlgc{mutate()} equals \hlgc{NA} for obs if input
  variable to \hlgc{mutate()} is missing for that obs. This is a good
  thing!
\end{itemize}

\end{frame}

\begin{frame}[fragile]{Creating indicators for \hlgc{df\_schoolv2} data
frame}

Create TRUE/FALSE indicator that school is less than 50 percent white

\begin{Shaded}
\begin{Highlighting}[]
\NormalTok{df_schoolv2_temp <-}\StringTok{ }\NormalTok{df_schoolv2 }\OperatorTok\StringTok{ }\KeywordTok{mutate}\NormalTok{(}\DataTypeTok{lt50pctwhite =}\NormalTok{ pct_white}\OperatorTok{<}\DecValTok{50}\NormalTok{)}
\NormalTok{df_schoolv2_temp }\OperatorTok\StringTok{ }\KeywordTok{select}\NormalTok{(pct_white,lt50pctwhite) }\OperatorTok\StringTok{ }\KeywordTok{head}\NormalTok{(}\DataTypeTok{n=}\DecValTok{3}\NormalTok{)}
\CommentTok{#> # A tibble: 3 x 2}
\CommentTok{#>   pct_white lt50pctwhite}
\CommentTok{#>       <dbl> <lgl>       }
\CommentTok{#> 1      11.8 TRUE        }
\CommentTok{#> 2       0   TRUE        }
\CommentTok{#> 3       0   TRUE}
\KeywordTok{str}\NormalTok{(df_schoolv2_temp}\OperatorTok{$}\NormalTok{lt50pctwhite)}
\CommentTok{#>  logi [1:21301] TRUE TRUE TRUE TRUE TRUE TRUE ...}
\end{Highlighting}
\end{Shaded}

Create 0/1 integer indicator rather than logical indicator

\begin{Shaded}
\begin{Highlighting}[]
\NormalTok{df_schoolv2_temp <-}\StringTok{ }\NormalTok{df_schoolv2 }\OperatorTok\StringTok{ }\KeywordTok{mutate}\NormalTok{(}\DataTypeTok{lt50pctwhite =} \KeywordTok{as.integer}\NormalTok{(pct_white}\OperatorTok{<}\DecValTok{50}\NormalTok{))}
\NormalTok{df_schoolv2_temp }\OperatorTok\StringTok{ }\KeywordTok{select}\NormalTok{(pct_white,lt50pctwhite) }\OperatorTok\StringTok{ }\KeywordTok{head}\NormalTok{(}\DataTypeTok{n=}\DecValTok{3}\NormalTok{)}
\CommentTok{#> # A tibble: 3 x 2}
\CommentTok{#>   pct_white lt50pctwhite}
\CommentTok{#>       <dbl>        <int>}
\CommentTok{#> 1      11.8            1}
\CommentTok{#> 2       0              1}
\CommentTok{#> 3       0              1}
\KeywordTok{str}\NormalTok{(df_schoolv2_temp}\OperatorTok{$}\NormalTok{lt50pctwhite)}
\CommentTok{#>  int [1:21301] 1 1 1 1 1 1 1 1 1 1 ...}
\end{Highlighting}
\end{Shaded}

\end{frame}

\begin{frame}[fragile]{Student exercises}

0/1 indicators of whether school received visit from each university

\begin{Shaded}
\begin{Highlighting}[]
\NormalTok{df_schoolv2 }\OperatorTok\StringTok{ }\KeywordTok{count}\NormalTok{(visits_berkeley)}
\CommentTok{#> # A tibble: 4 x 2}
\CommentTok{#>   visits_berkeley     n}
\CommentTok{#>             <int> <int>}
\CommentTok{#> 1               0 20732}
\CommentTok{#> 2               1   528}
\CommentTok{#> 3               2    36}
\CommentTok{#> 4               3     5}
\NormalTok{df_schoolv2_temp <-}\StringTok{ }\NormalTok{df_schoolv2 }\OperatorTok\StringTok{ }\KeywordTok{mutate}\NormalTok{(}\DataTypeTok{yesvis_berkeley =}\NormalTok{ visits_berkeley}\OperatorTok{>}\DecValTok{0}\NormalTok{)}

\NormalTok{df_schoolv2_temp }\OperatorTok\StringTok{ }\KeywordTok{filter}\NormalTok{(visits_berkeley}\OperatorTok{>}\DecValTok{0}\NormalTok{) }\OperatorTok\StringTok{ }\KeywordTok{select}\NormalTok{(visits_berkeley,yesvis_berkeley)}
\CommentTok{#> # A tibble: 569 x 2}
\CommentTok{#>    visits_berkeley yesvis_berkeley}
\CommentTok{#>              <int> <lgl>          }
\CommentTok{#>  1               2 TRUE           }
\CommentTok{#>  2               2 TRUE           }
\CommentTok{#>  3               2 TRUE           }
\CommentTok{#>  4               1 TRUE           }
\CommentTok{#>  5               1 TRUE           }
\CommentTok{#>  6               1 TRUE           }
\CommentTok{#>  7               1 TRUE           }
\CommentTok{#>  8               1 TRUE           }
\CommentTok{#>  9               1 TRUE           }
\CommentTok{#> 10               1 TRUE           }
\CommentTok{#> # ... with 559 more rows}
\NormalTok{df_schoolv2_temp }\OperatorTok\StringTok{ }\KeywordTok{filter}\NormalTok{(visits_berkeley}\OperatorTok{==}\DecValTok{0}\NormalTok{) }\OperatorTok\StringTok{ }\KeywordTok{select}\NormalTok{(visits_berkeley,yesvis_berkeley)}
\CommentTok{#> # A tibble: 20,732 x 2}
\CommentTok{#>    visits_berkeley yesvis_berkeley}
\CommentTok{#>              <int> <lgl>          }
\CommentTok{#>  1               0 FALSE          }
\CommentTok{#>  2               0 FALSE          }
\CommentTok{#>  3               0 FALSE          }
\CommentTok{#>  4               0 FALSE          }
\CommentTok{#>  5               0 FALSE          }
\CommentTok{#>  6               0 FALSE          }
\CommentTok{#>  7               0 FALSE          }
\CommentTok{#>  8               0 FALSE          }
\CommentTok{#>  9               0 FALSE          }
\CommentTok{#> 10               0 FALSE          }
\CommentTok{#> # ... with 20,722 more rows}
\end{Highlighting}
\end{Shaded}

\end{frame}

\begin{frame}[fragile]{Investigating \hlgc{wwlist} data frame}

?MAYBE CUT UNTIL NEXT LECTURE AND HAVE SECTION ON USING DESCRIPTIVE
STATS TO INVESTIGATE DATA?

\medskip Let's conduct some investigations of \hlgc{wwlist}, which is
frankly a pretty weird dataset!

\begin{itemize}
\tightlist
\item
  When conducting investigations, really important to be careful about
  missing values
\end{itemize}

\begin{Shaded}
\begin{Highlighting}[]
\KeywordTok{str}\NormalTok{(wwlist)}
\end{Highlighting}
\end{Shaded}

Variable \hlgc{receive\_date} indicates date prospect list data received
from College Board

\begin{Shaded}
\begin{Highlighting}[]
\NormalTok{wwlist }\OperatorTok\StringTok{ }\KeywordTok{count}\NormalTok{(receive_date)}
\CommentTok{#> # A tibble: 15 x 2}
\CommentTok{#>    receive_date     n}
\CommentTok{#>    <date>       <int>}
\CommentTok{#>  1 2016-05-31   50975}
\CommentTok{#>  2 2016-06-01   23195}
\CommentTok{#>  3 2016-06-02    4710}
\CommentTok{#>  4 2016-08-02    3929}
\CommentTok{#>  5 2016-08-03    5650}
\CommentTok{#>  6 2016-11-09    1178}
\CommentTok{#>  7 2017-01-23    1105}
\CommentTok{#>  8 2017-05-05   73430}
\CommentTok{#>  9 2017-06-05    3058}
\CommentTok{#> 10 2017-08-10   11396}
\CommentTok{#> 11 2017-10-11     827}
\CommentTok{#> 12 2018-01-17    2077}
\CommentTok{#> 13 2018-05-24   29756}
\CommentTok{#> 14 2018-06-20   53737}
\CommentTok{#> 15 2018-07-10    3373}
\end{Highlighting}
\end{Shaded}

\end{frame}

\begin{frame}[fragile]{other}

\begin{Shaded}
\begin{Highlighting}[]
\CommentTok{#num_took_math, num_prof_math}
\NormalTok{ww_narrow <-}\StringTok{ }\NormalTok{wwlist }\OperatorTok\StringTok{ }\KeywordTok{select}\NormalTok{(psat_range, }\KeywordTok{contains}\NormalTok{(}\StringTok{"gpa"}\NormalTok{), state, zip, for_country, ethn_code, homeschool, firstgen)}
\KeywordTok{load}\NormalTok{(}\StringTok{"../../data/recruiting/recruit_event_somevars.Rdata"}\NormalTok{)}
\end{Highlighting}
\end{Shaded}

\end{frame}

\end{document}
